\section{Ďalšie algoritmy}

Algoritmov na výpočet rýchlej Fourierovej transformácie je nespočetne
veľa. Mnohé z týchto algoritmov sú len úpravami už spomínaných
algoritmov, snažiace sa poväčšinou o efektívnejšiu implementáciu -
využitie symetrie komplexných čísel, používanie reálnych čísel tam kde
treba a mnohé ďalšie optimalizácie až po efektívne ukladanie v pamäti
kvôli rýchlosti procesorovej cache. Ďalšie sú síce efektívne, ale v
dobe čoraz rýchlejších počítačov schopných paralelných výpočtov ich
algoritmická zložitosť prevažuje výhody, ako je to napríklad u
Winogradovho algoritmu. V tejto kapitole si spomenieme
niektoré ďalšie zaujímavé myšlienky, ale nebudeme ich rozoberať
podrobne, iba načrtneme výsledky.

\subsection{Split radix}

\subsection{Good-Thomas algorithm}
