\section{Raderov algoritmus}
V poradí už druhý algoritmus, ktorý vie počítať Fourierovu
transformáciu prvočíselnej dĺžky. Narozdiel od Bluesteinovho algoritmu
ale vie počítať iba prvočíselné dĺžky. Základom bude napodiv algebra.
Začneme sadou tvrdení

\todo{Tieto definicie sharnut z aplikaciou - nasobenie polynomov}
\begin{definicia}[grupa]
\end{definicia}
\begin{lema}
    $\Z_p(.)$ je cyklická grupa.
\end{lema}
\begin{definicia}[generator]
\end{definicia}
\begin{lema}
    Každá cyklická grupa má generátor. Navyše, nech $g$ je generátor a
    $n$ je veľkosť grupy. Potom
    $g^0,g^1,g^2,\dots,g^{n-1}$ sú navzájom rôzne prvky grupy a teda
    existuje bijekcia medzi číslami $0,\dots,n-1$ a prvkami grupy.
\end{lema}

Nech $p$ je prvočíslo. Potom podľa lemy \todo{ref} $Z_p(.)$ je
cyklická grupa a využítím lemy \todo{ref} vidíme, že existuje bijekcia
medzi prvkami grupy (číslami $1,2,\dots,p-1$) a číslami $0,1,\dots,p-2$.
Ak teda v rovnici \ref{eq:dft_omega} vyčleníme špeciálne 0 zo sumy,
dostávame
\begin{align}
    X_0 &= \sum_{l=0}^{p-1} x_l \omega_{p,0l} = \sum_{l=0}^{p-1} x_l \\
    X_{g^{-k}} &= \sum_{l=0}^{p-1} x_l \omega_{p,(g^{-k}) l} \\
            &= 
        x_0 + \sum_{l=1}^{p-1} x_l \omega_{p,(g^{-k}) l} \\
        &=
        x_0 + \sum_{g^j, g^j \in Z_p} x_{(g^j)} \omega_{p, (g^{-k})
        (g^j)} \\
        &= x_0 + \sum_{j \in 0,1,\dots,p-2} x_{g^j} \omega_{p,
        g^{j-k}}, \quad k \in 0,1,\dots,p-2
\end{align}
Označme si postupnosti
\begin{align}
    A_k &= x_{g^k} \\
    B_k &= \omega_{p, g^{-k}}
\end{align}
Pretože $g^{-(p-x)} = g^{x - p} = g^{x}$, môžeme písať
\begin{equation}
    X_{g^{-k}} -x_0=  \sum_{j=0}^{p-2} A_j B_{k-j}
\end{equation}
Teraz už môžeme vidieť jasnú cyklickú konvolúciu o veľkosti $p-2$.
Túto konvolúciu môžeme vypočítať buď priamo aplikovaním konvolučnej
vety \todo{ref} a spočítaním fourierových transformácii postupností
$A,B$. Pohodlnejšie je ale, podobne ako u Bluesteinovho algoritmu,
doplniť postupnosti $A,B$ na dĺžku veľkosti mocniny dva a použiť
niektorý z rýchlych algoritmov ktoré sme spomínali.
Poslednou nepríjemnosťou je index $g^{-k}$, kde by sme radšej videli
$k$. Ako sme ale spomínali, existuje bijekcia medzi týmito dvoma
množinami.

\todo{pokec o generatoroch / primitive roots, ako ich hladat, ako
rychlo umocnovat - vyuzije sa dalej v rychlom nasobeni}

\input code/rader
\todo{lit:wiki:rader}
