\section{Brunnov algoritmus}

Hoci je tento algoritmus nie veľmi používaný a menej numericky
stabilný ako doteraz popísané algoritmy, predsa ho len uvádzame.
Dôvod je jednoduchý - skrýva v sebe nápaditú a zaujímavú myšlienku,
ktorá si zaslúži pozornosť. Algoritmus využíva prístup založený na
polynómoch. Najskôr si však zhrnieme dôležité poznatky

\begin{lema}
    Pre polynón $x^k$ premennej $x$ a ľubovoľný konštantu $/alpha$
    platí
    \begin{equation}
        x^k \equiv \alpha^k \quad \mod x-\alpha
    \end{equation}
    kde $\mod$ je klasický zvyšok po delení polynómov
    \label{lema:polymod}
\end{lema}
\begin{dokaz}
    Spomeňme si na školský algebraický príklad
    \begin{equation}
        (x^k - y^k) = (x - y) (x^{k-1} + x^{k-2} y + \dots 
            + x y^{k-2} + y^{k-1})
    \end{equation}
    Potom zjavne
    \begin{equation}
        (x^k - \alpha^k) = (x - \alpha) * (\dots) \equiv 0 
        \quad mod x-\alpha
    \end{equation}
    čo ľahko upravíme na finálne tvrdenie
\end{dokaz}

Zopakujme si rovnicu pre diskrétnu Fourierovu transformáciu
\begin{equation}
    X_k = \sum_{n=0}^{N-1} x_n e^{-\frac{2\pi}{N}nk}
\end{equation}
Môžeme si ju predstaviť ako polynóm v premennej $e^{-\frac{2\pi}{N}k}$.
Teda
\begin{equation}
    X(z) = \sum_{n=0}^{N-1} x_n z^n
\end{equation}
Kde za $z$ postupne dosadíme
\begin{equation}
    \omega_k= e^{-\frac{2\pi}{N}k}, \quad k=0,1,\dots,N-1
\end{equation}
Podľa lemy \ref{lema:polymod} aplikovanej na súčet je vypočítanie
hodnoty polynómu v bode $t$ rovné zvyšku po delení $z-t$.
Celá úloha sa teda zmenila na $N$ počítaní zvyškov po delení polynómu
$X(z)$ polynómami $z-\omega_k, k=0,1,\dots,N-1$. Na prvý pohľad
rovnako ťažká úloha. Brunn sa ale nedal zastaviť a prišiel na
nasledujúci spôsob. Predpokladajme najskôr, že chceme počítať
diskrétnu Fourierovu transformáciu dĺžky ktorá je mocninou 2.
Existuje zovšeobecnenie aj pre iné zložené čísla, ale kvôli svojej
zložitosti si nenašlo cestu do tejto publikácie.
Základná myšlienka na ktorej budeme stavať je
\begin{lema}
    Nech $U,V$ sú polynómy. Potom pre všetky polynómy $X$ platí
    \begin{align}
        X \mod U &= (X \mod UV) \mod U \\
        X \mod V &= (X \mod UV) \mod V
    \end{align}
\end{lema}
\begin{dokaz}
    Dôkaz prenecháme ako jednoduché cvičenie čitateľovi.
\end{dokaz}
Nasleduje ďalšie pozorovanie, a to konkrétne
\begin{lema}
    $z^n - 1 = (z - \omega_0)(z-\omega_1)\dots(z-\omega_{n-1})$
\end{lema}
\begin{dokaz}
    $\forall k \in \{0,1,\dots,n-1\}: \omega_k$ je koreňom rovnice
    $z^n-1=0$ ako sa ľahko môžeme presvedčiť. Podľa základnej vety
    algebry, rovnica $n$-tého stupňa má najviac $n$ koreňov. Potom
    ale čísla $\omega_k$ musia byť nutne všetky korene $z^n-1$
    a teda $z^n-1$ sa dá zapísať tak ako bolo napísané.
\end{dokaz}
Celý Brunnov algoritmus je založený na šikovnej faktorizácii tohoto
polynómu. My si ho uvedieme pre mocniny dvojky, avšak existuje
všeobecnejšia verzia ktorá vie riešiť aj iné zložené čísla.
Všimnime si nasledujúcu lemu.
\begin{lema}
    \begin{align}
        z^{2k} -1 = (z^k - 1)( z^k + 1) \\
        z^{2k} + \alpha z^k + 1 = ( z^k + \beta z^{k/2} + 1)
            ( z^k - \beta z^{k/2} + 1), \quad
            \text{ pre }\alpha \in [-2,2], k\ge 2
    \end{align}
    kde $\beta = \sqrt{2-\alpha} \in [-2,2]$
\end{lema}
\begin{dokaz}
    Roznásobením pravých strán priamo dostávame pravé strany. V
    prípade druhej rovnice využívame navyše fakt $\alpha \in [-2,2]$
    aby sme mohli
\end{dokaz}
Jej pozoruhodnou vlastnosťou je, že ak vezmeme $n$ mocninu dvojky,
vieme $z^n-1$ postupne faktorizovať na polynómy polovičného stupňa,
každý s nanajvýš troma nenulovými členmi, a po $\log n$ fázach sa
dopracujeme na polynómy prvého stupňa, ktoré sú presne $z-\omega_k, 
k\in0,1,\dots,n-1$. 
Vráťme sa ale späť k algoritmu. Algoritmus začne vygenerovaním
polynómu $X(z)$. Následne vypočítame $X_{n,0} =X(z) \mod z^n -1$.
Pokračujeme $X_{n/2,0} = X_{n,0} \mod z^{n/2}-1$ a 
$X_{n/2,1} = X_{n,0} \mod z^{n/2}+1$. Je zjavné ako budeme pokračovať
ďalej, až dospejeme do stavu kedy sú polynómy kvadratické. Teraz nám
už nič nebráni vypočítať ich korene a pomocou lemy \ref{lema:polymod}
spočítať finálne výsledky $X_k, k\in 0,1,\dots,n-1$. Za pozornosť
stoja dve zmienky. Prvou je algoritmická zložitosť
$X_{2k,i} \mod z^{2k} + \alpha z^k + \beta$. Vďaka tomu, že polynóm,
ktorým delíme pozostáva len z troch členov, vieme ním deliť v
lineárnom čase od veľkosti prvého polynómu, teda v čase $O(k)$. V
$i$-tej fáze preto vykonáme $i$ delení o veľkosti $\frac{n}{2^i}$ a
$2i$ faktorizácii o konštantnom čase. Spolu teda lineárny čas pre
jednu fázu, fáz je $\log n$ a teda algoritmus má časovú
zložitosť $O(n \log n)$.
Druhom zmienkou je vec, ktorú sme doposiaľ zamlčali. Síce máme
vypočítané všetky hodnoty $X_k$, ale nevieme v akom poradí.
V tom sa nám hodí nasledujúca lema, pomocou ktorej vieme v každom
ktoku rekurzie preusporiadať hodnoty tak aby sme na konci dostali
$X_0,X_1,\dots,X_{n-1}$.
\begin{lema}
Nech $n\ge4$ je mocnina dvojky.
\begin{itemize}
\item
    Nech $x_0,\dots,x_{n/2}$ sú korene $z^{n/2} -1$ a
    $y_0,\dots,y_{n/2}$ sú korene $z_{n/2}+1$. Navyše, nech sú
    postupnosti $x_i, y_i$ usporiadané vzostupne podľa \todo{uhla}.
    Potom korene rovnice $z^n -1$ usporiadané podľa \todo{uhla} sú
    $x_0,y_0,x_1,y_1,\dots,x_{n-1},y_{n-1}$.
\item
    Podobne nech
    $x_0,\dots,x_{n/2}$ sú korene $z^{n/2} -\beta z^{n/4} +1$ a
    $y_0,\dots,y_{n/2}$ sú korene $z_{n/2} +\beta z^{n/4} +1$. Ako v
    predchádzajúcom prípade, nech sú
    postupnosti $x_i, y_i$ usporiadané vzostupne podľa \todo{uhla}.
    Potom korene rovnice $z^n +\alpha +1$ kde $\alpha,\beta$ sú v
    \todo{relacii ako v ref + todo a predpokladame ze alpha je
    dosiahnutelna rozkladom!!!} usporiadané podľa \todo{uhla} sú
    $x_0,y_0,y_1,x_1, \, x_2,y_2,y_3,x_3, \, \dots,
     x_{n/2-2},y_{n/2-2},y_{n/2-1},x_{n/2-1}$.

\end{itemize}
\end{lema}
\begin{dokaz}
Prvá časť dôkazu je jednoznačňe ľahšia. Môžeme sa ľahko presvedčiť, že korene
$z^{n/2}-1$ sú $\omega_{n,2i}$ a korene $z^{n/2}+1$ sú
$\omega_{n,2i+1}$ kde $i\in 0,1,\dots, n/2-1$. Navyše vieme, že spolu
tieto korene sú všetky korene polynómu $z^n-1$. Tvrdenie je potom
zjavné nakoľko uhol $\omega_{n,i}$ závisí lineárne od $i$.

Druhá časť dôkazu je väčší oriešok. Použijeme matematickú indukciu.
Predpokladajme na začiatku $n=4$.
Máme teda rovnicu
$z^4 + \alpha z^2 + 1 = (z^2 - \beta z + 1)(z^2 + \beta + 1)$.
Korene $x_0,x_1$ prvého polynómu pravej strany musia byť komplexne
združené, nakoľko predpokladáme $\beta\in[-2,2]$. To isté platí
aj o koreňoch $y_0,y_1$. Ak predpokladáme, že všetky 4 korene sú
rôzne, čo si môžeme dovoliť z predpokladu že daný polynóm vznikol
faktorizáciou $z^{n 2^m} -1$ pre nejaké $m\in\Nz$,
sú len 2 možnosti ich usporiadania -
buď $x_0,y_0,y_1,x_1$ alebo $y_0,x_0,x_1,y_1$ (Necháme na čitaťeľa si
premyslieť prečo). Stačí teda ukázať, že $\arg x_0 < \arg y_0$.
Jednoduchým výpočtom dostávame 
\begin{align}
    x = \frac{1}{2} (\beta \pm \sqrt{\beta^2 - 4})
    y = \frac{1}{2} (-\beta \pm \sqrt{\beta^2 - 4})
\end{align}
Nahliadnutím na štruktúru daných komplexných čísel a fakt $\beta\ge0$
ľahko vidíme, že $x$ má riešenie v prvom kvadrante zatiaľ čo $y$ až v
druhom. Tvrdenie pre $n=4$ sme dokázali.
Predpokladajme teraz, že tvrdenie platí pre nejaké $n\ge4$, ktoré je
mocninou dvojky.
Potom pre $2n$ platí nasledovné:
Korene $z^n - \beta z^{n/2} + 1$ sú $\sqrt{x_0},\sqrt{x_1},\sqrt{x_2},
\dots,\sqrt{x_{n/4-1}}$ a $ -\sqrt{x_0}, -\sqrt{x_1}, \dots,
-\sqrt{x_{n/4-1}}$. Navyše, vo vypísanom poradí sú aj zoradené podľa
uhla. Dôkaz je jednoduchý - odmocnením sme zmenili uhol u každého
koreňa na polovicu. Navyše platí $\arg -\sqrt{x}=\arg \sqrt{x} + \pi$.
Týmto je zabezpečená rastúcosť argumentu pre druhú polovicu koreňov.
A konečne, 
\begin{equation}
    \arg \sqrt{x_{n/4-1}} < \pi \le \arg -\sqrt{x_0}
    \label{eq:ineq_minus_sqrt}
\end{equation}
Podobný argument platí aj pre hodnoty $y$. Potom použitím indukčného
predpokladu na hodnoty 
$\sqrt{x_0},\dots \sqrt{x_{n/4-1}},
 \sqrt{y_0},\dots \sqrt{y_{n/4-1}}$ a
$-\sqrt{x_0},\dots -\sqrt{x_{n/4-1}},
 -\sqrt{y_0},\dots -\sqrt{y_{n/4-1}}$ plus
 využitím nerovnosti \ref{eq:ineq_minus_sqrt} dostávame požadované
 tvrdenie.
\end{dokaz}

Na záver bez dôkazu uvedieme rozklad polynómov pre všeobecný prípad.
\begin{veta}
\begin{equation}
    \Phi_{n,a}(z) = \left\{
        \begin{array}{l l}
            z^{2n} - 2\cos(2\pi a) z^n +1 & 0<a<1 \\
            z^{2n} -1 & 0=a
        \end{array}
    \right.
\end{equation}
Potom pre ľubovoľné $r\in\N$ také, že $r$ je deliteľom $n$ platí
\begin{equation}
    \Phi_{n,a}(z) = \left\{
    \begin{array}{l l}
        \Pi_{k=0}^{r-1} \Phi_{n/r, (a+k)/r} & 0<a\le 0.5 \\
        \Pi_{k=0}^{r-1} \Phi_{n/r, (1-a+k)/r} & 0.5<a<1 \\
        \Pi_{k=0}^{r-1} \Phi_{n/r, k/(2r)} & a=0

    \end{array}
    \right.
\end{equation}

\end{veta}

\input code/brunn
