\section{Radix FFT}

Pretým, než si ukážeme tento algoritmus, objavíme najskôr
krásu motýlích krídel\footnote{Z anglického názvu ``Butterfly''}.
To, čo robí štandardný DFT algoritmus pomalým, je jeho ignorancia.
Ignoruje totiž 2 veľmi dôležité fakty o komplexných koreňoch $\omega$,
ktorými sú
\begin{itemize}
 \item Symetria: $\omega_N^{k+N/2} = -\omega_N^k$, ak je $N$ párne
 \item Periodicita: $\omega_n^{k+N} = \omega_N^k$
\end{itemize}
Ich všímavým pozorovaním môžeme dospieť napríklad k nasledujúcemu
faktu:
\begin{equation}
\begin{split}
X_k & = \sum_{n=0}^{N-1} x_n \omega_N^{kn}, \qquad k=0,1,\dots,N-1 \\
    & = \sum_{n \rm{\:je\:párne}} x_n \omega_N^{kn}
      + \sum_{n \rm{\:je\:nepárne}} x_n \omega_N^{kn} \\
    & = \sum_{n=0}^{\lceil N/2 \rceil-1} x_{2n} \omega_N^{2kn} 
      + \sum_{n=0}^{\lfloor N/2 \rfloor-1} x_{2n+1} \omega_N^{k(2n+1)}
\end{split}
\end{equation}
V ďalšom texte budeme automaticky uvažovať prípad, že N je párne.
Využijeme rovnosť $\omega_N^2 = \omega_{N/2}$ a výsledok upravíme na
tvar
\begin{equation}
\begin{split}
X_k & = \sum_{n=0}^{(N/2)-1} x_{2n} \omega_{N/2}^{kn} 
      + \omega_N^k \sum_{n=0}^{(N/2)-1} x_{2n+1} \omega_{N/2}^{kn}
\end{split}
\end{equation}
Nech $y_n=x_{2n}, \quad z_n=x_{2n+1}$. Potom $X_k = Y_k + \omega_N^k
Z_k$, kde $X_k, Y_k$ sú $N/2$ bodové diskrétne Furierove transformácie
postupností $x$, resp. $y$.
