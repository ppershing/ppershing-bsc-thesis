% vim:spell spelllang=sk
\section{Bluesteinov algoritmus}

Už sme si ukázali rýchlu Fourierovu transformáciu na
mocninách dvojky a na zložených číslach. Čo nám ale stále
prekáža sú prvočísla, pre ktoré zatiľ nevieme rýchly algoritmus.
Jedným z algoritmov, ktorý pracuje na vstupe ľubovoľnej veľkosti
je práve Bluesteinov (tiež zvaný Chirp Z-transform) algoritmus.
Jeho výhodou je, že je všeobecnejší ako fourierova transformácia,
a to menovite vie rátať Z-transformácie.

\def\Ztransform{\mathcal{Z}}

\begin{definicia}[(Jednostranná) z-transformácia]
    Nech $x_n, n\in \Nz$ je postupnosť komplexných čísel a nech 
    $z\in\C$.
    Potom
    \begin{equation}
        X(z) = \Ztransform\{x[n]\} =
        \sum_{n=0}^{\infty} x_n z^{-n}
    \end{equation}
    Nazveme z-transformáciou postupnosti $x_n$.
\end{definicia}

Naším záujmom bude vypočítať z-transformáciu v $m$ po seme idúcich
bodoch.
Viac formálne, zaujíma nás výpočet
\begin{equation}
    X_k = \sum_{i=0}^{n-1} x_i z^{ik}, \quad k = 0,1,\dots,m-1
\end{equation}
\begin{poznamka}
    Je dobré si uvedomiť, že vypočítanie tej istej transformácie
    s $z_{\alpha + ik}$ je to isté ako násobenie výsledku
    $z_\alpha$ a teda vieme jednoducho riešiť aj tento všeobecnejší
    prípad.
\end{poznamka}

Celý algoritmus sa opiera o nasledujúci trik. Pre ľubovoľné čísla
$i,k$ platí identita
\begin{equation}
    ik = \frac{k^2}{2} + \frac{i^2}{2} - \frac{(i-k)^2}{2}
\end{equation}
Dosadením identity do rovnice a upravením dostávame
\begin{equation}
    X_k = z^{k^2/2}
        \sum_{i=0}^{n-1} 
            \left(x_i z^{i^2/2} \right)
            z^{-(i-k)^2/2}
        , \quad k = 0,1,\dots,m-1
\end{equation}
Na prvý pohľad sme nič nezjednodušili. To nie je celkom pravda,
pretože daná suma je konvolúcia. Presnejšie povedané nech
\begin{align}
    a_i &= x_i z^{i^2/2} \\
    b_i &= z^{-i^2/2} \\
    c_i &= z^{i^2/2}
\end{align}
Potom vieme danú sumu zapísať ako
\begin{equation}
    X_k = c_k \sum_{i=0}^{n-1} a_i b_{k-i}
\end{equation}
Zaťiaľ sme si veľmi nepomohli. Konvolúciu síce vieme spočítať pomocou
rýchlej Fourierovej transformácie, ale sme na mŕtvom bode, pretože
je opäť prvočíselnej veľkosti a navyše nie je cyklická. 
Tu ale prichádza k slovu hlavná pointa.
Konvolúciu nemusíme robiť o veľkosti $n$.
Presnejšie, nech $N\ge m+n -1$.
Rozšírme postupnosti $a_i,b_i$ nasledovne:
\begin{align}
    a_i &=0, \quad n\le i<N \\
    b_{N-i} &= b_{i}, \quad 0 < i < N \\
    b_i &= 0, \quad n\le i \le N-n
\end{align}
Potom
\begin{equation}
    X_k = c_k \sum_{i=0}^{N-1} a_i b_{k-i \mod N},
        \quad k = 0,1,\dots,m-1
\end{equation}
je cyklická konvolúcia, ktorú vieme počítať pomocou FFT.
Navyše, pokiaľ spĺňame podmienku $N\ge m+n-1$, $N$ môže byť ľubovoľné
číslo. Veľmi vhodnou voľbou je $N$ ako mocnina dvojky, v tomto prípade
$N < 2(m+n-1)$ a pomocou split-radix algoritmu vieme spraviť
konvolúcie v čase $O(N \log N)$.
Výsledok konvolúcie potom vieme upraviť v lineárnom čase na požadovaný
výstup. Preto časová zložitosť Bluesteinovho algoritmu je
$O( (m+n) \log (n+m))$.

\todo{uhladit kod}
\todo{vycisti tento kod - zaved staticke metody + vyuzi split-radix}
\begin{python}

import dct;
import cmath;

class Ztransform:
    def nextPowerOfTwo(self, x):
        i = 0
        x -= 1 # for power of two, we want same result
        while x > 0:
            i += 1
            x /= 2

        x = 1
        while i > 0:
            i -=1
            x *= 2
        return x
        
    def ztransform(self, data, z, m):
        n = len(data)
        N = self.nextPowerOfTwo(n+m-1)

        a = []
        b = []
        
        # prepare a
        for i in range(n):
            a.append(data[i] * z ** (0.5 * i * i))

        for i in range(N-n):
            a.append(0)

        # prepare b
        for i in range(0,m):
            b.append(z ** (-0.5 * i * i))

        for i in range(N-m-n):
            b.append(0)

        for i in range(n):
            k = n - i
            b.append(z ** (-0.5 * k * k))

        # do convolution
        tmp = dct.DFT();
        a = tmp.transform(a)
        b = tmp.transform(b);

        for i in range(N):
            a[i] *= b[i]

        a = tmp.inverseTransform(a)

        #prepare result
        result = []

        for i in range(m):
            result.append(a[i] * z ** (0.5 * i * i))

        return result

    def fft_transform(self, data):
        return self.ztransform(data, 
                               cmath.exp( -2* 1j* cmath.pi / len(data)),
                               len(data))

    def fft_inverse_transform(self, data):
        return self.ztransform(data, 
                               cmath.exp( 2* 1j* cmath.pi / len(data)),
                               len(data))

\end{python}


Bluesteinov algoritmus ako vidíme vie počítať Fourierovu transformáciu
pre prvočíselné dĺžky. Vie však toho viac. Jedným z jeho možných
použití je spektrálna analýza úzkeho pásu frekvencií.
Medzi ďalšie použitie platí napríklad analýza signálu s lineárne sa
meniaciou frekvenciou. Takéto signály sa vyskytujú pri fyzikálnom
pohybe vysielača a prijímača, ako sú napríklad satelity alebo analýza
signálov prichádzajúcich z vermíru. Viac o tomto použití sa dá dočítať
na konci
\cite{nasa}.


\todo{lit: http://www.economicexpert.com/a/Bluestein:s:FFT:algorithm.htm}
\todo{lit: http://en.wikipedia.org/wiki/Z-transform}
\todo{lit na dalsie studovanie + je tam aj rader a zrejme aj ine:}
%http://books.google.com/books?id=wzYuOF6HFX0C&pg=PA350&dq=bluestein+chirp+z+transform+algorithm&lr=&ei=D0u9Sf6mHJvuzQSQvei7Dw
\todo{lit o z transform:}
%http://books.google.com/books?id=SmDImt1zHXkC&pg=RA1-PA134&lpg=RA1-PA134&dq=chirp+z-transform+algorithm&source=bl&ots=5-xQ5-Zw9Q&sig=cI_bGlp6FJONbUvsrmv5ScvkLf4&hl=en&ei=mwm9SZupNIupsAbN_snZAw&sa=X&oi=book_result&resnum=5&ct=result

