% vim:spell spelllang=sk
\section{História}
Za úplný začiatok Fourierovej transformácie sa dá považovať rok 1768,
kedy sa narodil francúzsky matematik a~fyzik Jean Baptiste Joseph
Fourier v meste Auxerre. Jeho prínos, ktorý sa stretol s najväčšou
odozvou možno nájsť v~memoári z r. 1807. Neskôr bol publikovaný 
v~knihe \emph{La Théorie Analitique de la Chauleur} (Analytická teória
tepla), ktorá uzrela svetlo sveta v~roku 1822.
Zjednodušene, Fourierov prínos matematike bol v~pozorovaní, že každá
periodická funkcia s~istými nie veľmi obmedzujúcimi predpokladmi
môže byť vyjadrená ako kombinácia sínusov a~kosínusov s~rôznymi
frekvenciami a~amplitúdami (v dnešnej dobe sa táto suma volá Fourierov
rad). Nie je dôležité, ako komplikovaná daná funkcia je, ak spĺňa
matematické základy, môže byť vyjadrená ako spomínaný rad. Možno sa to
nezdá ako veľmi podivné, avšak v~dobe keď Fourier prišiel s~touto
myšlienkou to bolo prelomové a~neintuitívne a~tak sa Fourierovo dielo
zo začiatku stretlo so skepticizmom.

Navyše aj funkcie ktoré nie sú periodické, ale spĺňajú isté
matematické podmienky, sa dajú zapísať ako integrál sínusov a~kosínusov.
Formulácia v~tejto podobe sa nazýva Fourierova transformácia a~jej
využitie je ešte významnejšie ako využitie Fourierovho radu. Obe
reprezentácie ale zdieľajú spoločné známky, a~to síce, že pôvodné
funkcie vieme zrekonštruovať inverznou transformáciou a~to v~presne
tej istej podobe ako sa vyskytli. Toto nám umožňuje riešiť úlohu 
v~inej doméne ako bola pôvodne formulovaná, mnohokrát jednoduchšej, 
a~riešenie preniesť do pôvodnej domény. Konieckoncov - Fourierova
transformácia vznikla z~praktickej potreby matematiky počítať isté
úlohy a~toto ju učinilo dobre skúmanou oblasťou a~výborným nástrojom.

Začalo to pôvodnou úlohou o~difúzii tepla, ktorú Fourier formuloval 
a~vyriešil v~spomínanej publikácii. Za svoje roky však použitie
transformácie prešlo rôznymi podobami a~v~dnešnej dobe je snáď
najväčším použitím diskrétna Fourierova transformácia, ktorá sa
používa v~počítačovom spracovaní signálu, obrázkov a~zvuku.
