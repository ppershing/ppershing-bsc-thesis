% vim:spell spelllang=sk
\section{DCT}

Jednou z podobných transformácii k Fourierovej je diskrétna kosínová
transformácia, ktorú sme si uviedli v \todo{}. Ako uvidíme ďalej, táto
transformácia hrá dôležitú rolu v aplikáciach pri kompresii audia a
videa. Preto stojí za zmienku uistiť sa, že existuje efektívny
algoritmus na jej výpočet. V skutočnosti, väčšina algoritmov
popísaných v tejto práci sa dá s menšími či väčšími problémami
prerobiť na DCT. Preto si tieto algoritmy a ich úpravy nebudeme
uvádzať. To, čo si uvedieme je výpočet DCT priamo pomocou FFT s
predpracovaním údajov. Potom bude zrejmé, že DCT vieme počítať podobne
ako FFT (len s väčšou konštantou) v čase $O(n \log n)$,
čo bude pre nás postačujúce.
Začnime rovnicou pre DCT:
\begin{equation}
    X_k = \sum_{l=0}^{n-1} x_l \cos\left(
        \frac{\pi}{n} (l + \frac{1}{2}) k
        \right)
\end{equation}
ktorú upravíme do podoby
\begin{equation}
    X_k = \sum_{l=0}^{n-1} x_l \cos\left(
        \frac{\pi}{2n} (2l + 1) k
        \right)
\end{equation}

Pozrime sa teraz na Fourierovu transformáciu
\begin{equation}
    Y_p = \sum_{q=0}^{4n-1} y_q e^{-\frac{2\pi\imag}{4n} p q} =
          \sum_{q=0}^{4n-1} y_q \left(
                \cos ( \frac{\pi}{2 n} p q )
                + \imag \sin ( \frac{\pi}{2n} p q)
                \right)
\end{equation}
Našim cieľom je nájsť súvislosť medzi týmito dvoma zápismi. V prvom
rade, pri Fourierovej transformácii máme navyše imaginárnu časť,
ktorej by sme sa radi zbavili. Podobne ako vo Fourierových radoch
\todo{}, sínusové koeficienty vieme jednoducho vynulovať použitím
párnej symetrie. Presne povedané nech $y_q = y_{4n-q}, 0<q<4n$.
Potom
\begin{equation}
    Y_p =    \sum_{q=0}^{4n-1} y_q \left(
                \cos ( \frac{\pi}{2 n} p q )
                +  0 \imag 
                \right)
\end{equation}
Rovnice \todo{ref,ref} sa hneď viacej podobajú, stále ešte máme čo
upravovať. Vieme napríklad využiť symetriu kosínusu a upraviť $Y$ na
tvar

\begin{equation}
    Y_p =  y_{2n} \cos( \pi pq) - y_0 + 2 \sum_{q=0}^{2n-1} y_q 
                \cos ( \frac{\pi}{2 n} p q )
\end{equation}
Teraz je už ale jasné, kam vietor veje. Položme
\begin{align}
    y_{2l} &= 0, \quad & l\in 0,1, \dots, n-1
    y_{2l+1} &= x_l, \quad & l \in 0,1, \dots, n-1
\end{align}
Potom
\begin{equation}
    Y_p = 2 \sum_{l=0}^{n-1} y_l 
                \cos ( \frac{\pi}{2 n} p l ) = 2 X_p
\end{equation}
Diskrétnu kosínovú transformáciu teda vieme dostať ako
FFT vstupných dát, ktorým pridáme symetriu a na párne miesta vložíme
nuly.
\begin{python}
from dct import DFT,DCT

class DCTUsingFFT:
    def transform(data):
        temp = []
        n = len(data)
        for i in range(n):
            temp.append(0)
            temp.append(data[i])

        for i in range(n):
            temp.append(0)
            temp.append(data[n-1-i])

        tmp = DFT();
        result = tmp.transform(temp)
        result[n:]=[]
        return map(lambda x:x.real/2, result)
    transform = staticmethod(transform)
\end{python}




