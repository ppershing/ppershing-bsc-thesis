% vim:spell spelllang=sk
\section{Úvod}

Zopakujme si poznatky o diskrétnej Fourierovej transformácii.
\begin{definicia}
    Diskrétnou Fourierovou transformáciou postupnosti $x_0, \dots,
    x_{n-1}$ budeme označovať postupnosť $X_0, \dots, X_{n-1}$
    definovanú nasledovne
    \begin{equation}
        X_k = \sum_{l=0}^{n-1} x_l e^{-\frac{2\pi\imag}{n}kl}
        \label{eq:dft}
    \end{equation}
\label{def:dft}
\end{definicia}
Ak označíme \todo{roots of unity} ako
\begin{equation}
    \omega_{n,k} = e^ { -\frac{2\pi \imag}{n} k}
    \label{eq:omega}
\end{equation}
možeme predchádzajúcu definíciu prepísať na

\begin{equation}
    X_k = \sum_{l=0}^{n-1} x_l \omega_{n,k}^l =
          \sum_{l=0}^{n-1} x_l \omega_{n,kl}
    \label{eq:dft_omega}
\end{equation}

Pokračovať v tejto kapitole budeme najskôr najjednoduchšími
algoritmami pre počítanie fourierovej transformácie veľkosti mocnín
dvojky. Potom prejdeme na všeobecný prípad zložených čísel a následne
prvočísel. Kapitolu zavŕšime spomenutím ďalších algoritmov, o ktorých
existencii sa oplatí vedieť a ako čerešničku na torte pridáme
počítanie diskrétnej kosínovej transformácie.
