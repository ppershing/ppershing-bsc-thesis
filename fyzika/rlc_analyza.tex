\subsection{Analýza RLC obvodov}

RLC analýza je analýza obvodov zložených z odporov, kondenzátorov a
cievok. Tieto rezonančné obvody hrajú dôležitú úlohu vo fyzike.
Základným krokom, ktorý nám umožní ich analýzu pomocou Fourierovej
transformácie je princíp superpozície
\begin{veta}[Princíp superpozície]
 Nech $U_1(t), I_1(t)$ a $U_2(t), I_2(t)$ sú dve riešenia RLC obvodu
 \footnote{Pod pojmom riešenie myslíme "riešenie sústavy
 diferenciálnych rovníc popísaných daným obvodom"}.
 Potom $U(t)=a_1 U_1(t) + a_2 U_2(t), I(t)=a_1 I_1(t) + a_2 I_2(t)$ je
 tiež riešením daného obvodu
\end{veta}
\begin{dokaz}
    Načrtneme si základné črty dôkazu, nepôjdeme však do podrobností.
    Každý RLC obvod vieme popísať ako sústavu diferenciálnych rovníc - 
    jedna rovnica pre vzťah napätia a prúdu na každej súčiastke +
    rovnice pre všetky uzly, ktoré popisujú nemožnosť hromadenia sa
    náboja na jednom mieste.
    Rovnice pre jednotlivé súčiastky
    \begin{itemize}
        \item Rezistor: $U(t) = R I(t)$
        \item Kondenzátor: $I(t) = C \pd{U}{t}$
        \item Cievka: $U(t) = L \pd{I}{t}$
        \item Uzol: $\sum_{k} I_k(t) = 0$
    \end{itemize}
    Všetky tieto 4 typy diferenciálnych rovníc majú spoločnú vlastnosť
    - linearitu medzi napätím a prúdom. Nie je tažké overiť, že pre ne
      platí princíp superpozície. Potom ale platí princíp superpozície
      pre celú sústavu týchto rovníc a teda pre RLC obvod.
\end{dokaz}
\begin{poznamka}
    Čitateľ mohol nadubodnúť dojem, že superpozícia u obvodov je
    evidentná. Veľmi rýchlo ho ale vyvedieme z omylu. Už len taká
    jednoduchá súčiastka ako dióda je prudko nelineárna. Ideálna dióda
    prepúšťa prúd len jedným smerom. Aplikáciou princípu superpozície
    dvoch prúdov rovnakej veľkosti ale opačného smeru by sme ľakho
    mohli dospieť k záveru, že za nulového prúdu tečúceho obvodom
    tečie nenulový prúd diódou. Zjavný spor. Superpozícia preto nie je 
    \todo{allmighty} nástroj na riešenie všetkých elektrických obvodov
\end{poznamka}

Podobne ako v prípade optiky, zavedieme si pojem komplexnej amplitúdy.

\todo{lit:}
%http://www.google.com/url?sa=U&start=5&q=http://www.eas.asu.edu/~holbert/eee202/EEE202_Lect12_DiffEqSolutionTransientCircuits.ppt&ei=XhjPSbXxC8HF_Qbnj_3yCQ&sig2=6yYiVDxQt1Nq3jTpLBUm5Q&usg=AFQjCNGVag3OJgkQIbwCUy_fRgBD02Y_Zw
