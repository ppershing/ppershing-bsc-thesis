\subsection{Spektrálna analýza}

Snáď najdôležitejšou aplikáciou fourierovej transformácie vo fyzike a
chémii je spektrálna interferometria. V tejto kapitole v skratke
rozvedieme najdôležitejšie pojmy a mechanizmy o tejto pre chemikov
životne dôležitej metóde skúmania látok.

Ako už názov naznačuje, spektrálna analýza má za úlohu analyzovať
látky. Deje sa tak prostredníctvom infračervených vĺn. Čitateľ by sa
mohol opýtať, prečo sa používajú tieto vlny a jeho zvedavosti bude za
chvíľu zadosťučinené.

\subsubsection{Chemické väzby}
Základom každej chemickej zlúčeniny sú atómy, medzi ktorými vznikajú
väzby. Väzba je akási neviditeľná pružina, ktorá vzniká ako dôsledok
pôsobenia elektromagenetickej a slabej jadrovej sily. Existuje istá
vzialenosť molekúl, keď sú tieto sily v rovnováhe a celková energia
väzby je minimálna. Vychýlením atómu zo svojej polohy prevládne jedna
z týchto dvoch síl a atóm má snahu dostať sa do svojej pôvodnej
polohy. Samozrejme, atómy za predpokladu bežnej teploty nestoja na
mieste ale kmitajú okolo rovnovážnej polohy a väzby ich držia pokope.

To čo je ale pri spektrálnej analýze dôležité je, že rôzne väzby sa
správajú rôzne. Kratšie väzby kmitajú rýchlejšie, väzba od ťažkých
atómov kmitá pomalšie, a takisto vlastnosti väzby závisia aj od
elektrických vlastností atómov. Preto sa každá väzba dá identifikovať
podľa svojej frekvencie kmitania. Prehľad niektorých vybraných typov
väzieb sa dá nájsť v tabuľke \ref{tab:vazby} (čiastočne prevzaná z \cite{wiki:spectro}).

\begin{table}[htb]
\begin{tabular}{| l | r | r |}
\hline
väzba & špecifický typ väzby & frekvencia ($\cm^{-1}$). \\ \hline
C-O & alkoholy & 1040-1060, 1100 \\ \hline
C-H & C=CH & 3020 \\ \hline
C-H & metyl & 1260 \\ \hline
C=O & aldehyd & 1725 \\ \hline
C-C & aromatické C=C & 1450, 1500, 1580, 1600 \\ \hline
O-H & alkoholy & 3200-3400 \\ \hline
N-H & amíny & 3400-3500, 1560-1640 \\ \hline
\end{tabular}
\caption{Ukážka absorbčných frekvencií väzieb}\label{tab:vazby}
\end{table}

Otázkou ale ostáva, ako odhaliť frekvenciu kmitania väzieb. Odpoveď
nám núka fyzika sama. Atómy sa bežne nachádzajú v základnom stave.
Avšak pôsobením žiarenia s frekvenciou rovnou frekvencii kmitania
väzby môžeme atóm rozkmitať ešte viac. Je to jav podobný rezonancii - 
vonkajším pôsobením vynútime kmity rezonátora - väzby.
Takto rozkmitaný atóm sa môže dostať do excitovaného stavu. Pri tomto
prechode medzi stavmi atóm pohltí energiu. Zo zákona zachovania
energie musí tým pádom dopadajúce žiarenie stratiť nejakú časť
energie. No a táto strata energie (na príslušnej frekvencii) je
základom pre spektrometriu.


\begin{poznamka}
Viac o chemických väzbách sa dá nájsť v \cite{wiki:bonds} a
\cite{Chem}. Taktiež na stránke \cite{webspectra} sa dá nájsť
interaktívny program na porovnávanie spektier.
\end{poznamka}

\subsubsection{Konštrukcia interferometra}



\subsubsection{Výpočet absorpcie}

