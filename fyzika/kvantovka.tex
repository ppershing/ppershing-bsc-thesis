\subsubsection{Heisenbergova nerovnosť}

Táto nerovnosť hovorí, že pre ľubovoľný signál je nemožné aby bol
naraz spektrálne a časovo ohraničený.
Aby sme zaviedli presnejšie kvantitatívne vyjadrenie, zavedieme si pojem
disperzie funkcie okolo bodu.

%%% {{{ definicia disperzie
\def\disperzia#1{\mathcal{D}_{#1}}
\begin{definicia}
    Dispezriou funkcie $f \in \LLinf, f \not = 0$ okolo bodu $a$ nazveme
    \begin{equation}
        \disperzia{a} f = \frac{\int (x-a)^2 |f(x)|^2 \dd x}
                               {\int |f(x)|^2 \dd x}
    \end{equation}
\end{definicia}
\begin{poznamka}
    Pozornejší čitatelia si mohli všimnúť, že takto definovaná
    disperzia je prirodzeným rozšírením definície disperzie funckie
    $|f(x)|^2$ v štatistike
\end{poznamka}

Interpretácia disperzie je akási miera neschopnosti funkcie byť
koncentrovanou okolo istého bodu. Pokiaľ $f$ dosahuje význačné hodnoty
len v tesnom okolí $a$, tak $\disperzia{a}f$ je malá. Naopak, ak
$f$ má význačnú časť ďaleko od $a$, disperzia bude veľká.
%%% }}}

%%% {{{ Heisenbergova nerovnost
\begin{lema}
    \begin{equation}
        \disperzia{a}f > 0
    \end{equation}
\end{lema}
\begin{dokaz}
    Vieme, že $f\not=0$ a preto $0 \not=\int |f(x)|^2 \dd x <\infty$.
    Nerovnosť je teda ekvivalentná s
    \begin{equation}
        \int (x-a)^2 |f(x)|^2 \dd x > 0
    \end{equation}
    Predpokladajme naopak, že ľavá strana je rovná nule (je zjavné, že
    nemôže byť záporná). Potom
    \begin{equation}
    \begin{split}
     0 = & \int (x-a)^2 |f(x)|^2 \dd x \\
         =& \int x^2 |f(x+a)|^2 \dd x \\
         =& \int (x |f(x+a)|)^2 \dd x
    \end{split}
    \end{equation}
    Dostali sme teda, že $x|f(x+a)|$ je rovná nule skoro všade.
    Tým pádom ale musí platiť aj $|f(x+a)|=0$ skoro všade, čo je 
    v spore s tým, že $f$ je nenulová funkcia.
\end{dokaz}

Nasledujúca veta tvrdí, že $f$ a $\dual{f}$ nemôžu byť obe
koncentrované okolo bodov.

\begin{veta}
    Heisenbergova nerovnosť:
    Pre $\forall f\in \LLinf \land f\in \PSinf$ platí
    \begin{equation}
        (\disperzia{a}f)(\disperzia{\alpha}\dual{f}) \ge \frac{1}{4}
        \quad \text{pre všetky } a,\alpha \in \R
        \label{eq:hei_ineq}
    \end{equation}
\end{veta}
\begin{dokaz}
    Začnime prípadom $a=\alpha=0$.
    Predpokladajme $x f(x) \in \LLinf$ a $f'(x)\in \LLinf$.
    V prípade že $x f(x) \not\in\LLinf$, máme
    $\int x^2 |f(x)|^2 \dd x = \infty$ a teda $\disperzia{0}f=\infty$.
    Podobne, ak $f'(x)\not\in\LLinf$, 

    Majme funkciu $x\conjug{f(x)}f'(x)$, ktorá patrí $\PCinf$. 
    Integráciou per-partes na intervale $(a,b)$ dostaneme
    \begin{equation}
        \int_a^b x\conjug{f(x)}f'(x) \dd x =
          \left[ x|f(x)|^2\right]_a^b -
          \int_a^b \left(|f(x)|^2 + x f(x) \conjug{f(x)} \right)
            \dd x
    \end{equation}
    resp.
    \begin{equation}
    \begin{split}
        \int_a^b |f(x)|^2 \dd x &=
          \left[ x|f(x)|^2\right]_a^b 
        - \int_a^b x\conjug{f(x)}f'(x) \dd x -
          \int_a^b x f(x) \conjug{f'(x)} \dd x \\
        &= \left[ x|f(x)|^2\right]_a^b 
        - \int_a^b x\conjug{f(x)}f'(x) + x f(x) \conjug{f'(x)} \dd x
          \\
        &= \left[ x|f(x)|^2\right]_a^b 
        - \int_a^b \conjug{x} \conjug{f(x)}f'(x) + x f(x)
          \conjug{f'(x)} \dd x\\
        &= \left[ x|f(x)|^2\right]_a^b 
        - 2 \Re \int_a^b \conjug{xf(x)}f'(x) \dd x
    \end{split}
    \end{equation}
    Pretože $f(x),x f(x), f'(x)$ sú v $\LLinf$, limity integrálov
    v rovnici ak $a\imply-\infty$ a $b\imply\infty$ existujú.
    Preto musia existovať aj limity $a |f(a)|^2, b |f(b)|^2$ a musia
    byť nulové. Inak by muselo platiť
    $f(x)\sim\frac{c}{\sqrt{|x|}}$ a $f$ by nemohla byť v $\LLinf$.
    Limitovaním do nekonečna potom dostaneme
    \begin{equation}
        \int |f(x)|^2 \dd x= - 2 \Re \int \conjug{x f(x)}f'(x) \dd x
        \label{eq:hei_before_cauchy}
    \end{equation}
    Podľa \todo{ref; cauchy} Cauchy-Schwarzovej nerovnosti
    \begin{equation}
        |\int \conjug{x f(x)}f'(x) \dd x| \le
            \left( \int x^2 |f(x)|^2 \dd x \right)^2
            \left( \int |f'(x)|^2 \dd x \right)^2
        \label{eq:hei_cauchy}
    \end{equation}
    A skombinovaním \ref{eq:hei_before_cauchy} s \ref{eq:hei_cauchy}
    dostaneme
    \begin{equation}
        \left( \int |f(x)|^2 \dd x)\right)^2 \le 4
            \left( \int x^2 |f(x)|^2 \dd x \right)
            \left( \int |f'(x)|^2 \dd x \right)            
        \label{eq:hei_after_cauchy}
    \end{equation}
    Podľa \todo{perseval's theorem} platí
    $\int |f(a)|^2 \dd a = \frac{1}{2\pi} \int |\dual{f}(\alpha)|^2 \dd
    \alpha$ a tiež \todo{derivative theorem}
    \begin{equation}
        \int |f'(x)|^2 \dd x = \frac{1}{2 \pi}
            \int |\fourier[f'](\chi)|^2 \dd \chi = 
            \frac{1}{2 \pi} \int \chi^2 |\dual{f}(\chi)|^2 \dd \chi
    \end{equation}
    Rovnicu \ref{eq:hei_after_cauchy} tak možno prepísať na
    \begin{equation}
        \left( \int |f(x)|^2 \dd x \right)
        \left( \int |\dual{f}(\chi)|^2 \dd \chi \right)
        \le 4
        \left(\int x^2 |f(x)|^2 \dd x \right)
        \left(\int \chi^2 |\dual{f}(\chi)|^2 \dd \chi\right)
    \end{equation}
    Čo je presne rovnica \ref{eq:hei_ineq} pre $a=\alpha=0$ ak uvážime že
    $f\not=0$ a tiež $\dual{f}\not=0$ \footnote{Samozrejme v $\LLinf$}.
    Všeobecný prípad ľahko prevedieme na záver pre nulové $a,\alpha$
    nasledujúcim spôsobom.
    Položme
    \begin{equation}
        g(x) = e^{-\imag \alpha x} f(x + a)
    \end{equation}
    \todo{this} Ľahko nahliadneme, že $\disperzia{a} f = \disperzia{0}g$ a 
    $\disperzia{\alpha}\dual{f} = \disperzia{0}\dual{g}$.
\end{dokaz}
%%% }}}

