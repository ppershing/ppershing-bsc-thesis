% vim:spell spelllang=sk
\chapter{Úvod}

Fourierova transformácia je v~súčasnej dobe významný nástroj vo
viacerých vedeckých oblastiach. Jej výhody, problémy a~zákutia boli už
mnohonásobne preskúmané desiatkami matematikov, fyzikov a~inžinierov.
Táto práca sa snaží zhrnúť významné poznatky o~tejto transformácii 
a~tiež akumulovať čo najviac rôznorodých použití v~súčasnej vede
s~dôrazom na ukázanie súvislostí medzi nimi.

\section{Motivácia}
Hlavnou motiváciou k~vzniknutiu tohoto diela bola absencia uceleného
prehľadu o~aplikáciach Fourierovej transformácie. Hoci je samotná
transformácia diskutovaná v~stovkách kníh, rozsah aplikácií je natoľko
rôznorodý a~obsiahly, že väčšina publikácii uvádza iba ich malú časť.
Ďalšie publikácie sú priamo zamerané na na určité použitie, pre ktorý bola daná
publikácia písaná. Toto je veľká škoda, lebo čitateľ tak nedostáva
kompletný prehľad súvislostí medzi týmito aplikáciami a~teda
aj súvislosti medzi rôznymi vedeckými disciplínami. Autorovým
cieľom bolo snaha čo najviac skompletizovať poznatky a~ukázať kde 
a~na čo sa daná transformácia používa s~dôrazom na súvislosti medzi
jednotlivými aplikáciami.

\section{Členenie práce}
Na úvod, v~2. kapitole si postupne ukážeme Fourierove rady, ktoré sú
historickým predchodcom Fourierovej transformácie (ďalej len FT) ako takej.
Následne
venujeme podstatnú časť kapitoly formalizácii problému, rôznym
vlastnostiam Fourierových radov a~ich konvergencii. Po vybudovaní
dostatočného aparátu spravíme rýchly prechod na spojitú Fourierovu
transformáciu, ktorú rýchlo prebehneme bez väčšieho dôrazu na
matematické dôkazy. Kapitolu ukončíme diskrétnou FT a~transformáciami
založenými na FT.
%%Tretia kapitola je venovaná samotnému výpočtu diskrétnej Fourierovej
%transformácie po algoritmickej stránke, pretože toto je jadro použitia
%v súčasnom svete počítačového spracovania údajov.
%Štvrtá 
Tretia kapitola sa venuje samotným aplikáciam FT alebo príbuzným
transformáciam.  Ukážeme použitie % v matematike na riešenie
%diferenciálnych rovníc, rôznorodé využitie vo fyzike, ktoré je
%mnohokrát priam prekvapivé. Danú kapitolu zakončíme použitím
transformácie v~súčasnom informatickom svete, kde sa ukazuje ako
sľubný nástroj na manipuláciu a~kompresiu digitálneho signálu.
Rozvedieme postupne rôzne techniky a~modifikácie FT. Najskôr ako
živnú pôdu pre manipuláciu a~filtrovanie signálu, neskôr aplikácie 
v~oblasti kompresie obrazu, videa a~zvuku. Kapitolu zavŕšime použitím
FT v~teórii algoritmov, ako nástroj na rýchle násobenie veľkých
čísel a~polynómov.
