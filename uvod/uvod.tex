\chapter{Úvod}

Fourierova transformácia je v súčasnej dobe významný nástroj vo
viacerých vedeckých oblastiach. Jej výhody, problémy a zákutia boli už
mnohonásobne preskúmané desiatkami matematikov, fyzikov a inžinierov.
Táto práca sa snaží zhrnúť významné poznatky o tejto transformácii a
tiež akumulovať čo najviac rôznorodejší použití v súčasnej vede.

\section{Motivácia}
Hlavnou motiváciou k vzniknutiu tohoto diela bola absencia uceleného
prehľadu o aplikáciach Fourierovej transformácie. Hoci je samotná
transformácia diskutovaná v stovkách kníh, rozsah aplikácií je natoľko
rôznorodý a obsiahly, že väčšina publikácii uvádza iba ich malú časť,
alebo aplikácie zamerané priamo na vedecký odbor, pre ktorý bola daná
publikácia písaná. Toto je veľká škoda, lebo čitateľ tak nedostáva
kompletný prehľad a súvislosti medzi týmito aplikáciami a tým
pádom aj súvislosti medzi rôznymi vedeckými disciplínami. Autorovým
cieľom bolo snaha čo najviac skompletizovať poznatky a ukázať kde a
na čo sa daná transformácia používa s dôrazom na súvislosti medzi
vedeckými obormi.

\section{Členenie práce}
Na úvod, v 2. kapitole si postupne ukážeme Fourierove rady, ktoré sú
historickým predchodcom Fourierovej transformácie ako takej. Následne
venujeme podstatnú časť kapitoly formalizácii problému, rôznym
vlastnostiam Fourierových radov a ich konvergencii. Po vybudovaní
dostatočného aparátu spravíme rýchly prechod na Fourierovu
transformáciu (FT) a kapitolu zakončíme diskrétnou FT a 
transformáciami založenými na FT.
Tretia kapitola je venovaná samotnému výpočtu diskrétnej Fourierovej
transformácie po algoritmickej stránke, pretože toto je jadro použitia
v súčasnom svete počítačového spracovania údajov.
Štvrtá kapitola sa venuje samotným aplikáciam FT alebo príbuzným
transformáciam. Postupne si tam ukážeme použitie v matematike na riešenie
diferenciálnych rovníc, rôznorodé využitie vo fyzike, ktoré je
mnohokrát priam prekvapivé. Danú kapitolu zakončíme použitiu
transformácie v súčasnom informatickom svete, kde sa ukazuje ako
sľubný nástroj na manipuláciu a kompresiu digitálneho signálu.
