% vim:spell spelllang=sk
\subsection{Separácia premenných}

Táto užitočná metóda na riešenie diferenciálnych rovníc uzrela prvý
krát svetlo sveta práve keď sa Fourier začal zaoberať riešením rovnice
tepla. Ukážeme si ju najskôr na jednorozmernom prípade vedenia tepla,
a následne na dvojrozmernom riešení Laplacovej rovnice obdĺžnikovej
oblasti.

\subsubsection{Rovnica vedenia tepla}

Uvažujme kovovú tyč. Aby sme si zjednodušili život, budeme používať
normalizované a teda bezrozmerné fyzikálne jednotky. Tyč bude mať
dĺžku 1 a bude tepelne izolovaná od okolia. Na začiatku bude mať
teplotu 1 a jeden jej koniec ochladíme priložením na konštantnú
teplotu 0. Formálne zapísané
$T(x,0) = 1, 0<x\le1$ a $T(0,t)=0, 0\le t$. 
Druhý koniec je izolovaný a preto
cezeň nejde žiadny tepelný tok - podmienka ekvivalentná
$\pd{T}{x}(1,t)=0$. A finálne, posledná podmienka, podmienka
vedenia tepla, ako môže každý fyzik potvrdiť je
\begin{equation}
    \pd{T}{t} = \pdd{T}{x}, \quad0<x<1
    \label{eq:heat}
\end{equation}

Pojem separácie premenných je intuitívny. Začneme predpokladom
nezávislosti premenných. Presnejšie povedané, nech
$T(x,t) = X(x) F(t)$. Substituovaním do \ref{eq:heat}
dostávame
\begin{equation}
    \frac{X''}{X} = \frac{F'}{F} = - k^2
    \label{eq:separation}
\end{equation}
kde $-k^2$ je separačná konštanta. Znamienko mínus sme uhádli
fyzikálnou intuíciou - tyč sa bude postupne ochladzovať.
Pomôžeme si trochou analýzy:
\begin{lema}
    Rovnica $F'(t) = -k^2 F(t)$ má riešenie v tvare 
    $F(t) = C_0 e^{-k^2 t}$ kde $C_0 \in R$.
    Ďalej, rovnica $X'' + k^2 X = 0$ má riešenie v tvare
    $X(x) = C_0 \sin(k x) + C_1 \cos(k x)$. Navyše, tieto riešenia sú
    jediné riešenia danej rovnice.
\end{lema}
\begin{dokaz}
    Dôkaz je štandardnou súčasťou kurzu diferenciálnych rovníc a
    nebudeme ho uvádzať.
\end{dokaz}
Vyžitím tejto lemy a rovnice \ref{eq:separation} dostávame
\begin{equation}
    T(x,t) = (C_0 \sin (k x) + C_1 \cos (k x))(C_2 e^{-k^2 t})
\end{equation}
Okrajová podmienka $T(0,t) = 0$ a nenulovosť $e^{-k^2 t}$ 
implikuje $X(0)=0$ a teda $C_1 = 0$.
Podobne, podmienka nulového tepelného toku cez izolovaný koniec tyče
na druhom konci vedie k
$X'(1) = C_0 k \cos k = 0$. Každé nenulové riešenie potom musí spĺňať
\begin{equation}
    k = \frac{\pi}{2} + n \pi, \quad n \in \Z
\end{equation}
Výsledkom je teda spočítateľne veľa riešení $T$ v tvare $T=XF$,
menovite
$T_n = A_n \exp\left(-\pi^2 (n+\frac{1}{2})^2 t \right)
            \sin \left( \pi (n + \frac{1}{2}) x \right), n\in \Z$
Pretože daný problém je lineárny, každá lineárna kombinácia týchto
riešení musí byť riešením. Všeobecné riešenie je teda
\begin{equation}
T = \sum_{n=0}^{\infty} A_n \exp\left(-\pi^2 (n+\frac{1}{2})^2 t \right)
            \sin \left( \pi (n + \frac{1}{2}) x \right), n\in \Nz
    \label{eq:heat_expansion}
\end{equation}
kde sme využili fakt $\sin(-x) = -\sin(x)$ a teda sme s čistým
svedomím vynechali záporné indexy $n$.
Ostáva teraz jediná otázka, pravdupovediac tá najťažšia. Dá sa každé
riešenie \ref{eq:heat} zadané počiatočnou podmienkou $T(x,0) = f(x)$)
zapísať ako \ref{eq:heat_expansion}?. Ak áno, ako? 
Ak ukážeme, že pre ľubovoľnú počiatočnú podmienku 
$T(x,0) = C_2 X(x) = f(x), 0\le x \le 1$ vieme nájsť koeficienty aby
daný súčet konvergoval k $T(x,0)$ v čase $t=0$, máme riešenie pre
ľubovoľnú počiatočnú podmienku.

V tomto momente prichádzajú na rad Fourierove rady. Výsledok môžeme dosiahnuť
2 spôsobmi. Prvým je ručný výpočet
\begin{equation}
    \int_0^1 \sin\left(\pi(n+\frac{1}{2})x\right)
             \sin\left(\pi(m+\frac{1}{2})x\right) \dd x =
    \left\{
    \begin{array}{l l}
        0& m\not=n \\
        \frac{1}{2} & m=n
    \end{array}
    \right.
\end{equation}
ktorý ukazuje ortogonalitu a následná expanzia na otrogonálny systém
\footnote{navyše by sa patrilo ukázať kompletnosť tohoto ortogonálneho
systému}.
Poučnejšie pre čitateľa ale bude nasledujúce odvodenie, ktoré využíva
symetriu.

To, čo je naviac oproti štandardnému Fourierovmu radu funkcie $f$ 
sú podmienky na okrajové body.
Týchto podmienok sa ale dá efektívne zbaviť.
Nech $X_2$ je funkcia $X$ rozšírená na interval $[0,2]$ nasledovne:
\begin{align}
    X_2(x) &= X(x), \quad &0\le x \le 1 \\
    X_2(x) &= X(2-x), \quad &1\le x \le 2
\end{align}
Podmienka $X'(1) = 0$ je ekvivalentná tvrdeniu
$X_2$ je spojitá a diferencovateľná v bode 1.
Pokračujme ďalej a rozšírme $X_2$ na interval $[-2,2]$ nasledovne:
\begin{align}
    X_3(x) &= X_2(x), \quad &0\le x \le 2 \\
    X_3(x) &= - X_2(-x), \quad &-2 \le x \le 0
\end{align}
Tento zápis nie je nič iné ako nepárne rozšírenie.

Riešme úlohu pre počiatočnú podmienku $T(x,0)=1, 0<x\le1$ (celá tyč
bola na začiatku v pokojovej teplote 1 a schladili sme jeden jej
koniec na teplotu 0).
$X_3 = sign(x), x\in[-2,2]$. Využitím príkladu \todo{ref}, ktorý
posunieme o $\frac{1}{2}$ dolu roztiahneme $2$krát na $y$-ovej osi 
a zmenou intervalu z $[-\pi,\pi]$ na $[-2,2]$ okamžite dostávame
$X_3(x) = 2 \frac{2}{\pi} \sum_{m=1}^\infty
  \frac{\sin((2m-1) y)}{2m-1}$, kde $y=x \frac{\pi}{2}$,
  čiže
$X_3(x) = \frac{4}{\pi} \sum_{m=1}^\infty
  \frac{\sin((m-\frac{1}{2})\pi )}{2m-1}$, kde $y=x \frac{\pi}{2}$,
Posunutím indexovania z 1 na 0 a porovnaním s  \ref{eq:heat_expansion}
dostávame
\begin{equation}
T(x,t) = \sum_{m=0}^{\infty} \frac{4}{(2m+1)\pi} 
    \exp\left[-\pi^2 (m+\frac{1}{2})^2 t \right] 
    \sin\left( \pi(m+\frac{1}{2}) x \right)
\end{equation}
