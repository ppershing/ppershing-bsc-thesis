% vim:spell spelllang=sk
\section{Raderov algoritmus}
V poradí už druhý algoritmus, ktorý vie počítať Fourierovu
transformáciu prvočíselnej dĺžky. Narozdiel od Bluesteinovho algoritmu
ale vie počítať iba prvočíselné dĺžky. Základom bude napodiv algebra.
Začneme sadou tvrdení

\todo{Tieto definicie sharnut z aplikaciou - nasobenie polynomov}
\begin{definicia}[grupa]
\end{definicia}
\begin{lema}
    $\Z_p(.)$ je cyklická grupa.
\end{lema}
\begin{definicia}[generator]
\end{definicia}
\begin{lema}
    Každá cyklická grupa má generátor. Navyše, nech $g$ je generátor a
    $n$ je veľkosť grupy. Potom
    $g^0,g^1,g^2,\dots,g^{n-1}$ sú navzájom rôzne prvky grupy a teda
    existuje bijekcia medzi číslami $0,\dots,n-1$ a prvkami grupy.
\end{lema}

Nech $p$ je prvočíslo. Potom podľa lemy \todo{ref} $Z_p(.)$ je
cyklická grupa a využítím lemy \todo{ref} vidíme, že existuje bijekcia
medzi prvkami grupy (číslami $1,2,\dots,p-1$) a číslami $0,1,\dots,p-2$.
Ak teda v rovnici \ref{eq:dft_omega} vyčleníme špeciálne 0 zo sumy,
dostávame
\begin{align}
    X_0 &= \sum_{l=0}^{p-1} x_l \omega_{p,0l} = \sum_{l=0}^{p-1} x_l \\
    X_{g^{-k}} &= \sum_{l=0}^{p-1} x_l \omega_{p,(g^{-k}) l} \\
            &= 
        x_0 + \sum_{l=1}^{p-1} x_l \omega_{p,(g^{-k}) l} \\
        &=
        x_0 + \sum_{g^j, g^j \in Z_p} x_{(g^j)} \omega_{p, (g^{-k})
        (g^j)} \\
        &= x_0 + \sum_{j \in 0,1,\dots,p-2} x_{g^j} \omega_{p,
        g^{j-k}}, \quad k \in 0,1,\dots,p-2
\end{align}
Označme si postupnosti
\begin{align}
    A_k &= x_{g^k} \\
    B_k &= \omega_{p, g^{-k}}
\end{align}
Pretože $g^{-(p-x)} = g^{x - p} = g^{x}$, môžeme písať
\begin{equation}
    X_{g^{-k}} -x_0=  \sum_{j=0}^{p-2} A_j B_{k-j}
\end{equation}
Teraz už môžeme vidieť jasnú cyklickú konvolúciu o veľkosti $p-2$.
Túto konvolúciu môžeme vypočítať buď priamo aplikovaním konvolučnej
vety \todo{ref} a spočítaním fourierových transformácii postupností
$A,B$. Pohodlnejšie je ale, podobne ako u Bluesteinovho algoritmu,
doplniť postupnosti $A,B$ na dĺžku veľkosti mocniny dva a použiť
niektorý z rýchlych algoritmov ktoré sme spomínali.
Poslednou nepríjemnosťou je index $g^{-k}$, kde by sme radšej videli
$k$. Ako sme ale spomínali, existuje bijekcia medzi týmito dvoma
množinami.

\todo{pokec o generatoroch / primitive roots, ako ich hladat, ako
rychlo umocnovat - vyuzije sa dalej v rychlom nasobeni}

\begin{python}
import cmath
import math
import dct
import BitUtils

class GroupHelper:
    def IsPrime(n):
        return len(GroupHelper.Factor(n))==1
    IsPrime = staticmethod(IsPrime)

    def Factor(n):
        assert(n>1)
        result = []
        i = 2
        while (n>= i*i):
            while (n%i ==0):
                result.append(i)
                n /= i
            i += 1
        if (n>1):
            result.append(n)
        return result
    Factor = staticmethod(Factor)

    def PowerMod(x, exp, n):
        mask = 1
        pwr = x
        result = 1
        while (mask<= exp):
            if ( exp&mask == mask):
                result *= pwr
                result %= n
            mask *= 2
            pwr *= pwr
            pwr %= n
        return result
    PowerMod = staticmethod(PowerMod)

    def Generator(n):
        assert(GroupHelper.IsPrime(n))
        i = 2
        factors = GroupHelper.Factor(n-1)
        while True:
            is_generator = True
            for j in factors:
                if (GroupHelper.PowerMod(i, (n-1)/j, n) == 1):
                    is_generator = False
            if is_generator:
                return i
            i += 1
    Generator = staticmethod(Generator)


class Rader:
    def transform(data):
        p = len(data)
        assert(GroupHelper.IsPrime(p))
        g = GroupHelper.Generator(p)
        ginv = GroupHelper.PowerMod(g, p-2, p)


        N = BitUtils.nextPowerOfTwo(2*(p-1))

        a = []
        t = 1
        for i in range(p-1):
            a.append(data[t]);
            t = (t*g) % p
        
        for i in range(N - (p-1)):
            a.append(0)

        b = []
        t = 1
        for i in range(p-1):
            b.append(cmath.exp(-2*1j*math.pi /(p) * t))
            t = (t*ginv) % p
        for i in range(N - 2*(p-1)):
            b.append(0)
        for i in range(p-1):
            b.append(b[i])

        # do convolution
        tmp = dct.DFT();
        a = tmp.transform(a)
        b = tmp.transform(b)
        for i in range(N):
            a[i] *= b[i]

        a = tmp.inverseTransform(a)

        result = [sum(data)]
            
        for i in range(p-1):        
            result.append(data[0]);

        t = 1
        for i in range(p-1):
            result[t] += a[i]
            t = (t*ginv)%p

        return result
    transform = staticmethod(transform)
\end{python}

\todo{lit:wiki:rader}
