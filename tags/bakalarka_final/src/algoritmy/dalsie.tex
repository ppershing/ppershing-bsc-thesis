% vim:spell spelllang=sk
\section{Ďalšie algoritmy}

Algoritmov na výpočet rýchlej Fourierovej transformácie je nespočetne
veľa. Mnohé z týchto algoritmov sú len úpravami už spomínaných
algoritmov, snažiace sa poväčšinou o efektívnejšiu implementáciu -
využitie symetrie komplexných čísel, používanie reálnych čísel tam kde
treba a mnohé ďalšie optimalizácie až po efektívne ukladanie v pamäti
kvôli rýchlosti procesorovej cache. Ďalšie sú síce efektívne, ale v
dobe čoraz rýchlejších počítačov schopných paralelných výpočtov ich
algoritmická zložitosť prevažuje výhody, ako je to napríklad u
Winogradovho algoritmu. V tejto kapitole si spomenieme
niektoré ďalšie zaujímavé myšlienky, ale nebudeme ich rozoberať
podrobne, iba načrtneme výsledky.

\subsection{Split radix}
Tento algoritmus neprináša žiadnu prevratnú myšlienku, ktorú sme
doteraz nemali. Skôr je to zaužívaný prístup, ktorý bol preferovaný
pred radix2 decimáciou v čase či frekvencii z jediného dôvodu -
vyžmýkal z tohoto prístupu ešte čosi naviac. Je taktiež pozoruhodný
svojím rozdelením na nie rovnako veľké podčasti, odkiaľ je aj jeho
názov. Pozrime sa teda čo má pod kapotou.
Rovnicu \ref{eq:dft_omega} rozpíšeme nasledovne
\begin{equation}
\begin{split}
    X_k &= \sum_{l=0}^{n/2-1} x_{2l} \omega_{n, 2lk} +
          \sum_{l=0}^{n/4-1} x_{4l+1} \omega_{n,(4l+1)k} +
          \sum_{l=0}^{n/4-1} x_{4l+3} \omega_{n,(4l+3)k} \\          
        &= \sum_{l=0}^{n/2-1} x_{2l} \omega_{n/2, lk} +
          \omega_{n,k}
            \sum_{l=0}^{n/4-1} x_{4l+1} \omega_{n/4, lk} +
          \omega_{n,3k}
          \sum_{l=0}^{n/4-1} x_{4l+3} \omega_{n/4,lk}
\end{split}
\end{equation}
Výraz teda pozostáva z troch fourierových transformácii, postupne o
veľkostiach $n/2,n/4,n/4$. Ak ich výslsky označíme
$U,Z,Z'$ a využijeme ich periodickosť modulo $n/2,n/4,n/4$,
dostávame identity
\begin{align}
    X_k  &= U_k + \left( \omega_{n,k} Z_k + 
                \omega_{n,3k} Z'_k \right) \\
    X_{k+n/4}  &= U_{k+n/4} - \imag \left( \omega_{n,k} Z_k -
                \omega_{n,3k} Z'_k \right) \\
    X_{k+n/2}  &= U_{k} -  \left( \omega_{n,k} Z_k + 
                \omega_{n,3k} Z'_k \right) \\
    X_{k+3n/4}  &= U_{k+n/4} + \imag \left( \omega_{n,k} Z_k -
                \omega_{n,3k} Z'_k \right) \\
\end{align}
pre $k=0,1,\dots,n/4-1$. Algoritmus teda oproti 2radixu využíva
špeciálne prípady, kedy je násobenie twiddle faktorom $\omega$
prakticky zdarma, pretože ide o sčítanie resp. odčítanie zložiek
komplexných čísel.

\subsection{Good-Thomas algorithm}
Druhý algoritmus, ktorý si v tejto časti v skratke popíšeme je 
Good-Thomasov algoritmus, niekedy tiež nazývaný "prime factor
algorithm". Algoritmus je náramne podobný Cooley-Tukeymu algoritmu. 
Oba istým
spôsobom rozložia výpočet na dvojrozmernú foruierovu transformáciu.
Pravdupovediac, istú chvíľu panoval názor, že oba algoritmy sú
rovnaké. Good-Thomasov algoritmus na svoj výpočet aby sa $n$ dalo
zapísať ako $r_1 r_2$, kde $r_1,r_2$ sú nesúdeliteľné, narozdiel od
Cool-Tukeyho algoritmu. Na druhú stranu, tento predpoklad, ako si
ukážeme nás vie zbaviť násobenia twiddle faktormi, ktoré boli potrebné
u Cool-Tukeyho algoritmu.

V prípade, že $r_1,r_2$ sú nesúdeliteľné, vieme zaviesť nasledujúcu
bijekciu medzi $k \equivalence k_1,k_2$ a $l \equivalence l_1,l_2$.
\begin{align}
    l &= l_1 r_2 + l_2 r_1 \\    
    k &= k_1 ( r_2^{-1} \mod r_1) r_2 +
        k_2 ( r_1^{-1} \mod r_2) r_1 \mod n \\&=
        k_1 r'_2 r_2 + k_2 r'_1 r_1 \mod n
\end{align}
Druhá rovnica možno vyzerá podozrivo, ale v skutočnosti je to len
riešenie čínskej zvyškovej vety pre rovnice
$k = k_1 \mod r_1$ a $k = k_2 \mod r_2$. Čitaťel ši môže všimnúť
potrebu nesúdeliteľnosti, v prvej rovnici kvôli zabezpečeniu bijekcie
a v druhej rovnici navyše kvôli existencii modulárnych inverzov.
Rovnicu \ref{eq:dft_omega} potom vieme prepísať do nasledovnej podoby:
\begin{equation}
    \begin{split}
    X_{k_1,k_2} &= \sum_{l=0}^{n-1} x_l \omega_{n,kl} \\
            &= \sum_{l_1=0}^{r_1-1} \sum_{l_2=0}^{r_2-1}
              x_{l_1 r_2 + l_2 r_1} 
              \omega_{n,(k_1 r'_2 r_2 + k_2 r'_1 r_1 \mod n) 
                            (l_1 r_2 + l_2 r_1)}  \\
            &= \sum_{l_1=0}^{r_1-1} \sum_{l_2=0}^{r_2-1}
              x_{l_1, l_2} 
              \omega_{n,(k_1 r'_2 r_2 + k_2 r'_1 r_1) 
                            (l_1 r_2 + l_2 r_1)}
    \end{split}
    \label{eq:gt_odvodenie1}
\end{equation}
Ďalej vieme $\omega$ upraviť nasledovne:
\begin{equation}
    \begin{split}
    \omega_{n,(k_1 r'_2 r_2 + k_2 r'_1 r_1) 
                            (l_1 r_2 + l_2 r_1)} &=
    \omega_{n, (k_1 r'_2 r_2)(l_1 r_2) +
                (k_1 r'_2 r_2)(l_2 r_1) +
                (k_2 r'_1 r_1)(l_1 r_2) +
                (k_2 r'_1 r_1)(l_2 r_1)} \\
    &=
    \omega_{n, (k_1 r'_2 r_2)(l_1 r_2)} 
    \omega_{n, (k_2 r'_1 r_1)(l_2 r_1)} \\
    &=
    \omega_{n, (k_1 r'2 r_2)(l_1 r_2)} 
    \omega_{n, (k_2 r'_1 r_1)(l_2 r_1)} \\
    &=
    \omega_{r_1, (k_1 r'2 r_2)l_1} 
    \omega_{r_2, (k_2 r'_1 r_1)l_2} \\
    &=
    \omega_{r_1, k_1 l_1} 
    \omega_{r_2, k_2 l_2} \\    
    \end{split}
\end{equation}
kde v poslednej rovnosti sme využili fakt $r'_2 r_2 = 1 \mod r_1$
a $r'_1 r_1 = 1 \mod r_2$.
Potom ale rovnica \ref{eq:gt_odvodenie1} prejde na
\begin{equation}
    X_{k_1,k_2} 
            = \sum_{l_1=0}^{r_1-1} 
              \left(
                \sum_{l_2=0}^{r_2-1}
                x_{l_1, l_2} \omega_{r_2, k_2 l_2}  
                \right)
              \omega_{r_1, k_1 l_1} 
\end{equation}
Kde už môžeme vidieť dvojrozmernú fourierovu transformáciu. Vhodným
preindexovaním teda vieme rýchlo počítať prípady, kedy sú $r_1,r_2$
nesúdeliteľné. Praktickosť tohoto algoritmu ale nie je výrazná,
pretože samotné preindexovanie vyžaduje modulárnu aritmetiku a tak je
výpočtovo pomerne náročné. Algoritmus teda skôr slúži ako ukážka
ďalšieho šikovného prístupu k výpočtu Fourierovej transformácie.

\todo{lit:}
%http://en.wikipedia.org/wiki/Split-radix_FFT_algorithm
%http://en.wikipedia.org/wiki/Prime-factor_FFT_algorithm
%mrkni aj inu literaturu
