% vim:spell spelllang=sk
\section{Viacrozmerná Fourierova transformácia}

Niekedy je potrebné vedieť aplikovať Fourierovu transformáciu vo
viacerých rozmeroch. Typické použitia sú vo fyzike (3-rozmerný
priestor), matematike (diferenciálne rovnice o~viacerých neznámych) 
a~informatike (2D signály ako napríklad obraz). Je preto výhodné venovať
problematike viacerých rozmerov. Prvou otázkou ktorú si
treba zodpovedať sú základné funkcie použité na rozklad. Po vzore
jednorozmerného prípadu by sme mohli použiť sínusy a~kosínusy. Prvým
nápadom môže byť použiť vlny, aké vznikajú napríklad pri dopade kvapky
do vody. Ich rezom cez stred je sínusoida resp. kosínusoida. Takéto
riešenie ale nie je vhodné, pretože požaduje kruhovú
symetriu \footnote{Pre problémy s kruhovou symetriou je vhodné sa
oboznámiť s~Fourier-Besselovými radmi}. Namiesto toho sa používajú
funkcie, ktoré sú násobkom dvoch Fourierových funkcií pre dve rôzne
premenné. V~prípade (exponenciálnych) Fourierových radov na štvorcovej oblasti
$[-\pi,\pi]\cross[-\pi,\pi]$ je to
$e^{-\imag n x} e^{-\imag m y}$. Môžeme si overiť ortogonalitu
\begin{equation*}
 \int_{[-\pi,\pi]\cross[-\pi,\pi]} (e^{-\imag n_1 x} e^{-\imag m_1 y}) 
      (e^{-\imag n_2 x} e^{-\imag m_2 y}) \dd x,y =
    \left\{
        \begin{array}{l l}
            4\pi^2& n_1 = n_2\text{ a }m_1 = m_2\\
            0& \text{v opačnom prípade}
        \end{array}
    \right.
\end{equation*}
Daná forma sa tiež dá kompaktnejšie vektorovo zapísať ako 
$e^{-\imag (n,m).(x,y)}$.

Nás bude predovšetkým zaujímať viacrozmerná diskrétna Fourierova
transformácia, ktorej vzorec je
\begin{equation}
    X_{k_1,k_2,\dots,k_d} = \sum_{n_1=0}^{N_1-1}
        \sum_{n_2=0}^{N_2-1} \dots \sum_{n_d=0}^{N_d-1}
        e^{-\frac{2\pi\imag}{N_1} n_1 k_1}
        e^{-\frac{2\pi\imag}{N_2} n_2 k_2} \dots
        e^{-\frac{2\pi\imag}{N_d} n_d k_d}
        x_{n_1,n_2,\dots,n_d}
    \label{eq:multi_dft_full}
\end{equation}
Prepísané do vektorovej podoby, kde $\vect{n}=(n_1,n_2,\dots,n_d)$,
$\vect{k}=(k_1,k_2,\dots,k_d)$,
$\vect{N-1}=(N_1-1,N_2-1,\dots,N_d-1)$ a $\vect{n/N}=(n_1/N_1,
n_2/N_2, \dots, n_d/N_D)$ prejde na tvar
\begin{equation*}
    X_{\vect{k}} = \sum_{\vect{n}=\vect{0}}^{\vect{N-1}}
      e^{-2\pi\imag \vect{k}.\vect{(n/N)}} \;x_{\vect{n}}
\end{equation*}
Inverzná transformácia je definovaná analogicky ako
\begin{equation*}
    x_{\vect{k}} = \frac{1}{\prod_{l=1}^{d} N_l} 
            \sum_{\vect{n}=\vect{0}}^{\vect{N-1}}
                e^{2\pi\imag \vect{k}.\vect{(n/N)}}\; X_{\vect{n}}
\end{equation*}
Nás však bude omnoho viac zaujímať forma \eqref{eq:multi_dft_full}.
Zaoberajme sa chvíľu 2-rozmerným prípadom.
\begin{equation*}
    \begin{split}
    X_{k_1,k_2} &=
        \sum_{n_1=0}^{N_1-1}
        \sum_{n_2=0}^{N_2-1} 
        e^{-\frac{2\pi\imag}{N_1} n_1 k_1}
        e^{-\frac{2\pi\imag}{N_2} n_2 k_2} 
        x_{n_1,n_2} \\&=
        \sum_{n_1=0}^{N_1-1}
        e^{-\frac{2\pi\imag}{N_1} n_1 k_1}
        \sum_{n_2=0}^{N_2-1} 
        e^{-\frac{2\pi\imag}{N_2} n_2 k_2} 
        x_{n_1,n_2} \\&=
        \sum_{n_1=0}^{N_1-1}
        e^{-\frac{2\pi\imag}{N_1} n_1 k_1}
          Y_{n_2}[n_1]
    \end{split}
\end{equation*}
kde $Y_{n_2}[i]$ je definované ako
\begin{equation*}
    Y_{n_2}[n_1]=
        \sum_{n_2=0}^{N_2-1} e^{-\frac{2\pi\imag}{N_2} n_2 k_2} x_{n_1,n_2}
\end{equation*}
Celá transformácia sa dá teda zložiť z~dvoch častí. Každá časť je
Fourierovou transformáciou v~jednom indexe. Presnejšie povedané, ak
$x$ zapíšeme do matice, potom $Y$ získame aplikovaním Fourierovej
transformácie na každý riadok matice a $X$ získame aplikovaním
Fourierovej transformácie na každý stĺpec matice $Y$.
Vidíme teda, že viacrozmerná diskrétna Fourierova transformácia sa dá
efektívne počítať pomocou jednorozmernej verzie - postupne aplikujeme
Fourierovu transformáciu v každej dimenzii. Toto robí viacrozmernú DFT
atraktívnu pre spracovanie priestorových dát, nakoľko
vieme využívať efektívne algoritmy pre výpočet jednorozmernej DFT.
