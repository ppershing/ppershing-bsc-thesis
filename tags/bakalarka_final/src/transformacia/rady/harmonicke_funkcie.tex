% vim:spell spelllang=sk
\subsection{Harmonické funkcie}

%%% {{{ Fourierove rady, sin cos
Teória okolo Fourierových radov sa zaoberá základnou otázkou "Ako
aproximovať funkciu pomocou základných funkcií?".
Za základné funkcie si môžeme zvoliť ľubovoľnú sadu funkcii a~potom
študovať, či je daná aproximácia možná. Jedným zo zaujímavých
odrazových mostíkov, ktorým sa venoval Fourier a~teda aj celá táto
publikácia sú harmonické funkcie ako $\sin(n x), \cos(nx), n \in \N$.
Na obrázkoch \ref{fig:harmonic_illustration} je znázornený 
ich priebeh a~skladanie.

\begin{figure}[htp]
    \centering
    \includegraphics{obrazky/transformacia/rady/phase_difference}
    \includegraphics{obrazky/transformacia/rady/frequency_difference}
    \includegraphics{obrazky/transformacia/rady/addition}
    \caption{Fázový posun, rôzne frekvencie a skladanie harmonických
    funkcií}\label{fig:harmonic_illustration}
\end{figure}

Ako sa presvedčíme v~ďalšom texte, pomocou harmonických funkcií a~ich
skladania budeme vedieť vyjadriť veľkú triedu iných funkcií a~celá
táto práca sa venuje ich využitiu.

\begin{definicia} Nech $f(x)$ je ľubovoľná funkcia definovaná na
intervale $(-\pi,\pi)$. Napíšme
    \begin{equation}
        f(x) \sim \frac{1}{2} a_0 + \sum_{n=1}^{\infty} a_n
        \cos n x + b_n \sin n x
    \label{eq:trig_series}
    \end{equation}
a~daný rad nazvime Fourierovým radom.
    \label{def:fourier_series}
\end{definicia}

\begin{poznamka}
    Pozornému oku neunikne konštanta $\frac{1}{2}$ pred $a_0$, ktorá
    zatiaľ nemá žiadny zjavný význam. Môžeme však ubezpečiť čitateľa,
    že je to vhodne zvolená konštanta, aby boli zvyšné výsledky 
    a~zápisy krajšie. Taktiež je to zaužívaná konvencia a~nebudeme
    ju meniť.
\end{poznamka}

Rad v~tejto podobe, kde konštanty $a_n, b_n$ ešte treba bližšie
špecifikovať sa nazýva trigonometrický nekonečný rad. Môže a~nemusí
konvergovať, a~pre tie hodnoty $x$ kde konverguje, môže a~nemusí
konvergovať k $f(x)$. Celou úlohou tejto kapitoly bude špecifikovať
aké sú hodnoty $a_n, b_n$ a~kedy rad konverguje.
%%% }}}

%%% {{{ Otrogonalita sin, cos a vzorec pre a_n,b_n
Začneme základnými závislosťami. Integrovaním môžeme overiť, že pre
ľubovoľné $m,n \in \Z$ platí
\begin{equation}
\begin{split}
     \intLL \cos \frac{m \pi x}{L} 
     \cos\frac{n \pi x}{L}\, \dd x &= 
     \left\{
        \begin{array}{l l}
            0,& \quad m\not=n \\
            L,& \quad m=n>0 \\
            2L,& \quad m=n=0
        \end{array}
     \right. \\
     \intLL \sin \frac{m \pi x}{L} 
     \sin\frac{n \pi x}{L}\, \dd x &= 
     \left\{
        \begin{array}{l l}
            0,& \quad m\not=n \\
            L,& \quad m=n > 0
        \end{array}    
     \right. \\
     \intLL \cos \frac{m \pi x}{L}
     \sin \frac{n \pi x}{L}\, \dd x &= 0,
        \quad \text{pre všetky } m,n
        \label{eq:basic_orthogonality}
\end{split}
\end{equation}
Taktiež platí pre $n \in N, n>0$
\begin{equation}
\begin{split}
     \intLL \cos \frac{n \pi x}{L} \dd x = {}& 0\\
     \intLL \sin \frac{n \pi x}{L} \dd x = {}& 0 
    \label{eq:harmonic_integrals}
\end{split}    
\end{equation}


Pokračujme teda v~našom probléme určiť koeficienty $a_n,b_n$ ak
poznáme funkciu $f(x)$.
Predpokladajme, že funkcia $f(x)$ sa dá zapísať podľa
\eqref{eq:trig_series}. Navyše predpokladajme, že daný rad môže byť
integrovateľný člen po člene, teda že integrál zo sumy je suma
integrálov (a~teda tiež predpokladajme, že $f(x)$ je integrovateľná).
Potom integrovaním od $-\pi$ po $\pi$ dostaneme
%
\begin{equation*}
    \intpipi f(x) \dd x = \frac{a_0}{2} \intpipi \dd x +
        \sum_{k=1}^{\infty} \left( 
            a_k \intpipi \cos kx  \dd x +
            b_k \intpipi \sin kx  \dd x
        \right)
\end{equation*}
Využitím \eqref{eq:harmonic_integrals} zmiznú všetky integrály v~sume 
a~výsledok je
\begin{equation*}
   a_0 = \frac{1}{\pi} \intpipi f(x) \dd x 
\end{equation*}
%
Ako ďalší krok zoberme \eqref{eq:trig_series}, vynásobme obe strany
$\cos(nx)$ a~budeme integrovať podobne ako minule.
%
\begin{equation*}
    \begin{split}
    \intpipi f(x) \cos nx  \dd x =& \frac{a_0}{2} \intpipi \cos nx \dd x 
    \\ &+
        \sum_{k=1}^{\infty} \left(
            a_k \intpipi \cos kx \cos nx \dd x + 
            b_k \intpipi \sin kx \cos nx \dd x
        \right)
    \end{split}
\end{equation*}
%
Aplikovaním \eqref{eq:harmonic_integrals} na prvý integrál 
a \eqref{eq:basic_orthogonality} na integrály v~sume, všetky integrály
až na jeden budú nulové a~ostane len integrál za koeficientom $a_n$.
%
\begin{equation*}
    \intpipi f(x) \cos nx \dd x = a_n \intpipi \cos^2 nx \dd x = a_n \pi
\end{equation*}
%
Podobne môžeme odvodiť
%
\begin{equation*}
    \intpipi f(x) \sin nx \dd x = b_n \intpipi \sin^2 nx \dd x = b_n \pi
\end{equation*}
%
Zhrnutím dosiahnutých výsledkov dokopy, môžeme tvrdiť
%
\begin{equation}
\begin{split}
    a_n &=  \frac{1}{\pi} \intpipi f(x) \cos n x \dd x \quad n\in \Nz \\
    b_n &=  \frac{1}{\pi} \intpipi f(x) \sin n x \dd x \quad n\in \N
\end{split}    
    \label{eq:fourierove_koeficienty}
\end{equation}

Ak $f(x)$ je integrovateľná a~dá sa rozvinúť na
trigonometrický rad, a~ak trigonometrický rad získaný násobením
$\cos nx$ a $\sin nx$ pre $n\in \Nz$ sa dá integrovať člen po člene,
potom koeficienty $a_n, b_n$ sú dané podľa vzorca
\eqref{eq:fourierove_koeficienty}.

\begin{definicia}
    Koeficienty vypočítané podľa vzorca
    \eqref{eq:fourierove_koeficienty} nazveme 
    {\emph Fourierove koeficienty funkcie $f(x)$}.
\end{definicia}

%%% }}}

%%% {{{ Priklad: pulz
\begin{priklad}
Majme funkciu $f:\R\imply\R$ definovanú nasledovne
    \begin{equation*}
        f(x) = \left\{
            \begin{array}{l l}
                0 \quad x \in (-\pi,0) \\
                1 \quad x \in (0,\pi)
            \end{array}
        \right.
    \end{equation*}
% 
    \begin{poznamka}
        Všimnime si, že naša funkcia nie je definovaná v~bode 0.
        Avšak pretože Fourierove koeficienty sú počítané integrovaním,
        zmena funkcie na konečnom počte bodov nemení hodnotu výsledku
        a~teda ju možno dodefinovať na akúkoľvek hodnotu.
    \end{poznamka}
%
    Pre $n \in \Nz$ platí
    \begin{equation}
        a_n = \frac{1}{\pi} \intpipi f(x) \cos nx \dd x =
        \frac{1}{\pi} \int_0^{\pi} \cos nx \dd x
    \end{equation}
    Integrovaním dostávame
    \begin{equation*}
    \begin{split}
        a_0 &= 1 \\
        a_n &= \frac{1}{\pi} \left. \frac{1}{n} \sin nx
        \right|_0^{\pi} = 0, \quad n>0        
    \end{split}
    \end{equation*}
%
    Podobne, pre $n\in \N$ máme
    \begin{equation*}
        b_n = \frac{1}{\pi} \int_0^{\pi} \sin nx \dd x =
           \frac{1}{\pi} \left. \frac{-1}{n} \cos nx \right|_0^{\pi} =
           \frac{1 - \cos n\pi}{n\pi}
    \end{equation*}
    Čo po výpočte dáva výsledok
    \begin{equation*}
        b_n = \left\{
                \begin{array}{l l}
                    0 \quad & n \text{ je párne} \\
                    \frac{2}{n\pi}  \quad &n \text{ je nepárne}
                \end{array}
            \right.
    \end{equation*}
%
    Výsledný Fourierov rad pre $f(x)$ je 
    \begin{equation*}
    \begin{split}
        f(x) &\sim \frac{1}{2} + \frac{2}{\pi} \sin x + \frac{2}{3\pi}
            \sin 3x + \frac{2}{5\pi} \sin 5x + \cdots \\
            &\sim \frac{1}{2} + \frac{2}{\pi} \sum_{m=1}^{\infty}
                \frac{\sin\left( (2m-1) x\right)}{2m-1}
    \end{split}
    \end{equation*}
%
    Označme čiastočný súčet $S_n$ tohoto radu ako 
    \begin{equation*}
    \begin{split}
        S_0(x) &= \frac{1}{2} \\
        S_n(x) &= \frac{1}{2} + \frac{2}{\pi} \sum_{m=1}^{n}
                \frac{\sin\left( (2m-1) x\right)}{2m-1} \quad n>0
    \end{split}
    \end{equation*}
 %   
    Graf funkcie $f(x)$ a~prvých pár čiastočných súčtov $S_n$ sa dá
    nájsť na obrázku \ref{fig:example_pulse}

    \begin{figure}[htp]
        \centering
        \includegraphics{obrazky/transformacia/rady/example_pulse}
        \caption{Čiastočné súčty Fourierovho radu obdĺžnikovej funkcie}
        \label{fig:example_pulse}
    \end{figure}    
    \label{priklad:fourier_series_rect}
\end{priklad}
%%% }}}

%%% {{{ priklad: x
\begin{priklad}
   Uvažujme funkciu $f:\R\imply\R$
   \begin{equation*}
        f(x) = x, \quad x \in(-\pi,\pi)
   \end{equation*}
%
    Pretože funkcia $x \cos nx$ je nepárna, môžeme tvrdiť
   \begin{equation*}
        a_n = \frac{1}{\pi} \intpipi x \cos nx \dd x =0
   \end{equation*}
    Koeficienty $b_n$ dopočítame jednoduchou integráciou
   \begin{equation*}
   \begin{split}
        b_n &= \frac{1}{\pi} \intpipi x \sin nx \dd x \\
        &= \frac{1}{\pi} \left.
            \frac{- x \cos nx}{n}
            \right|_{-\pi}^\pi -
            \frac{1}{\pi} \intpipi \frac{-1}{n} \cos nx \dd x  \\
        &=
        \frac{-2 \cos nx}{n} + 
            \left.
                \frac{\sin nx}{n^2}
            \right|_{-\pi}^{\pi} \\
        &= -\frac{2 \cos nx}{n}
   \end{split}
   \end{equation*}
   A~teda Fourierov rad  $f(x)$ je
 %  
    \begin{equation*}
    \begin{split}
        f(x) &\sim 2 \sin x - \sin 2x + \frac{2}{3} \sin 3x - \frac{2}{4}
        \sin 4x + \cdots \\
        &\sim 2\sum_{n=1}^{\infty} \frac{(-1)^{n+1}}{n} \sin nx
    \end{split}
    \end{equation*}
  %  
    Graf funkcie $f(x)$ a prvých pár čiastočných súčtov $S_n$ sa dá
    nájsť na obrázku \ref{fig:example_lin}
 %
    \begin{figure}[htp]
        \centering
        \includegraphics{obrazky/transformacia/rady/example_lin}
        \caption{Čiastočné súčty Fourierovho radu funkcie $f(x)=x$ na
        intervale $(-\pi,\pi)$}
        \label{fig:example_lin}
    \end{figure}
    \label{priklad:fourier_series_linear}
\end{priklad}

%%% }}}

%%% {{{ Priklad: |x|
\begin{priklad}
    Ako tretí príklad si uvedieme funkciu
    \begin{equation*}
        f(x) = |x|, \quad x\in(-\pi,\pi)
    \end{equation*}
    Vieme, že $|x| \sin nx$ je nepárna funkcia a~preto
    \begin{equation*}
        b_n = \frac{1}{\pi} \intpipi |x| \sin nx \dd x = 0
    \end{equation*}
    Pre výpočet členov $a_n$ si rozdelíme integrál na dva intervaly:
    \begin{equation*}
    \begin{split}
        a_0 &= \frac{1}{\pi} \intpipi |x| \dd x = \pi \\
        a_n &= \frac{1}{\pi} \intpipi |x| \cos nx \dd x 
            = \frac{1}{\pi} \int_0^{\pi} x \cos nx \dd x +
                \frac{1}{\pi} \int_{-\pi}^0 -x \cos nx \dd x \\
            &= \frac{2}{\pi} \int_0^{\pi} x \cos nx \dd x 
             = \frac{2}{\pi} \left.
                \frac{x \sin nx}{n}
                \right|_0^{\pi} -
                \int_0^{\pi} \frac{\sin nx}{n} \dd x \\
            &= \frac{2}{\pi} \left.
                    \frac{-\cos nx}{n^2} \right|_0^\pi 
            = \frac{2}{\pi} \frac{(1-\cos n\pi)}{n^2}
    \end{split}
    \end{equation*}
    A~výsledný Fourierov rad je
    \begin{equation*}
    \begin{split}
        f(x) &\sim \frac{\pi}{2} - \frac{4}{\pi} \cos x -
            \frac{4}{9\pi} \cos 3x - \cdots \\
            &\sim \frac{\pi}{2} - \frac{4}{\pi}
                \sum_{m=1}^{\infty} \frac{\cos \left((2m-1) x\right)}{(2m-1)^2}            
    \end{split}
    \end{equation*}
  %  
    Graf funkcie $f(x)$ a~prvých pár čiastočných súčtov $S_n$ sa dá
    nájsť na obrázku \ref{fig:example_saw}

    \begin{figure}[htp]
        \centering
        \includegraphics{obrazky/transformacia/rady/example_saw}
        \caption{Čiastočné súčty Fourierovho radu funkcie $f(x)=|x|$ na
        intervale $(-\pi,\pi)$}
        \label{fig:example_saw}
    \end{figure}    
 V~tomto príklade vidíme, že čiastočné súčty sa približujú k~funkcii
 rýchlejšie ako v~predchádzajúcich prípadoch a~rovnomerne.
 Formálnejšie si tento jav popíšeme neskôr, keď budeme mať vybudovaný
 dostatočný aparát.
 \label{priklad:fourier_series_abs}
\end{priklad}
%%% }}}

Na záver si uvedieme užitočné identity pre harmonické funkcie
\begin{lema}[Rozklad sínusu]
    \begin{equation*}
        \sin(x\pm y) = \sin x \cos y \pm \sin y \cos x
    \end{equation*}
\end{lema}
\smalltodo{DOKAZ}

\begin{lema}[Rozklad kosínusu]
    \begin{equation*}
        \cos(x\pm y) = \cos x \cos y \mp \sin x \sin y
    \end{equation*}
\end{lema}
\smalltodo{DOKAZ}

\begin{lema}[O~lineárnej kombinácii]
    \begin{equation*}
        a \sin x + b \cos x = \sqrt{a^2 + b^2} \cos (x - \phi)
    \end{equation*}
    kde $\phi = \arg (b + i a)$ (Funkcia $\arg z$ reprezentuje
    uhol medzi reálnou osou a~vektorom reprezentujúcim komplexné číslo
    $z$).
    \label{lema:lk_sin_cos}
\end{lema}
\smalltodo{DOKAZ}
\begin{lema}[Lineárna kombinácia harmonických funkcií]
    Nech $f_i(x) = a_i \cos (x + \phi_i)$.
    Potom môžeme zapísať $F(x) = \sum_{k=1}^n f_k(x)$ ako
    $F(x) = a \cos (x + \phi)$.
    \label{lema:lk_harmonickych_funkcii}
\end{lema}
\smalltodo{odkomentuj}
%\begin{dokaz}
%    Dôkaz je jednoduchý - najskôr použijeme lemu \ref{lema:lk_sin_cos}
%    a každú funkciu $f_i$ prevedieme na lineárnu kombináciu $\sin x,
%    \cos x$. Následne preusporiadame členy súčtu aby boli sínusy a
%    kosínusy pokope a sčítame ich koeficienty. Celá suma je potom
%    lineárna kombinácia $\sin x, \cos x$ a opäť použitím lemy
%    \ref{lema:lk_sin_cos} ju prevedieme na požadovaný tvar
%\end{dokaz}
