% vim:spell spelllang=sk
%\subsection{Signal processing}
\section{Signal processing}
%%% {{{ uvodny pokec
Či chcete alebo nechcete, spracovanie digitálneho signálu je veľmi dôležitou
súčasťou každodenného života. Okolo nás sa každú sekundu prenesú veľké
množstvá údajov rôzneho typu. A práve tu hrá životne dôležitú hereckú
rolu konverzia analógového a digitálneho signálu. Vysielanie
televízie, rádia, alebo rozprávanie sa modemov, to všetko sú analógové
signály, ktoré treba nejakým spôsobom preniesť do digitálneho sveta. V
rannej dobe boli technológie čisto analógové, ale v dnešnej dobe sa
stretávame s náročným digitálnym spracovaním. Či ide o záznam zvuku
mikrofónom alebo meranie napätia voltmetrom, všade sa vynára tá istá
otázka.
%% }}}

%\subsubsection{Je možné digitalizovať analógový signál?}
\subsection{Je možné digitalizovať analógový signál?}
%% {{{ sampling theorem
Odpoveď je nie. Nie bez ďalších predpokladov. Signál môže byť
ľubovoľný a nech by sme akokoľvek rýchlo zaznamenávali, stále nemusíme
zaznamenávať dostatočne rýchlo. Taktiež nemôžeme zaznamenávať s
nekonečnou presnosťou. Toto sú limitujúce faktory, ktoré prekážajú
záznamu signálu. Zaoberajme sa preto otázkou, či za nejakých
zjednodušených predpokladov je možné rekonštruovať signál iba z
čiastočných údajov. Možno je to prekvapujúce, ale dá sa to. Základ
tejto teórie vypracovali páni Nyquist a Shannon. Sformulujme a dokážme
si preto jednu základnú vetu zo signal processingu.

\begin{veta}[Nyquist-Shannon sampling theorem]
    Nech $f$ je ľubovoľná spojitá funkcia z $\LLinf$. 
    Ak $\exists B$ také že $f$ je zhora ohraničená frekvenciou $B$,
    tak $f$ sa dá zrekonštruovať z~bodov $f(\frac{k\pi}{B})$ pre
    $k\in Z$ podľa vzorca 
    $f(x) = \sum_{l=-\infty}^{\infty}
                f(\frac{l \pi}{B})  \sinc(l \pi - B x)$.
\end{veta}

\begin{dokaz}
    Pretože signál je frekvenčne obmedzený, platí
    \begin{equation}
        f(x) = \int_{-\infty}^{\infty} F(\alpha) \exp[ \imag x \alpha]
        \dd\alpha = \int_{-B}^{B} F(\alpha) \exp[ \imag x \alpha]
        \dd\alpha
        \label{eq:nyq_band}
    \end{equation}
    Z podmienky $f\in\LLab$ vieme,
    že $F$ existuje a  $F \in \LLinf$. Taktiež
    pretože $F(x) = 0$ pre $|x| > B$, môžeme tvrdiť $F \in \LL(-B,B)$.
    Preto sa dá písať\footnote{V dôkaze používame verziu Fourierovho
    radu na intervale nie $(-\pi,\pi)$ ale na intervale $(-B,B)$.
    Veríme, že čitateľ si je sám schopný odvodiť si variácie viet,
    ktoré sme dokazovali v prvej časti publikácie}
    \begin{equation}
        F(\alpha) = \sum_{k=-\infty}^{\infty} F_k \exp[\frac{\imag \pi
        \alpha k}{B}],\quad|\alpha|<B
        \label{eq:nyq_series}
    \end{equation}
    Táto suma konverguje k $F$ v norme $\LLab$. 
    Samozrejme, mimo intervalu $(-B,B)$ konverguje k~periodickému
    rozšíreniu.
    Koeficienty $F_k$ vieme jednoznačne určiť ako analógiu vzorca
    \ref{eq:fourierove_exp_koeficienty}.
    \begin{equation}
        F_k = \frac{1}{2B} \int_{-B}^{B} F(\alpha) \exp[\frac{- \imag
        \pi k \alpha}{B}] d\alpha
        \label{eq:nyq_koef}
    \end{equation}
    Teraz porovnaním \eqref{eq:nyq_band} a \eqref{eq:nyq_koef} odvodíme        
%    \footnote{Veľmi pozorný čitateľ si mohol všimnúť, že
%        \eqref{eq:nyq_band} je ekvivalencia v $\LL(-B,B)$, ale
%        \eqref{eq:nyq_koef} je rovnosť. Preto
%        opodstatnenosť spájania týchto dvoch rovníc je vážne ohrozená.
%        Záchranu ponúka predpoklad spojitosti $f(x)$. Podľa \todo{} je
%        \eqref{eq:nyq_band} zároveň aj bodovou rovnosťou}
    
    \begin{equation*}
        F_k = \frac{1}{2B} f(- \pi k / B)
    \end{equation*}
    a dosadením do \eqref{eq:nyq_series} s použitím substitúcie $l=-k$ dostaneme
    \begin{equation}
        F(\alpha) = \frac{1}{2B} \sum_{l=-\infty}^{\infty}
            f(\frac{l \pi}{B}) \exp[-\frac{\imag l \pi \alpha}{B}]
        \label{eq:nyq_fa}
    \end{equation}
    Nakoniec, skombinovaním \eqref{eq:nyq_fa} a \eqref{eq:nyq_band}
    dostávame
    \begin{align*}
        f(x) & =  \int_{-B}^{B} \frac{1}{2B} \left(\sum_{l=-\infty}^{\infty} 
                f(\frac{l \pi}{B}) \exp[-\frac{\imag l \pi \alpha}{B}]
                \right)
                \exp[ \imag x \alpha] \dd\alpha \\
            & =  \frac{1}{2B} \int_{-B}^{B} \sum_{l=-\infty}^{\infty} 
                f(\frac{l \pi}{B}) \exp[-\frac{\imag l \pi \alpha}{B}
                + \imag x \alpha] \dd\alpha
    \end{align*}
    Potrebovali by sme zameniť poradie integrovania a sumácie.
    Označme $S_n(\alpha) = \sum_{l=-n}^{n} f(\frac{l \pi}{B})
    \exp[-\frac{\imag l \pi \alpha}{B}]$.
    Vieme, že $\lim_{n\imply \infty} S_n(\alpha) = 2 B F(\alpha)$.
    Špeciálne dosadením $\alpha=0$ dostávame
     $\lim_{n \imply \infty} S_n(0) = 2 B F(0)$.
     Preto $\exists C: \forall n: |S_n(0)| < C$.
     Lenže $|S_n(\alpha)| \le |S_n(0)|$ a teda $\forall n: C >
     S_n(\alpha)$.
     Zároveň, $\int_{-B}^B |C| d\alpha = 2 B C < \infty$.
     Preto podľa vety o dominantnej konvergencii
        \smalltodo{http://en.wikipedia.org/wiki/Dominated\_convergence\_theorem}
     môžeme zameniť poradie integrácie a sumácie.     
    \begin{align*}
       f(x) & =  \frac{1}{2B}  \sum_{l=-\infty}^{\infty} 
                f(\frac{l \pi}{B}) \int_{-B}^{B} 
                \exp[-\frac{\imag l \pi \alpha}{B}
                + \imag x \alpha] \dd\alpha \\
            & =  \frac{1}{2B} \sum_{l=-\infty}^{\infty}
                f(\frac{l \pi}{B}) 2B \frac{\sin(l \pi - B x)}{l \pi -
                B x} \\
            & =  \sum_{l=-\infty}^{\infty}
                f(\frac{l \pi}{B})  \sinc(l \pi - B x)
    \end{align*}
\end{dokaz}

\nocite{priestley}
