% vim:spell spelllang=sk
%\subsection{Kompresia videa/objemových obrázkov}
\section{Kompresia videa/objemových obrázkov}

Ako sme v~predchádzajúcej kapitole ukázali, Fourierova transformácia,
presnejšie jej odroda diskrétna kosínová transformácia, sa dá v~praxi
použiť na stratovú kompresiu obrázkov s~kompresným pomerom až do 1:10.
Môže sa preto vyskytnúť myšlienka rozšíriť dvojrozmenú transformáciu
na trojrozmernú.
V~praxi sa ukazujú dva typy trojrozmerných dát.
Prvým typom je reálne trojrozmerný obrázok. Tieto obrázky sú
väčšinou medicínskeho zamerania a~vznikajú pomocou prístrojov ako CT
(Computed tomography), NMRI (Nuclear Magnetic Resonance Imaging).
Nasnímané obrázky sú vo vysokom rozlíšení, a~tak je nutné používať kompresiu.
Problémom ostáva ale bezstratovosť. Lekárske zameranie vyžaduje vysokú
presnosť a~nemôže si preto dovoliť viditeľné straty na kvalite obrazu.

Napriek tomu sa dá použiť DCT aj tu. Minimálne je výhodnejšie ako
používaná JPEG kompresia v~rámci jednotlivých obrázkov, o~čom sa
čitateľ môže presvedčiť v \cite{medical_dct}. 
Autor prirodzene rozšíril 2D verziu na 3D verziu, 
ktorá transformuje bloky $8\cross8\cross8$ a~následne kvantizuje a~komprimuje
podobne ako pri JPEGu. V závere je ukázané, že 3D DCT má lepší
kompresný pomer v porovnaní so štandardným JPEG kódovaním jednotlivých
snímkov.

Ako sa ukazuje ďalej, výhodným spôsobom na kompresiu medicínskych dát
môže byť Waveletová transformácia. Táto transformácia má ale praktickú
nevýhodu - je veľmi náročná na spracovanie, hlavne na dekódovanie,
pretože na dekódovanie konkrétneho obrázka potrebuje spracovať omnoho
viacej informácie. Metóda, ako čiastočne odstrániť túto vadu sa dá
nájsť v \cite{wavelet3d}.

Medzi prácu s~objemovými dátami sa dá považovať aj takzvaný 3D volumetric
rendering - technika renderovania projekcie 3D údajov napríklad z~CT
alebo NMRI na obrazovku podobne ako klasické 3D renderovanie. Vieme
tak napríklad zobraziť reálnu ľudskú lebku ako sa otáča v~priestore.
Podľa \cite{volumdct} existuje raytracing algoritmus fungujúci s~3D DCT
skomprimovanými dátami, bez ich kompletnej dekompresie.
Toto je výrazný pokrok v~danej oblasti, nakoľko spracúvané dáta sú
mnohokrát veľké a~schopnosť priamo renderovať zo skomprimovaných
údajov je lákavá.

Druhým veľkým zdrojom trojrozmerných dát sú videá. Pravdupovediac,
nejde o~3D dáta v~pravom slova zmysle. Sú to 2D dáta meniace sa v~čase.
Možno sa však na ne pozerať ako na trojrozmerné dáta s $z$-ovou osou časovou 
a~nie priestorovou. Takýto pohľad vôbec nie je prekvapivý - nakoniec,
tak ako existuje priestorová súvislosť medzi jednotlivými pixelmi,
existuje aj časová súvislosť.

%\subsubsection{XYZ Video kompresia}
\subsection{XYZ Video kompresia}
XYZ video kompresia je nápadne podobná tomu, ako prebiehala JPEG
kompresia. Video sa najskôr rozdelí na $8\cross8\cross8$ "video kocky",
zachytávajúce jeden blok $8\cross8$ pixelov po dobu ôsmich snímkov.
Originálne hodnoty z~rozsahu 0..255 sa posunú na -128..127.
Každá kocka sa potom prevedie do frekvenčnej domény pomocou 3D DCT,
podľa nasledujúceho vzorca nápadne pripomínajúceho rovnicu
\eqref{eq:dct_jpeg}.
\begin{equation*}
   f(u,v,w) = \frac{1}{8} C(u) C(v) C(w)
    \sum_{x=0}^7 \sum_{y=0}^7 \sum_{z=0}^7 f(x,y,z)
        \cos\frac{\pi(x+1/2) u }{8}
        \cos\frac{\pi(y+1/2) v }{8}
        \cos\frac{\pi(z+1/2) w }{8}
\end{equation*}
Ďalej nasleduje klasická kvantizácia a entropy coding, ako sme
videli pri JPEGu.

V~čom sa XYZ video kompresia líši od klasickej video kompresie je
práve transformácia aj vo frekvenčnej doméne. Klasické video kodeky
pracujú na princípe zmien oproti predchádzajúcemu snímku,
kombinovanými s~predikciou pohybu, pretože pri pohybe celého
snímku je predchádzajúca metóda neúčinná.

Absencia predikcie pohybu tak môže XYZ kompresii robiť problémy pri
pohybe kamery. Na druhej strane DCT nepovedala posledné slovo pri
kompresii videa. Podľa Raymonda Westwatera a~Borka Furtha 
(\cite{3ddct_adaptive}), situácia nie je taká
čiernobiela. Využitím adaptívnych kvantizátorov sa im podarilo
dosiahnuť vynikajúci kompresný pomer. Algoritmus má spomínané
nedostatky pri zoomovaní a~pohybe kamery. Tieto nedostatky sú ale
vyvážené jeho zložitosťou, ktorá je omnoho menšia ako u~klasických
kodekov. Autori preto vyzdvihujú jeho využitie pri real-time kompresii
ako je videokonferencia, kde má kompresný pomer dokonca vyšší ako
klasické algoritmy, má nízku náročnosť na kompresiu a~teda na
hardware. Nároky na dekompresiu sú síce väčšie ako v~prípade
štandardnej kompresie, ale to nie je kritická závada. Kompresia videa
pomocou kosínovej transformácie má preto svoje významné miesto 
v~informatike.

\nocite{video_compression}
