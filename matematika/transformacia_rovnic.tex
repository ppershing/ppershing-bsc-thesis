\subsection{Fourierova transformácia rovníc}

\subsubsection{Intro}


V tejto sekcii ukážeme jeden pozorudohný prístup k riešeniu
diferenciálnych rovníc. Hoci to patrí medzi matematické
metódy, demonštrácia bude mať čisto fyzikálny význam. Nebudeme
zachádzať do úplnej matematickej rigoróznosti a na niektorých miestach
si zavedieme zjednodušené predpoklady\footnote{Samozrejme, daný
problém sa dá vyriešiť aj bez týchto obmedzení, ale to je na dlhší
rozhovor}. Problém ktorý budeme riešiť sa vo fyzike nazýva dieektrikum
s lineárnou pamäťou. Pre fyzikálne menej zdatných doplním, že toto je
isté priblíženie elektromagnetizmu v látkach. Ešte predtým si ale
ukážeme demonštratívny prístup na malom matematickom pieskovisku:

\subsubsection{Pieskovisko}

Pred náročnou úlohou sa rozohrejeme a rozcvičíme na nasledujúcej (nie
veľmi ťažkej) úlohe: \\
``Nájdite funkciu $y \in \LLinf$, pre ktorú
$y''(x) - b^2 y(x) = f(x)$, kde $b\in R$ a $f \in \LLinf$''.
Môžeme si všimnúť, že tu nemáme žiadne okrajové podmienky, čo môže
pôsobiť trošku nezvyčajne. Okrajové podmienky sú ale šikovne skryté v
podmienke $y \in \LLinf$ a ako uvidíme, úloha bude mať jediné
riešenie.
Označme $Y(\alpha)=\fourier[y(x)]$. Potom $\fourier[y''(x)] =
-\alpha^2 Y(\alpha)$.
Aplikujme fourierovu transformáciu na obe strany rovnice.
Dostaneme $-\alpha^2 Y(\alpha) - b^2 Y(\alpha) = F(\alpha)$.
Po úprave 
\begin{equation}
    Y(\alpha)= -\frac{F(\alpha)}{\alpha^2 + b^2}
\end{equation}
Nasleduje malý trik zvaný ``pohrab sa v gebuli a nájdi niečo vhodné''
z ktorého vzíde
$\fourier^{-1}[\frac{b}{\pi(b^2+\alpha^2)}]=e^{-b|x|}$.
Upravíme preto rovnicu do tvaru
\begin{equation}
    Y(\alpha) = -\frac{\pi}{b} \frac{b}{\pi(b^2 + \alpha^2)}
    F(\alpha)
\end{equation}
Vidíme, že pravá strana je súčin dvoch fourierových transformácii,
ktorý po aplikácii inverznej transformácie prejde na konvolúciu:
\begin{eqnarray*}
    \fourier^{-1}[Y(\alpha)] &=& \fourier^{-1}\left[
            -\frac{\pi}{b} \frac{b}{\pi(b^2 + \alpha^2)}
            F(\alpha)\right] \\
    y(x) &=&  -\frac{\pi}{b} \fourier^{-1}\left[\frac{b}{\pi(b^2 +
    \alpha^2)} F(\alpha)\right] \\
    y(x) &=& -\frac{\pi}{b} \int_{-\infty}^{\infty} e^{-b |x-t|} f(t) dt
\end{eqnarray*}
Fourierova transformácia sa teda ukazuje ako veľmi dobrý nástroj na
prevádzanie diferenciálnych rovníc na algebraické.

\subsubsection{Maxwellowe rovnice pre elektromagnetizmus}


Základné rovnice ktoré budeme potrebovať sú Maxwellove\todo{spelling}
rovnice.

\def\rt{{(\vektor{r},t)}}
\def\kw{{(\vektor{k},\omega)}}

\begin{veta} Maxwellove rovnice
\begin{eqnarray}
  \div{\vektor{D}\rt} & = & 0 \\
  \curl{\vektor{E}\rt} & = & \partial_t \vektor{B} \rt \\
  \div{\vektor{B}\rt} & = & 0 \\
  \curl{\vektor{H}\rt} & = & \partial_t \vektor{D} \rt
\end{eqnarray}
\end{veta}

K týmto rovniciam treba ešte pridať spomínanú lineárnu pamäť.

\begin{eqnarray}
  \vektor{D}\rt = \int \eps(t-t') \vektor{E}(\vektor{r},t') dt' \\
  \vektor{H}\rt = \int \mu^{-1}(t-t') \vektor{B}(\vektor{r},t') dt'
\end{eqnarray}
Huh, vstrebané? Slabšie povahy asi rovno preskočia túto kapitolu. Tí
silnejší sa asi pýtajú otázku ``A toto chceme fakt riešiť?'' Áno.
Sú to 4 diferenciálne rovnice druhého rádu + 2 integrálne rovnice o 4
premenných. Znie úplne nepredstaviteľné vymotať sa z tých nepríjemných
závislostí medzi nimi. Ako si však ukážeme, analýza sa dá zvrhnúť na
algebru.

\subsubsection{Psychická príprava}
Hovoríme o fourierovej transformácii a tak sa na ňu pripravme. V prvom
rade si ujasnime čo budeme transformovať. Odpoveď je jednoduchá -
všetko. Po tomto triezvom uvážení môžeme začať odpovedať na 3 základné
otázky. Na čo prejde divergencia, rotácia, ten blbý integrál? Zhodou
okolností odpoveď na tretiu otázku už poznáme - integrál je konvolúcia
dvoch funkcií\footnote{Náhoda? Možno, ale ee...}. Čo je fourierovou
transformáciou divergencie?

%%% Transformacia divergencie

\begin{veta}
Fourierova transformácia divergencie
$\fourier[\div{\vektor{X}}(\vektor{r})]$  je $ \imag \vektor{k} .
\fourier[\vektor{X}](\vektor{k})$.
\end{veta}
\begin{dokaz}

\def\r{(\vektor{r})}
\def\k{(\vektor{k})}

\begin{eqnarray*}
    \fourier[\div{\vektor{X}\r}] &=& \fourier[\sum_{i=1}^3 \pd{}{x_i}
    X_i\r] \\
    &=& \sum_{i=1}^3 \fourier[ \pd{}{x_i} X_i\r]  \\
    &=& \sum_{i=1}^3 \imag k_i \fourier[X_i\r] \\
    &=& \imag ( \vektor{k} . \dual{X} \k)
\end{eqnarray*}
\end{dokaz}

Vidíme, že divergencia prechádza na skalárny súčin. Rotácia sa ale
nedá zahambiť, ako ľahko môžeme overiť výpočtom:

%% Transformacia rotacie

\begin{veta}
Fourierova transformácie rotácie 
$\fourier[\curl{\vektor{X}}(\vektor{r})]$  prejde na vektorový súčin 
$ \imag \vektor{k} \cross \fourier[\vektor{X}](\vektor{k})$.
\end{veta}

\begin{dokaz}
\begin{eqnarray*}
    \fourier[\curl{\vektor{X}\rt}]_i &=& 
    \fourier[\curl{\vektor{X}\rt}_i] \\
    &=& \fourier[\sum_{j=1}^3 \sum_{k=1}^3 \eps_{ijk} \pd{}{j} X_k] \\
    &=& \sum_{j=1}^3 \sum_{k=1}^3 \eps_{ijk} \fourier[\pd{}{j} X_k] \\
    &=& \sum_{j=1}^3 \sum_{k=1}^3 \eps_{ijk} \imag k_j \fourier[X_k]\kw \\
    &=& \imag ( \vektor{k} \cross \dual{X}\kw )
\end{eqnarray*} 
\end{dokaz}

\begin{poznamka}
$\eps_{ijk}$ je Levi-Civitov symbol. Platí $\eps_{ijk}$ je 1 ak je
permutácia čísel $i,j,k$ párna, -1 ak je permutácia nepárna a 0 ak
$i=j \lor i=k \lor j=k$. Levi-Civitov symbol sa dá použiť na zápis
rotácie a vektorového súčinu pomocou dvojitej sumy, ako sme to
spravili v dôkaze. Viac o ňom sa dá dočítať v \todo{}.
\end{poznamka}


Vidíme teda, že divergencia, rotácia a konvolúcia sú ako stvorené na
Fourierovu transformáciu pretože veci značne zjednodušuje. Spravíme
preto radikálny krok a trasnformujeme celé rovnice. Jednoducho
povedané, ak $x=y$, budeme riešiť rovnicu $\fourier[x]=\fourier[y]$ a
výsledok transformujeme do pôvodnej domény.

Transformáciou rovníc dostávame omnoho jednoduchšiu verziu
\begin{eqnarray}
\imag \vektor{k} . \dual{D}\kw &=& 0 \label{eq:maxwellf1} \\
\imag \vektor{k} \cross \dual{E}\kw &=& - \imag \omega \dual{B}\kw
\label{eq:maxwellf2} \\
\imag \vektor{k} . \dual{B}\kw &=& 0 \label{eq:maxwellf3}\\
\imag \vektor{k} \cross \dual{H}\kw &=& \imag \omega \dual{D}\kw \label{eq:maxwellf4}\\
\dual{D}\kw &=& 2 \pi \dual{\eps}(\omega) \vektor{E}\kw \label{eq:linf1}\\
\dual{H}\kw &=& 2 \pi \dual{\mu^{-1}}(\omega) \vektor{B}\kw
\label{eq:linf2}
\end{eqnarray}
Od tohoto okamihu budeme automaticky predpokladať, že dané vektory sú
funkciami premenných $\vektor{k},\omega$ a nebudeme to explicitne
rozpisovať.
Taktiež, môžeme z rovníc vyškrtať imaginárnu jednotku $\imag$.
Vektorovým vynásobením \ref{eq:maxwellf2} zľava vektorom $\vektor{k}$
dostávame
\begin{equation*}
    \vektor{k} \cross (\vektor{k} \cross \dual{E}) = - \vektor{k}\cross
    \omega \dual{B}
\end{equation*}
Upravme a použime identitu $\vektor{x} \cross (\vektor{x} \cross
\vektor{y}) = \vektor{x} (\vektor{x} . \vektor{y}) - x^2
\vektor{y}$. Teraz sa dopustíme prvého zjednodušenia - predpokladajme
že $\dual{\eps}(\omega)\not=0$ a takisto
$\dual{\mu^{-1}}(\omega)\not=0$. Potom z \ref{eq:maxwellf1} a \ref{eq:linf1} dostávame
$\vektor{k}.\dual{E}=0$ a ostane nám
\begin{equation*}
    - k^2 \dual E = - \omega \vektor{k} \cross
    \frac{1}{2 \pi} \dual{\mu^{-1}}^{-1}(\omega) \dual{H}
\end{equation*}
Označme si $\dual{\mu^{-1}}^{-1}$ ako $\dual{\mu}$.

\begin{equation*}
    - \vektor{k}^2 \dual E = - \omega \vektor{k} \cross
    \frac{1}{2 \pi} \dual{\mu}(\omega) \dual{H}
\end{equation*}

Dosadením \ref{eq:maxwellf4} a následne \ref{eq:linf2} máme
\begin{equation*}
  k^2 \dual{E} = \omega^2 \dual{\mu}\dual{\eps} \dual{E}    
\end{equation*}
resp.
\begin{equation}
    \label{eq:maxwelldirac}
  (k^2 - \omega^2 \dual{\mu}(\omega)\dual{\eps}(\omega)) \dual{E} =0
\end{equation}
V tomto okamihu necháme matematiku bokom a začneme robiť matematicky
nie úplne korektné kroky (Ako by fyzik povedal - každá funkcia sa dá
spraviť spojitá a mnohonásobne diferencovateľná bez jej viditeľnej
zmeny). Predpokladajme, že rovnica 
  $k^2 - \omega^2 \dual{\mu}(\omega)\dual{\eps}(\omega)=0$ má len
  konečne veľa riešení. Výsledkom je 
\begin{itemize}
\item Ak hodnoty $\dual{E}$ v týchto bodoch sú ohraničené, po spätnej
    transformácii dostaneme nulu.
\item Hodnoty musia byť preto neohraničené. To trochu kazí matematickú
krásu, pretože také funkcie neexistujú. Avšak môžeme sa preniesť do
doménu lineárnych funkcionálov a tam nám pomôže Diracova funkcia
\item $\dual{E}\kw$ môžeme teda zapísať ako $\dual{E}\kw = \vektor{a}\kw
\delta( k^2 - \omega^2 \dual{\mu}(\omega)\dual{\eps}(\omega))$ kde
$\vektor{a}$ je vhodná funkcia.
\end{itemize}

\begin{equation}
    \vektor{E}\rt = \int \dual{E}\kw \exp\left[\imag (\vektor{k}.\vektor{r} +
    \omega t)\right]\,d^3k\;d\omega
\end{equation}
Ak označíme jednotlivé riešenia \ref{eq:maxwelldirac} ako $\omega_i, i
\in 1,\dots,N$ môžeme predchádzajúcu rovnicu prepísať ako
\begin{equation}
    \vektor{E}\rt = \int \sum_{i=1}^N
    \frac{\dual{E}(\vektor{k},\omega_i(k))}{|g'(\omega_i(\vektor{k}))|}
    \exp\left[\imag (\vektor{k} . \vektor{r} + \omega_i(\vektor{k})
    t)\right] d^3k
\end{equation}
Preznačením dostaneme všeobecné riešenie
\begin{equation}
    \label{eq:maxwellvysledok}
    \vektor{E}\rt = \int \sum_{i=1}^N \vektor{\alpha_i}(\vektor{k})
    \exp\left[\imag (\vektor{k}.\vektor{r} + \omega_i(\vektor{k})
    t)\right] d^3 k
\end{equation}

Na záver - všimnime si, čo dostaneme v prípade štandardnej vlnovej
rovnice. V tomto prípade $\vektor{D},\vektor{H}$ závisia iba od
aktuálnej hodnoty $\vektor{E},\vektor{B}$. V našom prípade to môžeme
formálne dosiahnuť ak $\mu^{-1}(t-t')=\mu_0^{-1} \delta(t-t')$ a
$\eps(t-t')=\eps_0 \delta(t-t')$.
Potom $\dual{\eps}(\omega) = \eps_0$ a $\dual{\mu}(\omega)=\mu_0$.
Rovnica \ref{eq:maxwelldirac} prejde na
\begin{equation}
    k^2 - \omega^2 \eps_0 \mu_0 = 0
\end{equation}
Dávajúca riešenie $\omega_{1/2}=\pm \frac{|\vektor{k}|}{\sqrt{\eps_0
\mu_0}}$. Využijeme $\sqrt{\eps_0 \mu_0}=c^{-1}$ a dosadíme do
\ref{eq:maxwellvysledok}.

\begin{equation}
 \vektor{E}\rt = \int \vektor{\alpha}(\vektor{k}) \exp \left[\imag(
 \vektor{k}.\vektor{r} - |\vektor{k}| c) \right] +
 \vektor{\beta}(\vektor{k}) \exp \left[\imag(
 \vektor{k}.\vektor{r} + |\vektor{k}| c)\right] d^3k
\end{equation}
Výsledok neni nič iné ako rovnica elektromagnetickej vlny vo vákuu.

\subsubsection{Conclusion}
Fourierova transformácia rovníc je veľmi silný nástroj na riešenie
diferenciálnych rovníc. Zjednodušený návod na použitie je 
\begin{itemize}
\item pôvodný diferenciálny problém
\item Fourierova transformácia rovníc $\imply$ nový algebraický
problém
\item vyriešime algebraický problém
\item riešenie + inverzná Fourierova transformácia $\imply$ riešenie
pôvodnej úlohy
\end{itemize}

Znie to jednoducho a úderne, prečo teda neriešiť všetky diferenciálne
rovnice takto? Odpoveďou sú 2 podmienky, ktoré musí spĺňať problém a
nie všdy platia.
\begin{itemize}
\item Pôvodný problém musí byť v správnej doméne ($\LLab,\LLinf$)
\item Po vyriešení transformovanej úlohy ju musíme vedieť
transformovať späť. Žiaľ, toto častokrát nevieme spraviť kvôli
zložitosti výsledku
\end{itemize}
