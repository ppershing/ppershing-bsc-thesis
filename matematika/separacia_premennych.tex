\subsection{Separácia premenných}

Táto užitočná metóda na riešenie diferenciálnych rovníc uzrela prvý
krát svetlo sveta práve keď sa Fourier začal zaoberať riešením rovnice
tepla. Ukážeme si ju najskôr na jednorozmernom prípade vedenia tepla,
a následne na dvojrozmernom riešení Laplacovej rovnice obdĺžnikovej
oblasti.

\subsubsection{Rovnica vedenia tepla}

Uvažujme kovovú tyč. Aby sme si zjednodušili život, budeme používať
normalizované a teda bezrozmerné fyzikálne jednotky. Tyč bude mať
dĺžku 1 a bude tepelne izolovaná od okolia. Na začiatku bude mať
teplotu $T_0$ a jeden jej koniec ochladíme priložením na konštantnú
teplotu $T_1<T_0$. Formálne zapísané
$T(x,0) = T_0, 0<x\le1$ a $T(0,t)=T_1, 0\le t$. Druhý koniec je izolovaný a preto
cezeň nejde žiadny tepelný tok - podmienka ekvivalentná
$\pd{T}{x}(1,t)=0$. A finálne, posledná podmienka, podmienka
vedenia tepla, ako môže každý fyzik potvrdiť je
\begin{equation}
    \pd{T}{t} = \pdd{T}{x}, \quad0<x<1
    \label{eq:heat}
\end{equation}

Pojem separácie premenných je intuitívny. Začneme predpokladom
nezávislosti premenných. Presnejšie povedané, nech
$T(x,t) = X(x) F(t)$. Substituovaním do \ref{eq:heat}
dostávame
\begin{equation}
    \frac{X''}{X} = \frac{F'}{F} = - k^2
    \label{eq:separation}
\end{equation}
kde $-k^2$ je separačná konštanta. Znamienko mínus sme uhádli
fyzikálnou intuíciou - tyč sa bude postupne ochladzovať.
Pomôžeme si trochou analýzy:
\begin{lema}
    Rovnica $F'(t) = -k^2 F(t)$ má riešenie v tvare 
    $F(t) = C_0 e^{-k^2 t}$ kde $C_0 \in R$.
    Ďalej, rovnica $X'' + k^2 X = 0$ má riešenie v tvare
    $X(x) = C_0 \sin(k x) + C_1 \cos(k x)$. Navyše, tieto riešenia sú
    jediné riešenia danej rovnice.
\end{lema}
\begin{dokaz}
    Dôkaz je štandardnou súčasťou kurzu diferenciálnych rovníc a
    nebudeme ho uvádzať.
\end{dokaz}
Vyžitím tejto lemy a rovnice \ref{eq:separation} dostávame
\begin{equation}
    T(x,t) = (C_0 \sin (k x) + C_1 \cos (k x))(C_2 e^{-k^2 t})
\end{equation}
Okrajová podmienka $T(0,t) = 0$ a nenulovosť $e^{-k^2 t}$ 
implikuje $X(0)=0$ a teda $C_1 = 0$.
Podobne, podmienka nulového tepelného toku cez izolovaný koniec tyče
na druhom konci vedie k
$X'(1) = C_0 k \cos k = 0$. Každé nenulové riešenie potom musí spĺňať
\begin{equation}
    k = \frac{\pi}{2} + n \pi, \quad n \in \Z
\end{equation}
Výsledkom je teda spočítateľne veľa riešení $T$ v tvare $T=XF$,
menovite
$T_n = A_n \exp\left(-\pi^2 (n+\frac{1}{2})^2 t \right)
            \sin \left( \pi (n + \frac{1}{2}) x \right), n\in \Z$
Pretože daný problém je lineárny, každá lineárna kombinácia týchto
riešení musí byť riešením. Všeobecné riešenie je teda
\begin{equation}
T = \sum_{n=0}^{\infty} A_n \exp\left(-\pi^2 (n+\frac{1}{2})^2 t \right)
            \sin \left( \pi (n + \frac{1}{2}) x \right), n\in \Nz
    \label{eq:heat_expansion}
\end{equation}
kde sme využili fakt $\sin(-x) = -\sin(x)$ a teda sme s čistým
svedomím vynechali záporné indexy $n$.
Ostáva teraz jediná otázka, pravdupovediac tá najťažšia. Dá sa každé
riešenie \ref{eq:heat} zapísať ako \ref{eq:heat_expansion}. A v tomto
momente prichádzajú na rad Fourierove rady. Výsledok môžeme dosiahnuť
2 spôsobmi. Prvým je ručný výpočet
\begin{equation}
    \int_0^1 \sin\left(\pi(n+\frac{1}{2})x\right)
             \sin\left(\pi(m+\frac{1}{2})x\right) \dd x =
    \left\{
    \begin{array}{l l}
        0& m\not=n \\
        \frac{1}{2} & m=n
    \end{array}
    \right.
\end{equation}
a následná expanzia na otrogonálny systém. Poučnejšie pre čitateľa ale
bude nasledujúce odvodenie:
