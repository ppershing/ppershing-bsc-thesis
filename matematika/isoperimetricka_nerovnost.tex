\subsection{Isoperimetrická nerovnosť}

Fourierova transformácia má mnoho prekvapivých použití. Ako sme sa už
mohli presvedčiť, jej metódy sa dajú použiť na riešenie rôznych
problémov. Ako ďalšiu ukážku jej moci si dokážeme intuitívnu vetu
ktorá ukazuje vzťah medzi obdovom a obsahom ľubovoľného útvaru.
Ešte predtým ale uvedieme jednu lemu:
\begin{lema}
  Nech $C$ je hladká uzavretá krivka na ploche. Potom ju možno popísať
  ako $(x(t),y(t))$ kde $t\in[-\pi,\pi]$ a $x(-\pi)=x(\pi),
  y(-\pi)=y(\pi)$ a navyše platí
  \begin{equation}
    \left(\frac{\dd x}{\dd t}\right)^2 +
    \left(\frac{\dd y}{\dd t}\right)^2 = konst.
  \end{equation}    
  \label{lema:ekvidist_parametric}
\end{lema}
\begin{dokaz}
    Krivka je hladká a môžeme ju parametricky popísať ako
    \begin{equation}
        \left(x(t),y(t)\right), \quad t\in[-\pi,\pi]
    \end{equation}
    kde $x,y$ sú spojité funkcie spĺňajúce $x(-\pi)=x(\pi), y(-\pi)=y(\pi)$.
  Označme
   \begin{equation}
        s(t) = \int_{-\pi}^t \sqrt{ (x'(w))^2 + (y'(w))^2} \dd w
   \end{equation}
   Môžeme si všimnúť, že $s(t)$ je rastúca a preto je monotónna
   (Funkcie $x'(w)$ a $y'(w)$ nemôžu byť obe nulové na spoločnom intervale,
   lebo by to bolo v spore s monotónnosťou parametrického popisu)-
   Zaveďme
    \begin{equation}
        \dual{t} = -\pi + \frac{2 \pi s(t)}{s(\pi)}
    \end{equation}
   Podľa pozorovania o monotónnosti, vieme že zobrazenie
   $t\imply \dual{t}$ je bijekcia a preto existujú (spojité funkcie)
   $\dual{x},\dual{y}$ spĺňajúce
   \begin{equation}
        (\dual{x}(\dual{t}),\dual{y}(\dual{t})) = (x(t),y(t))
   \end{equation}
   Potom
   \begin{equation}
        x'(t) = \dual{x}'(\dual{t}) =
        (\dual{x}(-\pi + \frac{2 \pi s(t)}{s(\pi)})' =
        \dual{x}'(-\pi + \frac{2 \pi s(t)}{s(\pi)} 
        \frac{2\pi}{s(\pi)}
        \sqrt{ (x'(t))^2 + (y'(t))^2)} =
        \dual{x}'(\dual{t}) \frac{2\pi}{s(\pi)}
        \sqrt{ (x'(t))^2 + (y'(t))^2)}
   \end{equation}
   a po troche úprav dostávame finálnu podobu   
   \begin{equation}
    \left(\frac{\dd \dual{x}}{\dd \dual{t}}\right)^2 +
    \left(\frac{\dd \dual{y}}{\dd \dual{t}}\right)^2 = 
    \frac{s(\pi)}{2\pi}^2
    \label{eq:ekvidist_parametric}
   \end{equation}
\end{dokaz}

\begin{veta}[Isoperimetrická nerovnosť]
Nech $C$ je hladká uzavretá krivka na ploche. Nech $S$ je jej obsah a $O$ je
jej obvod. Potom
    \begin{equation}
      4\pi S \le O^2
    \end{equation}
\end{veta}

\begin{dokaz}
    Podľa lemy \ref{lema:ekvidist_parametric} existujú
    $x(t),y(t), t\in[-\pi,\pi]$ spĺňajúce rovnicu
    \ref{eq:ekvidist_parametric}.
    Vzorce pre objem a obvod sa pomocou nich dajú vyjadriť ako
    \begin{align}        
        O =& \intpipi \sqrt{\left(\frac{\dd x}{\dd t} \right)^2 +
            \left(\frac{\dd y}{\dd t} \right)^2} \dd t = s(\pi) \\
        S =& \intpipi x \frac{\dd y}{\dd t} \dd t    
    \end{align}
    Pretože $x,y$ sú hladké \todo{niekde definovat hladkost}, môžeme
    ich podľa vety \todo{ref: rovnomerna konvergencia}
    zapísať ako
    \begin{align}
        x(t) &= \frac{a_0}{2} + \sum_{n=1}^{\infty} \left(
               a_n \cos(n t) + b_n \sin(n t) \right) \\
        y(t) &= \frac{c_0}{2} + \sum_{n=1}^{\infty} \left(
               c_n \cos(n t) + d_n \sin(n t) \right)
    \end{align}
    Podľa \todo{ref: derivacia radu} navyše platí
    \begin{align}
        x'(t) &= \sum_{n=1}^{\infty} n \left(
            -a_n \sin(nt) + b_n \cos(n t) \right) \\
        x'(t) &= \sum_{n=1}^{\infty} n \left(
            -c_n \sin(nt) + d_n \cos(n t) \right)
    \end{align}
    \todo{perseval} implikuje s použitím
    \todo{nasledujuca vec neplati}
    \begin{equation}
    \begin{split}
        \frac{O^2}{2\pi} &= \intpipi ((x')^2 + (y')^2) \dd t \\
        &=
        \intpipi \sum_{n=1}^{\infty} n^2 \left(
            a_n ^2 \sin^2(nt)+ b_n^2 \cos^2(nt)
            -2 a_n b_n \sin(nt) \cos(nt) +
            c_n ^2 \sin^2(nt)+ d_n^2 \cos^2(nt)
            -2 c_n d_n \sin(nt) \cos(nt) \right) \\
        &= \pi \sum_{n=1}^\infty n^2 \left( a_n^2 + b_n^2 + 
            c_n^2 + d_n^2 \right)
    \end{split}
    \end{equation}
    Kde sme si mohli dovoliť vymeniť poradie integrácie a sumácie
    a využili sme \todo{ref:basic orthogonality}.
    Na druhej strane máme
    \begin{equation}
    \begin{split}
        S =& \intpipi x(t) y'(t) \dd t \\
         =& \frac{1}{4}
            \intpipi \left(
                    (x(t) + y'(t))^2 - 
                    (x(t) - y'(t))^2 \right) \dd t \\
         =& \frac{1}{4} \intpipi
           \left(
            \frac{a_0}{2} + \sum_{n=1}^{\infty}
              (a_n - c_n) \cos(nt) + (b_n + d_n) \sin(nt)
            \right)^2 \\&-
           \left(
            \frac{a_0}{2} + \sum_{n=1}^{\infty}
              (a_n + c_n) \cos(nt) + (b_n - d_n) \sin(nt)
            \right)^2 \\
         =& \text{po niekoľkých úpravách a využití ortogonality a
        integrovaní} \\
         =& \pi \sum_{n=1}^{\infty} n \left( a_n d_n - b_n c_n \right)
    \end{split}
    \end{equation}
    Šikovnými úpravami môžeme teda dospieť k záveru
    \begin{equation}
        \frac{O^2}{2\pi} - 2S = 
         \pi \sum_{n=1}^{\infty} \left[
            n(a_n - d_n)^2 + n (b_n+c_n)^2 + n(n-1)(a_n^2 + b_n^2 +c_n
            ^2 + d_n ^2)
         \right]
    \end{equation}
    Pretože pravá strana je nezáporná, môžeme overiť tvrdenie vety.
    Na druhej strane 
    $\frac{O^2}{2\pi} - 2S = 0$ vtedy a len vtedy ak
    
    \begin{align}
        a_n^2 + b_n^2 + c_n^2 + d_n^2 &= 0, \quad n>1 \\
        a_1 - d_1 &=0 \\
        b_1 + c_1 &=0
    \end{align}
    Toto implikuje
    \begin{align}
        x(t) &= a_0  + a_1 \cos(t) - c_1 \sin(t) \\
        y(t) &= c_0  + c_1 \cos(t) + a_1 \sin(t)
    \end{align}
    a ľahko sa dá presvedčiť
    $(x(t)-a_0)^2 + (y(t) - c_0)^2 = konšt.$ a teda daná krivka je
    kružnica.
\end{dokaz}

\todo{dokaz bez smoothness, dalsie aplikacie v geometrii - literatura}
%http://books.google.com/books?id=qT54mGvjBSwC&pg=PA133&lpg=PA133&dq=isoperimetric+inequality+fourier+series&source=bl&ots=z9p_gyUp6y&sig=yvFAPh8mCVrAa6pi_hIlT6hqLkc&hl=en&ei=8NO2SemmMI3U-QaOmMGICw&sa=X&oi=book_result&resnum=3&ct=result
\todo{http://www.sosmath.com/fourier/fourier5/fourier5.html}
