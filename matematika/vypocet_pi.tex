% vim:spell spelllang=sk
\subsection{Výpočet $\pi$}

Vráťme sa k príkladu \todo{ref}.
Vieme, že
\begin{equation}
    S_n(x) = \frac{1}{2} + \frac{2}{\pi} 
        \sum_{m=1} \frac{\sin((2m-1)x}{2m-1}
\end{equation}
konverguje bodovo\footnote{spomeňme si na vetu \todo{ref}} 
k funkcii $f(x)=0, x\in (-\pi,0)$ a $f(x)=1, x\in (0,\pi)$.
To znamená, že čiastočné súčty konvergujú k hodnote $f(x)$ špeciálne
pre prípady $x=\frac{\pi}{4}$ a $x=\frac{\pi}{2}$. Dosadením do radu
dostávame
\begin{align}
   1= f(\frac{\pi}{4}) = \frac{1}{2} + \frac{2}{\pi}
    \sum_{m=1}^{\infty} \frac{\frac{\pi}{4} (2m-1)}{2m-1} \\
   1= f(\frac{\pi}{2}) = \frac{1}{2} + \frac{2}{\pi}
    \sum_{m=1}^{\infty} \frac{\frac{\pi}{2} (2m-1)}{2m-1} \\
\end{align}
resp.
\begin{align}
   \frac{1}{2}= \frac{2}{\pi}
    \sum_{m=1}^{\infty} \frac{\frac{\sqrt{2}}{2} (-1)^{\floor{m/2+1}}}{2m-1} \\
   \frac{1}{2} = \frac{2}{\pi}
    \sum_{m=1}^{\infty} \frac{(-1)^{2m-1}}{2m-1} \\
\end{align}
Prepísané do krajšej podoby
\begin{align}
    \frac{\pi}{4} = 1 - \frac{1}{3} + \frac{1}{5} - \frac{1}{7} +
    \frac{1}{9} - \frac{1}{11} + \cdots \\ 
    \frac{\pi}{2 \sqrt{2}} = 1 + \frac{1}{3} - \frac{1}{5} - \frac{1}{7} +
    \frac{1}{9} + \frac{1}{11} - \cdots \\
\end{align}
