% vim:spell spelllang=sk
\chapter{Záver}

V~prvej časti práce je popísaná Fourierova transformácia. Po krátkom
historickom úvode nasleduje pomerne dlhé uvedenie do problematiky
Fourierových radov. Rozoberáme ich princípy, matematické vlastnosti,
konvergenciu a~existenciu inverznej transformácie. Ďalej sa venujeme
Gibbsovmu fenoménu, ktorý ukazuje vlastnosti konvergencie nielen pre
rady ale aj pre samotnú spojitú a~diskrétnu transformáciu. Po tomto
matematickom úvode nasleduje rýchly pohľad na spojitú a~diskrétnu
Fourierovu transformáciu, viacrozmerné verzie a~nakoniec príbuzné
transformácie.
Druhá časť publikácie je venovaná čisto použitiam Fourierovej 
a~diskrétnej kosínovej transformácie v~informatike. Zaoberáme sa tam
otázkou digitalizácie, spracovaním a~kompresiou digitálneho signálu -
či už zvuku alebo obrazu či videa. Zároveň ukazujeme rôzny výhody 
a~nevýhody použitých transformácii a~čitateľovi vysvetľujeme prečo sa
použila práve daná variácia Fourierovej transformácie. Kapitolu
zakončujeme aplikáciami v~teórii algoritmov a~bezpečnosti ako nástroj
na rýchle násobenie polynómov a~generovanie bezpečných hashovacích
funkcií.

Hlavným prínosom práce je systematické pozbieranie a~overenie jednotlivých
aplikácii Fourierovej transformácie a~jej modifikácii v~informatike 
s~dôrazom na ich vzájomné prepojenie. Autor sám venoval nezanedbateľné
úsilie na vyskúšanie týchto aplikácii a~problémov, s~ktorými musia
čeliť. Práca ale rozsahovo zďaleka
nepokrýva všetky aspekty Fourierovej transformácie aké pôvodne chcela
spomenúť. Z~priestorových dôvodov bol autor nútený
vynechať množstvo ďalších zaujímavých použití. Do budúcnosti by sa
chcelo na práci pokračovať, spomenúť matematické a~fyzikálne využitia,
ktoré sú mnohokrát až prekvapivé. Taktiež je autorovým cieľom 
v~budúcnosti rozobrať jednotlivé algoritmy 
na výpočet rýchlej Fourierovej transformácie,
popísať ich výhody, nevýhody a~ozrejmiť situácie, v~ktorých sa
používajú.
