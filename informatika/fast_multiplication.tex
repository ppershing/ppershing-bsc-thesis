% vim:spell spelllang=sk
\subsection{Rýchle násobenie polynómov}

V tejto sekcii si porozprávame niečo o použití Fourierovej
transformácie na rýchle násobenie polynómov. Naša púť začne hľadaním
analógie Fourierovej transformácie v konečných poliach.

\smalltodo{odkomentuj nasledujuci text}
%Ešte predtým
%však zhrnieme klasické metódy násobenia polynómov.
%
%\subsubsection{Klasický prístup k násobeniu polynómov}
%Asi najtriviálnejším spôsobom na násobenie polynómov je použitie
%``násobenia zo školy'', kedy postupne vynásobíme prvý polynóm
%každou cifrou druhého, medzivýsledky podpisujeme a nakoniec sčítame.
%Táto implementácia má časovú zložitosť $O(N^2)$ a pamäťovú zložitosť
%tiež $O(N^2)$. Veľmi jednoducho algoritmus ale vieme upraviť tak,
%aby sčítaval medzivýsledky priebežne, čím dostávame pamäťovú zložitosť
%$O(N)$.
%\todo{python kod}
%
%Rýchlejší prístup nám ponúka dynamické programovanie. Nech sú obe
%čísla $x,y$ veľkosti $n$. Potom ich môžeme zapísať ako
%\begin{equation*}
%x= a. 10^{n/2} + b, \quad y=c. 10^{n/2} + d
%\end{equation*}
%Štandardným vynásobením dostávame výraz
%\begin{equation*}
%x.y = a.c 10^n + (a.d+b.c) 10^{n/2} + b.d.
%\end{equation*}
%ktorý potrebuje 4 násobenia veľkosti $n/2$.
%Pravá strana sa ale dá šikovne prepísať na 
%\begin{equation*}
%x.y = ac 10^n + (ac + bd + (a-b)(d-c)) 10^{n/2} + bd
%\end{equation*}
%Môžeme si všimnúť, že sme oproti pôvodnému zápisu ušetrili jedno násobenie.
%Dostávame teda 3 násobenia veľkosti $n/2$ a 6 sčítaní/odčítaní (
%veľkosti maximálne $n$).
%Tento postup môžeme rekurzívne aplikovať a ak by sme napísali
%rekurentnú reláciu, dostali by sme $T(1)=1, \quad T(n)=3 T(n/2) +
%c.n$. Podľa Master theorem \cite[p. 73-90]{Introduction} dostávame
%$T(n) < \overline{c} n^{log_2 3}$ pre nejakú konštantu $\overline{c}$. 
%Algoritmus má tým pádom časovú zložitosť $O(n^{1.59})$.
%
%\begin{poznamka}
%Bližšie informácie o tomto algoritme sa dajú nájsť v
% \cite[p. 118-119]{CompAlg} pod názvom Karatsubov algoritmus.
%\end{poznamka}
%\todo{python karatsuba}

