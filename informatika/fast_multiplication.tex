% vim:spell spelllang=sk
%\subsection{Rýchle násobenie polynómov}
\section{Rýchle násobenie polynómov}

V~tejto sekcii si porozprávame niečo o~použití Fourierovej
transformácie na rýchle násobenie polynómov. Naša púť začne hľadaním
analógie Fourierovej transformácie v~konečných poliach.

\smalltodo{odkomentuj nasledujuci text}
%Ešte predtým
%však zhrnieme klasické metódy násobenia polynómov.
%
%\subsubsection{Klasický prístup k násobeniu polynómov}
%Asi najtriviálnejším spôsobom na násobenie polynómov je použitie
%``násobenia zo školy'', kedy postupne vynásobíme prvý polynóm
%každou cifrou druhého, medzivýsledky podpisujeme a nakoniec sčítame.
%Táto implementácia má časovú zložitosť $O(N^2)$ a pamäťovú zložitosť
%tiež $O(N^2)$. Veľmi jednoducho algoritmus ale vieme upraviť tak,
%aby sčítaval medzivýsledky priebežne, čím dostávame pamäťovú zložitosť
%$O(N)$.
%\todo{python kod}
%
%Rýchlejší prístup nám ponúka dynamické programovanie. Nech sú obe
%čísla $x,y$ veľkosti $n$. Potom ich môžeme zapísať ako
%\begin{equation*}
%x= a. 10^{n/2} + b, \quad y=c. 10^{n/2} + d
%\end{equation*}
%Štandardným vynásobením dostávame výraz
%\begin{equation*}
%x.y = a.c 10^n + (a.d+b.c) 10^{n/2} + b.d.
%\end{equation*}
%ktorý potrebuje 4 násobenia veľkosti $n/2$.
%Pravá strana sa ale dá šikovne prepísať na 
%\begin{equation*}
%x.y = ac 10^n + (ac + bd + (a-b)(d-c)) 10^{n/2} + bd
%\end{equation*}
%Môžeme si všimnúť, že sme oproti pôvodnému zápisu ušetrili jedno násobenie.
% Dostávame teda 3 násobenia veľkosti $n/2$ a 6 sčítaní/odčítaní (
%veľkosti maximálne $n$).
%Tento postup môžeme rekurzívne aplikovať a ak by sme napísali
%rekurentnú reláciu, dostali by sme $T(1)=1, \quad T(n)=3 T(n/2) +
%c.n$. Podľa Master theorem \cite[p. 73-90]{Introduction} dostávame
%$T(n) < \overline{c} n^{log_2 3}$ pre nejakú konštantu $\overline{c}$. 
%Algoritmus má tým pádom časovú zložitosť $O(n^{1.59})$.
%
%\begin{poznamka}
%Bližšie informácie o tomto algoritme sa dajú nájsť v
% \cite[p. 118-119]{CompAlg} pod názvom Karatsubov algoritmus.
%\end{poznamka}
%\todo{python karatsuba}

Táto kapitola sa bude zaoberať problémom rýchleho násobenia polynómov.
Rýchle násobenie polynómov samozrejme súvisí aj s~rýchlym násobením
veľkých čísel a~taktiež delením. Preto myšlienky tejto kapitoly majú
 nesmierny praktický význam pri veľkých výpočtoch.


Najjednoduchší prístup k~násobeniu polynómov dáva časovú zložitosť
$O(n^2)$. O~trochu lepšie je na tom Karatsubov algoritmus, ktorý vie
násobiť v~čase $O(n^{\log_{2}3})=O(n^{1.59})$. Ani táto časová
zložitosť ale nie je postačujúca. Preto sa treba pozrieť na násobenie
polynómov z iného pohľadu. Jeden takýto prístup je "však je to konvolúcia"
a~prehlásiť, že ju vieme vypočítať za pomoci Fourierovej
transformácie. My sa však vydáme inou cestou, aby sme ukázali reálny
súvis Fourierovej transformácie v~poli $F$ s~okruhom polynómov nad
týmto poľom.

Formálna definícia problému:
Nech $p(x)=\sum_{i=0}^{n} p_i x^i, q(x) = \sum_{j=0}^{m} q_j x^j$.
Potom $r(x) = p(x) q(x) = \sum_{k=0}^{m+n+1} r_k x^k$ kde
$r_k = \sum_{i=0}^{min(k,n)} p_i q_{k-i}$.

Pátranie po algoritme začneme nasledujúcim veľmi užitočným tvrdením
\begin{lema}
 Nech $p(x)$ je polynóm stupňa $n$ a~nech má aspoň $n+1$ nulových
 bodov. Potom $p(x)=0$.
 \label{lema:zero_poly}
\end{lema}

\begin{lema}[O~interpolácii]
 Majme daných $n+1$ bodov $(x_0,y_0), (x_1,y_1), \dots, (x_n,y_n)$,
 takých že pre $i\not=j, x_i\not=x_j$.
 Potom existuje polynóm $p$ stupňa najviac $n$ taký, že
 $\forall i\in 0,1,\dots n: p(x_i)=y_i$.
\label{lema:interpolacia}
\end{lema}
\begin{dokaz}
 Definujme 
 \begin{equation*}
    l_j(x) = \prod_{i=0,i\not=j}^{n} \frac{x-x_i}{x_j-x_i}
 \end{equation*}
 $l_j$ je polynóm stupňa $n$ a~platí 
 \begin{equation*}
    l_j(x_i) = \left\{
            \begin{array}{ll}
                1&i=j\\
                0&i\not=j
            \end{array}
            \right.
 \end{equation*}
 Potom 
 \begin{equation}
 p(x) = L(x) = \displaystyle \sum_{j=0}^{n} y_j l_j(x)
   \label{eq:lagrange_interpolation}
 \end{equation}
\end{dokaz}

\begin{lema}
Nech $p(x)$ je polynóm stupňa $n$, $x_0,x_1,\dots,x_n$ je $n+1$
rôznych čísel, v~ktorých chceme polynóm vypočítať.
Potom v~okruhu polynómov stupňa najviac $n$ existuje bijekcia medzi $p(x)$
a $(p(x_0),p(x_1),\dots,p(x_{n}))$.
\end{lema}
\begin{dokaz}
Nech $p,q$ sú dva rôzne polynómy také, že $\forall i\in 0,1,\dots n:
p(x_i) = q(x_i)$. Potom $r=p-q$ má $n+1$ nulových bodov, je stupňa
najviac $n$ a~teda podľa lemy \ref{lema:zero_poly} dostávame $p=q$.
Zobrazenie je teda injektívne. Lema \ref{lema:interpolacia} ale
zároveň hovorí, že dané zobrazenie je surjektívne.
\end{dokaz}

\begin{poznamka}
  Predchádzajúca lema nám vlastne hovorí, že existuje jediný
  interpolačný polynóm stupňa nanajvýš $n$. Jeho zápis vo forme
  rovnice \eqref{eq:lagrange_interpolation} sa nazýva Lagrangeova
  forma interpolačného polynómu.
\end{poznamka}
Hlavná myšlienka predchádzajúcich dvoch liem spočíva v~transformovaní
problému násobenia polynómov z domény koeficientov do domény funkčných
hodnôt. Ak máme pre dva polynómy vypočítané hodnoty v~rôznych bodoch, násobenie
polynómov je úplne jednoduché - násobíme po dvojiciach dané hodnoty 
a~dostaneme hodnoty súčinu $p(x)q(x)$. Preto, na celý problém násobenia
polynómov sa môžeme pozerať aj nasledovne:

Nech $p,q$ sú polynómy stupňov $n,m$. Nech $x_0,x_1,\dots, x_{n+m}$ je
$n+m+1$ rôznych čísel. Potom násobenie polynómov môžeme spraviť
nasledovne
\begin{itemize}
  \item Vypočítajme hodnoty $p(x_i), q(x_i)$ pre $i\in0,1,\dots n+m$.
  Túto nazveme vyhodnocovanie polynómov.
  \item Vypočítame hodnoty $r(x_i) = p(x_i) q(x_i)$. Výpočet vieme
  spraviť v~lineárnom čase a~fázu nazveme násobenie/konvolúca.
  \item Z hodnôt $r(x_i)$ vypočítame postupne hodnoty $r_j, j\in
  0,1,\dots n+m$. Fázu nazývame interpolácia.
\end{itemize}

Na prvý pohľad sme si nepomohli, pretože prvú a~poslednú fázu nevieme
robiť veľmi rýchlo. Ako sa však ukáže neskôr, vhodná voľba hodnôt
$x_i$ ale môže výrazne urýchliť výpočet.

%\subsubsection{Rýchle vyhodnocovanie polynómov}
\subsection{Rýchle vyhodnocovanie polynómov}

Majme polynóm $p(x)=p_0 + p_1 x + \dots + p_{n-1} x^{n-1}$ stupňa $n-1$.
Jeho vyhodnotenie v~bode $x_i$ vieme
Hornerovou metódou vypočítať v~čase $O(n)$. Vo všeobecnosti teda
vyhodnotenie v $n$ bodoch nepôjde rýchlejšie ako $O(n^2)$.

Ak je však $n$ párne, môžeme $p$ zapísať ako
\begin{equation}
    p(x) = a(x^2) + x b(x^2)
    \label{eq:rozklad_polynomu}
\end{equation}
kde
\begin{align*}
    a(x) &= a_0 + a_2 x^2 + a_4 x^4 + \dots + a_{n-2} x^{n/2-1} \\
    b(x) &= a_1 + a_3 x^2 + a_5 x^4 + \dots + a_{n-1} x^{n/2-1}
\end{align*}
Všimnime si, že oba polynómy $b,c$ majú najviac polovičný
stupeň.
To čo je na rovnici \eqref{eq:rozklad_polynomu} zaujímavé a~čo využijeme
je fakt $(x)^2 = (-x)^2$.

\begin{lema}
  Nech $\{x_0, \dots, x_{n-1}\}$ množina $n$ bodov spĺňajúca symetrickú
  podmienku $\forall i \in 0,\dots,\frac{n}{2}-1: x_{i+n/2}=-x_i $.
  Pokiaľ $T(n)$ je časová zložitosť vyhodnocovania polynómu stupňa
  $n-1$ na týchto $n$ bodoch, potom
  $T(n) = 2 T(\frac{n}{2}) + c \frac{n}{2}$.
  \label{lema:cas_rozklad_polynomu}
\end{lema}
\begin{dokaz}
  Platí $x_0^2 = x_{n/2}^2, x_1^2 = x_{n/2+1}^2, \dots$ a~preto
  máme len $n/2$ rôznych štvorcov. Použitím rovnice
  \eqref{eq:rozklad_polynomu} vieme skombinovať celý výsledok ako
  $n/2$ násobení na získanie štvorcov,  $2$ vyhodnocovania na sade $n/2$
  bodov a~následne $n/2$ sčítaní resp. odčítaní, na skombinovanie
  výsledkov.
\end{dokaz}

\begin{definicia}[Supersymetrická množina]
  Nech $\{x_0, \dots, x_{n-1}\}$ množina $n$ bodov.
  Ak táto množina spĺňa symetrickú podmienku a~navyše aj množiny
  $\{x_0^2, x_1^2, \dots, x_{n/2-1}^2 \},
   \{x_0^4, x_1^4, \dots, x_{n/4-1}^4\}, \dots, \{x_0^{n/2},
   x_1^{n/2}\}$, danú množinu nazveme supersymetrickú.
\end{definicia}

\begin{lema}
    Nech $p$ je polynóm stupňa $n$ a $X$ je supersymetrická množina.
    Potom vyhodnotenie polynómu $p$ na množine $X$ má časovú zložitosť
    $O(n\log n)$.
\end{lema}
\begin{dokaz}
  Nakoľko je množina supersymetrická, lemu
  \ref{lema:cas_rozklad_polynomu} môžeme aplikovať rekurzívne.
  Teda
  $T(n) = 2 T(\frac{n}{2}) + c \frac{n}{2},
   T(\frac{n}{2}) = 2 T(\frac{n}{4}) + c \frac{n}{4}, \dots$.
  Podľa Master theorem je potom $T(n) \in O(n\log n)$.
\end{dokaz}

Jediné, čo ostáva je nájsť nejakú supersymetrickú množinu. Táto
množina v~prvom rade musí mať veľkosť $n=2^k$ pre nejaké $k\in\Z$.

\begin{definicia}
  Nech $\omega$ je element poľa $F$. $\omega$ nazveme $n$-tou
  odmocninou jednotky pokiaľ
   $ \omega^n = 1$ a $ \forall 0<i<n: \omega^i\not=1$.
\end{definicia}

\begin{lema}
 Nech $\omega$ je $2^k$-ta odmocnina jednotky.
 Potom množina $\{\omega^0, \omega^1, \omega^2, \dots
 \omega^{2^k-1}\}$ je supersymetrická množina.
\end{lema}
\begin{dokaz}
 $(\omega^{n/2+i})^2 = \omega^{n+2i} = \omega^n \omega^{2i} =
 \omega^{2i}$. Množina je preto "symetrická". Zároveň
 $\omega^2$ je  $2^{k-1}$-ta odmocnina jednotky a~preto
 môžeme dokázať supersymetrickosť rekurzívnym aplikovaním.
\end{dokaz}

\begin{poznamka}
    Môžeme si všimnúť, že pre pole komplexných čísel $\C$ je $n$-tá
    odmocnina z jednotky napríklad $\omega=e^{-2\pi/n}$. V~tomto prípade
    %vzorec \todo{} 
    vyhodnocovanie $p(x)$ v bodoch $\omega^0, \omega^1, \dots,
    \omega^{n-1}$
    prejde presne na diskrétnu Fourierovu
    transformáciu.
\end{poznamka}

%\subsubsection{Rýchla interpolácia polynómov}
\subsection{Rýchla interpolácia polynómov}
Po úspešnom zvládnutí vyhodnocovania a~násobenia musíme ešte rýchlo
interpolovať. Použijeme podobný trik - budeme sa snažiť rozdeliť
problém na dva podproblémy, využijúc pritom symetriu našej množiny $X$.

Majme polynóm $p(x)=p_0 + p_1 x + \dots + p_{n-1} x^{n-1}$ stupňa
$n-1$ kde $n$ je párne. Navyše, majme symetrickú množinu $X$ v~ktorej
bodoch chceme interpolovať polynóm $p$. 
Nech $y_i$ je hodnota ktorú chceme dosiahnuť v~bode $x_i$. Potom
\begin{align*}
    y_i &= \sum_{j=0}^{n-1} p_j x_i^j \\
    y_{n/2+i} &= \sum_{j=0}^{n-1} p_j x_{n/2+i}^j 
            = \sum_{j=0}^{n-1} p_j (-x_i)^j 
\end{align*}
Skombinovaním rovníc dostávame
\begin{align*}
    y_i + y_{n/2 + i} &= 2 \sum_{\substack{j=0,\\
        j\text{ je párne}}}^{n-1} p_j x_i^j
    = 2 \sum_{j=0}^{n/2-1} p_{2j} x_i^{2j} \\
    y_i - y_{n/2 + i} &= 2 \sum_{\substack{j=0,\\ 
        j\text{ je nepárne}}}^{n-1} p_j x_i^j
     = 2 \sum_{j=0}^{n/2-1} p_{2j+1} x_i^{2j+1} =
      2 x_i \sum_{j=0}^{n/2-1} p_{2j+1} x_i^{2j}
\end{align*}
Posledná sada rovníc nám hovorí, že výpočet interpolačného polynómu
$P$ vieme rozdeliť na dva výpočty interpolačných polynómov polovičného
stupňa, ktoré následne vieme jednoducho skombinovať.
\begin{lema}
    Interpolácia polynómu $p$ na supersymetrickej množine $X$ sa dá
    vypočítať v~čase $O(n \log n)$.
\end{lema}
\begin{dokaz}
    Nech $T(n)$ je čas potrebný na výpočet interpolácie.
    Potom $T(n)=2 T(\frac{n}{2}) + O(n)$ - najskôr si vypočítame
    $\forall i \in 0,1\dots \frac{n}{2}-1$ hodnoty 
    \begin{align}
        \label{eq:interpolacia_g}
        g_i &= 2^{-1}(y_i + y_{n/2+i}) \\
        \label{eq:interpolacia_h}
        h_i &= 2^{-1}(y_i - y_{n/2+i}) x_i^{-1}
    \end{align}
    Následne
    vypočítame interpolačné polynómy $G,H$ na množine
    $X'=\{x_i^2 : i \in 0,1,\dots \frac{n}{2}-1\}$, čo sú dve interpolácie
    polovičnej veľkosti a~výsledné koeficienty iba vložíme na párne
    resp. nepárne pozície $P$.
    Daná rekurencia má podľa Master Theorem časovú zložitosť $O(n \log
    n)$.
\end{dokaz}

Ukázali sme teda, že vieme robiť aj rýchlu interpoláciu. Preto vieme
násobiť polynómy v čase $O((n+m) \log (n+m))$.

%\subsubsection{Súvislosť s DFT}
\subsection{Súvislosť s DFT}
Na záver si ešte vysvetlíme jednotlivé súvislosti s~diskrétnou
Fourierovou transformáciou. Ako sme už písali vyššie, súvis
s~vyhodnocovaním polynómov je silno spätý s~rovnicou DFT.
Konkrétne, ak si vezmeme množinu 
$\{ \omega^i: i \in 0,1,\dots,n-1\}$ kde $\omega = e^{-\imag 2 \pi
\frac{1}{n}}$, 
a označíme $P_k = p(\omega^k) = \sum_{i=0}^{n-1} p_i (\omega^k)^i$
tak vypočítanie všetkých hodnôt $P$ prejde priamo na diskrétnu Fourierovu
transformáciu. Môžeme teda hovoriť, že diskrétna Fourierova
transformácia je vyhodnocovaním polynómov v~špeciálnych bodoch na
jednotkovej kružnici. Opačne, vyhodnocovanie polynómov nám dáva
nadhľad - síce sme Fourierovu transformáciu odvodili nad poľom
reálnych čísel a~k~odvodeniu sme sa dostali cez reálne po častiach
spojité funkcie, vidíme, že táto myšlienka sa dá generalizovať.
Presnejšie povedané, nech $F$ je pole charakteristiky $\not=2$ a~nech
$\omega$ je $n$-tá odmocnina jednotky v~danom poli.
Potom môžeme zaviesť DFT dĺžky $n$ nad prvkami tohoto poľa ako

\begin{equation*}
    X_k = \sum_{i=0}^{n-1} x_i \omega^{ik}
\end{equation*}
Inverzná transformácia je definovaná ako
\begin{equation*}
    x_k = n^{-1} \sum_{i=0}^{n-1} X_i \omega^{-ik}
        = n^{-1} \sum_{i=0}^{n-1} \sum_{j=0}^{n-1} 
            x_j \omega^{i(j-k)}
        = n^{-1} n x_k
\end{equation*}
kde poslednú rovnosť odvádzame na základe identity
\begin{equation*}
    \sum_{i=0}^{n-1} \omega^{ik} = \left\{
        \begin{array}{l l}
            n& k=0 \mod n\\
            0& \text{inak}
        \end{array}
        \right.
\end{equation*}
\begin{poznamka}
    $n^{-1}$ v~predchádzajúcej rovnici úzko súvisí s $2^{-1}$ v rovniciach
    \eqref{eq:interpolacia_g} a \eqref{eq:interpolacia_h}. Ak si
    spomenieme, túto $2^-1$ zahrnieme pri výpočte presne $k$-krát,
    zakaždým keď sa zavoláme na polku problému. Teda, dostávame
    $(2^{-1})^k = (2^k)^{-1} = n^{-1}$. Zároveň aj vysvetľuje, prečo
    požadujeme pole charakteristiky rôznej od 2.
\end{poznamka}

\nocite{pollard}
\nocite{compalg}
\nocite{practical_fast_multiplication}
