\subsection{Signal processing}

Či chcete alebo nechcete, signal processing je veľmi dôležitou
súčasťou každodenného života. Okolo nás sa každú sekundu prenesú veľké
množstvá údajov rôzneho typu. A práve tu hrá životne dôležitú hereckú
rolu konverzia analógového a digitálneho signálu. Vysielanie
televízie, rádia, alebo rozprávanie sa modemov, to všetko sú analógové
signály, ktoré treba nejakým spôsobom preniesť do digitálneho sveta. V
rannej dobe boli technológie čisto analógové, ale v dnešnej dobe sa
stretávame s náročným digitálnym spracovaním. Či ide o záznam zvuku
mikrofónom alebo meranie napätia voltmetrom, všade sa vynára tá istá
otázka.

\subsubsection{Je možné digitalizovať analógový signál?}
Odpoveď je nie. Nie bez ďalších predpokladov. Signál môže byť
ľubovoľný a nech by sme akokoľvek rýchlo zaznamenávali, stále nemusíme
zaznamenávať dostatočne rýchlo. Taktiež nemôžeme zaznamenávať s
nekonečnou presnosťou. Toto sú limitujúce faktory, ktoré prekážajú
záznamu signálu. Zaoberajme sa preto otázkou, či za nejakých
zjednodušených predpokladov je možné rekonštruovať signál iba z
čiastočných údajov. Možno je to prekvapujúce, ale dá sa to. Základ
tejto teórie vypracovali páni Nyquist a Shannon. Sformulujme a dokážme
si preto jednu základnú vetu zo signal processingu.

\begin{veta}
    (Nyquist-Shannon sampling theorem) Nech $f(x)$ je ľubovoľná
    funkcia z $\LLinf$. Ak $\exists B$ také že $f$ je bandlimited
    frekvenciou $B$, tak postačujúcou podmienkou
\end{veta}
