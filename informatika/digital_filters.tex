% vim:spell spelllang=sk
\subsection{Signal processing - digitálne filtre}
%% {{{
Digitálne filtre sú dôležitou súčasťou spracovania digitálneho
signálu. Podobne ako analógové filtre, ich úlohou je zo vstupného
signálu vybrať pre nás podstatnú časť, prípadne ju zosilniť alebo inak
modifikovať signál. V tomto dokumente si spomenieme niekoľko 
najzákladnejších filtrov používaných v praxi a súvisiacich s Fourierovou
transformáciou. Budeme ich demonštrovať na spracovaní obrazu, hoci
mnohé z nich sú viac dôležité v spracovaní zvuku/signálu 
\footnote{V elektronickej prílohe je možné nájsť aj demonštráciu 
týchto filtrov pre zvuk, v publikovanej podobe z pochopiteľných
dôvodov nie je
}.
Najskôr si popíšeme filtre pracujúce vo frekvenčnom rozsahu, potom
spomenieme filtre pracujúce na \todo{spatial} dátach a kapitolu
zavŕšime ukázaním súvisu medzi týmito dvoma prístupmi a
dekonvolúciou, ktorá sa snaží invertovať následky nežiadúcich filtrov.
%% }}}

\subsubsection{Ideálny lowpass a highpass filter}
Ideálny lowpass a highpass filter o limitujúcej frekvencii $f$ môžeme
popísať obrázkom \ref{fig:ideal_lowpass_filter} \todo{FIXME: REF}, 
kde na $x$-ovej osi je frekvencia
vstupného signálu a na $y$-ovej osi je hodnota výstupného signálu.
\begin{poznamka}
    Väčšina filtrov vo frekvenčnej oblasti sa dá popísať podobným grafom.
    Čitaťeľ ale musí brať ohľad na istú nepresnosť - daný graf
    nešpecifikuje presne filter. Špecifikuje len zmenu amplitúdy.
    Bežné filtre fázu nemenia a preto sa ticho predpokladá $\phi(f)=0$.
\end{poznamka}

Dané filtre prepúšťajú všetky frekvencie na jednu stranu od hraničnej
a na druhú stranu neprepustia nič. Ako si ukážeme vizuálne, tieto
filtre majú spoločný problém - "znovenie". Najvýraznejšie sa prejavuje
práve pri highpass filtri. Preto sú v~praxi nepoužiteľné.
Zvonenie úzko súvisí s Gibbsovým fenoménom, ktorý sme už skúmali v
\todo{}. Ideálny lowpass filter je totiž varianta čiastočného súčtu
Fourierovho radu prenesená na Fourierovu transformáciu. Pri filtroch
nás teda okrem ich frekvenčnej odozvy bude zaujímať aj priestorová
odozva na jednoduchý impulz, v diskrétnom prípade nenulový bod, v
spojitom prípade hranatý impulz určenej dĺžky. Kombinovaním oboch
grafov sa môžeme presvedčiť o vlastnostiach filtra. Dobrý filter by
mal mať rýchly úpadok signálu keď sa príde na hraničnú frekvenciu,
zároveň by však nemal vykazovať prvky zvonenia.

\begin{figure}[htp]
    \def\path{obrazky/informatika/digitalne_filtre}
    \centering
    \subfigure[]{
        \includegraphics{\path/ideal_lowpass_frequency_10}
    }
    \subfigure[]{
        \includegraphics{\path/ideal_lowpass_response_10}
    }
    \subfigure[]{
        \includegraphics{\path/ideal_lowpass_10}
    }
    \subfigure[]{
        \includegraphics{\path/ideal_lowpass_20}
    }
    \subfigure[]{
        \includegraphics{\path/ideal_lowpass_40}
    }
    \caption{Ideálny lowpass filter}
    \label{fig:ideal_lowpass_image}
\end{figure}
Na obrázku \ref{fig:ideal_lowpass_filter} môžeme nájsť
frekvenčnú odozvu filtru, jeho priestorovú odozvu na bodový impulz a
ukážku filtrovania obrazu. Je dôležité si všimnúť najmä zvonenie v
priestorovej odozve a následný dopad na kvalitu obrazu. Obrázok má
rozlíšenie 256x256 a sú naň postupne aplikované filtre o hraničnej
frekvencii 10,20,40\footnote{Pri diskrétnej transformácii je
frekvencia 1 zodpovedná sínusovému signálu o perióde dĺžky obrázku}.

Ideálny highpass filter bude mať presne opačnú odozvu ako ideálny
lowpass filter. Viac všeobecne, môžeme hovoriť, že frekvenčná odozva
highpass filtra je
\begin{equation}
    f_H(\omega) = 1 - f_L(\omega)
    \label{eq:highpass_filter}
\end{equation}
Pretože súčet oboch filtrov dodáva pôvodný obraz, je ihneď zjavné že
aj ideálny highpass filter bude mať podobné zvoniace efekty ako jeho
brat. Jeho charatketistika sa dá nájsť na obrázku \todo{ref}.
Pri highpass filtroch ešte spomenieme jednu výnimku rovnice
\ref{eq:highpass_filter}. Ide konkrétne o frekvenciu 0, čiže priemer.
Ak počítame výpočty v štandardných hodnotách 0-255, efektívne
vynulovanie rovnicou \ref{eq:highpass_filter} by spôsobilo posun jasu
k čiernym farbám. V tomto prípade je vhodné nastaviť výslednú hodnotu
buď konštantne na polovicu jasu, alebo ponechať pôvodný priemer, čiže
mať jednotkovú odozvu.
\todo{obrazok}

\subsubsection{Hladké filtre}
Ako sme v predchádzajúcej sekcii ukázali, ideálne filtre majú zvoniaci
efekt. Preto sa v praxi používajú filtre, ktoré majú spojitý a hladký
prechod medzi frekvenciami, ktoré filtrujú a frekvenciami, ktoré
nefiltrujú. Ukážeme si 2 rôzne filtre - Butterworth a Gaussov.
Začneme Gaussovým (lowpass) filtrom, ktorý je definovaný rovnicou
\todo{rovnica}.
Gaussov filter je priamym opakom ideálneho filtra. Jeho pozoruhodnou
vlastnosťou je rovnaká podoba vo frekvenčnej a priestorovej doméne.
Preto ho môžeme považovať akosi za "ideálne hladký" filter.



Prechodom medzi ideálnym a Gaussovým filtrom môžeme vyplniť
Butterworthovým filtrom.
Butterworthow filter rádu $n$ s hraničnou frekvenciou $\omega_0$
možeme zapísať ako
\todo{rovnica}.
Menením rádu $n$ meníme hladkosť filtra a zároveň jeho zvonenie.
Filtre do rádu $n=2$ majú zvoniaci efekt zanedbateľný, so vzrastajúcim
$n$ ale efekt postupne začína byť čím viac citeľnejší.

\subsubsection{Súvis medzi frekvenčnými a priestorovými filtrami}
Klasické obrazové filtre v priestorovej doméne ako napríklad medián, 
ostrenie hrán a
ich komplikovanejšie verzie vieme popísať ako maticu $M$ rozmerov $2k+1 \cross
2k+1$, tak aby pre výstup platilo 
\begin{equation}
    A_{i,j} = \sum_{r=0}^{2k+1} \sum_{s=0}^{2k+1} a_{i+r+k+1, j+s+k+1} M_{r,s}
\end{equation}
Pre jednorozmerné signály samozrejme použijeme vektor namiesto matice.
Identita sa dá zapísať ako 
\begin{equation}
    I = \begin{pmatrix}
            0 & 0 & 0 \\
            0 & 1 & 0 \\
            0 & 0 & 0
        \end{pmatrix}
\end{equation}
a filter na detekciu hrán je napríklad
\begin{equation}
    I = \begin{pmatrix}
            -1/8 & -1/8 & -1/8 \\
            -1/8 & 1 & -1/8 \\
            -1/8 & -1/8 & -1/8
        \end{pmatrix}
\end{equation}
Filtrovanie v časovej doméne je pomerne jednoduché a existuje veľké
množstvo filtrov. Súvislosť s filtrovaním vo frekvenčnej doméne je
preto veľmi zaujímavým cieľom. Je výhodné vedieť prevádzať filtre
medzi týmito doménami. Užitočným príkladom môže byť napríklad
aplikácie niekoľkých filtrov v časovej doméne za sebou na jeden obrázok.
Daný výpočet môže byť pomalý, ak sa začnú aplikovať veľké filtre.
navyše, pri neustálych výpočtoch môže dochádzať k strate presnosti v
dôsledku zaokrúhľovania. Preto je výhodnejšie previesť filtre na ich
frekvenčné dvojčatá, aplikovať ich vo frekvenčnej doméne.
Už na prvý pohľad výpočet filtra v časovej doméne matne pripomína
konvolúciu. Túto konvolúciu si teraz explicitne zapíšeme.
\begin{lema}
    Nech $M'$ je matica filtra $M$ "posunutá" do stredu. a preklopená
    podľa oboch osí. Presnejšie, nech 
    \begin{equation}
        M'_{i,j} = \left\{
            \begin{array}{l l}
               M_{-i-k-1,-j-k-1}, \quad &  -3k-1 \le i,j \le -k-1 \\
               0, \quad \textit{v ostatných prípadoch}
            \end{array}
            \right.
    \end{equation}
    Potom aplikácia filtra $M$ je (necyklická) konvolúcia obrazu
    $a$ s filtrom $M'$.
    \label{lema:filter_konvolucia}
\end{lema}
\begin{dokaz}
    \begin{equation}
        (a * M')_{x,y} = 
    \end{equation}
\end{dokaz}
Lema \ref{lema:filter_konvolucia} nám umožňuje prevádzať jednoduché
filtre medzi časovou a frekvenčnou doménou. Podľa konvolučnej vety je
totiž konvolúcia ekvivalentná jednoduchému násobeniu vo frekvenčnej
doméne. Samozrejme, ak uvažujeme diskrétnu Fourierovu transformáciu,
musíme uvažovať cyklickú konvolúciu, čo v praxi znamená rozšíriť
obrázok a filter na takú veľkosť, aby súčet ich veľkostí bol menší ako
veľkosť Fourierovej transformácie. Prevod medzi filtrovaním v časovej
a frekvenčnej doméne teda opäť sprostredkuje Fourierova
transformácia.

\subsubsection{Gaussov filter verzus iterovaný priemer}

Filtrom "symetrický priemer" nazveme filter, ktorého vektor je $1/4,1/2,1/4$.
Ide v podstate o filter 
\begin{equation}
A_i = \frac{\frac{a_{i-1}+a_i}{2} + \frac{a_i + a_{i+1}}{2}}{2}
\end{equation}
Ide o najjednoduchší príklad vyhladzovacieho
filtra\footnote{Presnejšie povedané najjednoduchší symetrický prípad}.
Jeho aplikovaní vyhladíme obrázok a stlmíme šum, samozrejme za cenu
jemného rozmazania hrán. Ak výsledný obrázok stále nie je dostatočne
hladký, môžeme túto procedúru zopakovať.
Opakovanou aplikáciou tohoto filtra dostávane filter "iterovaný
priemer". Cieľom tejto kapitoly je ukázať, že iterovaný priemer sa
blíži ku Gaussovmu lowpass filtru, ak počet iterácii rastie do
nekonečna.
Na obrázku \ref{fig:iterovany_median} máme zobrazený interovaný filter
s rôznym počtom iterácii.
\begin{figure}[htp]
    \centering
    \includegraphics{obrazky/informatika/signal_processing/iterated_median}
    \caption{Iterovaný medián}
    \label{fig:iterated_median}
\end{figure}

Už z obrázka možno vidieť trend približovaniu sa ku Gaussovej krivke.
\begin{lema}
    Vektor iterovaného mediánu po $k$ iteráciach je
    \begin{equation}
        IM(k) = \frac{1}{4^k} 
            (\binom{2k}{0}, \binom{2k}{1}, \binom{2k}{2}, \dots,
                \binom{2k}{2k})
    \end{equation}
    \label{lema:iterated_median}
\end{lema}
Z pravdepodobnosti a centrálnej limitnej vety vieme, že postupnosť z
lemy \ref{lema:iterated_median} konverguje ku Gaussovej krivke.
Iterovaný medián s dostatočným počtom iterácii je preto Gaussov filter
v priestorovej doméne. Fourierovou transformáciou Gaussovej krivka je
ale podľa \todo{} opäť Gaussova krivka.
To znamená, že iterovaný medián je postupným približovaním sa ku
Gaussovmu filtru ukazujúc zaujímavú súvislosť medzi vyhladzovaním v
priestorovej a časovej doméne.
