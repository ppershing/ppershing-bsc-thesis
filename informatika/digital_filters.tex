% vim:spell spelllang=sk
\subsection{Signal processing - digitálne filtre}
%% {{{
Digitálne filtre sú dôležitou súčasťou spracovania digitálneho
signálu. Podobne ako analógové filtre, ich úlohou je zo vstupného
signálu vybrať pre nás podstatnú časť, prípadne ju zosilniť alebo inak
modifikovať signál. V tomto dokumente si spomenieme niekoľko 
najzákladnejších filtrov používaných v praxi a súvisiacich s Fourierovou
transformáciou. Budeme ich demonštrovať na spracovaní obrazu, hoci
mnohé z nich sú viac dôležité v spracovaní zvuku/signálu 
\footnote{V elektronickej prílohe je možné nájsť aj demonštráciu 
týchto filtrov pre zvuk, v publikovanej podobe z pochopiteľných
dôvodov nie je
}.
Najskôr si popíšeme filtre pracujúce vo frekvenčnom rozsahu, potom
spomenieme filtre pracujúce na \todo{spatial} dátach a kapitolu
zavŕšime ukázaním súvisu medzi týmito dvoma prístupmi a
dekonvolúciou, ktorá sa snaží invertovať následky nežiadúcich filtrov.
%% }}}

\subsubsection{Ideálny lowpass a highpass filter}
Ideálny lowpass a highpass filter o limitujúcej frekvencii $f$ môžeme
popísať obrázkom \ref{fig:ideal_lowpass}, kde na $x$-ovej osi je frekvencia
vstupného signálu a na $y$-ovej osi je hodnota výstupného signálu.
\begin{poznamka}
    Väčšina filtrov vo frekvenčnej oblasti sa dá popísať podobným grafom.
    Čitaťeľ ale musí brať ohľad na istú nepresnosť - daný graf
    nešpecifikuje presne filter. Špecifikuje len zmenu amplitúdy.
    Bežné filtre fázu nemenia a preto sa ticho predpokladá $\phi(f)=0$.
\end{poznamka}

\begin{figure}[htp]
    \centering
    \includegraphics{obrazky/informatika/signal_processing/ideal_lowpass}
    \includegraphics{obrazky/informatika/signal_processing/ideal_highpass}
    \includegraphics{obrazky/informatika/signal_processing/ideal_lowpass_3d}
    \includegraphics{obrazky/informatika/signal_processing/ideal_highpass_3d}
    \caption{Frekvenčná priepustnosť ideálneho lowpass a highpass filtru}
    \label{fig:ideal_lowpass}
\end{figure}

Dané filtre prepúšťajú všetky frekvencie na jednu stranu od hraničnej
a na druhú stranu neprepustia nič. Ako si ukážeme vizuálne, tieto
filtre majú spoločný problém - "znovenie". Najvýraznejšie sa prejavuje
práve pri highpass filtri. Preto sú v praxi nepoužiteľné. Zvonenie
pochádza práve z ich ideality - obrazom jedného obrazového bodu, resp.
kruhu je sada sústredných kružníc a keď vymažeme všetky kružnice od
nejakej frekvencie, pri spätnej transformácii budú chýbať. Toto
zvonenie úzko súvisí s Gibbsovým fenoménom, ktorý sme už skúmali v
\todo{}. V praxi sa preto viacej používajú hladké filtre.

\begin{figure}[htp]
    \centering
%    \includegraphics{obrazky/informatika/signal_processing/ideal_lowpass}
    \caption{Ideálny lowpass filter}
    \label{fig:ideal_lowpass_image}
\end{figure}


\begin{figure}[htp]
    \centering
%    \includegraphics{obrazky/informatika/signal_processing/ideal_lowpass}
    \caption{Ideálny highpass filter}
    \label{fig:ideal_highpass_image}
\end{figure}


\subsubsection{Hladké filtre}
Ako sme v predchádzajúcej sekcii ukázali, ideálne filtre majú zvoniaci
efekt. Preto sa v praxi používajú filtre, ktoré majú spojitý a hladký
prechod medzi frekvenciami, ktoré filtrujú a frekvenciami, ktoré
nefiltrujú. Ukážeme si 2 rôzne filtre - Butterworth a Gaussov.



\subsubsection{Súvis medzi frekvenčnými a \todo{spatial} filtrami}
Klasické \todo{spatial}

\subsubsection{Gaussian ako iterovany median}
