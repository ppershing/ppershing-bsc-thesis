\section{Gibbsov fenomén}

V príkladoch \todo{} sme si ukázali Fourierove rady pre niektoré
nespojité funkcie. Táto kapitola bude venovaná ich spoločnej
vlastnosti - fenoménu \todo{overshooting}. Zopakujme si ešte raz
grafy týchto dvoch príkladov.
\begin{priklad}
    Pílová funkcia
    \begin{equation}
        f(x) = \left\{
            \begin{array}{l l}
                0 \quad x \in (-\pi,0) \\
                1 \quad x \in (0,\pi)
            \end{array}
        \right.
    \end{equation}
    Jej $n$-tý čiastočný Fourierov súčet je
    \begin{equation}
        S_n(x) = \frac{1}{2} + \frac{2}{\pi} \sum_{m=1}^{n}
                \frac{\sin\left[ (2m-1) x\right]}{2m-1} \quad n>0
    \end{equation}

    A ukážka grafu pre $n=10,100$
    \begin{figure}[htp]
        \centering
        \includegraphics{obrazky/transformacia/gibbs/gibbs_saw10}
        \includegraphics{obrazky/transformacia/gibbs/gibbs_saw100}
        \caption{}
        \label{fig:gibbs_saw}
    \end{figure}    
\end{priklad}
Môžeme si všimnúť zaujímavú vlastnosť, a síce že čiastočné súčty
nemajú charakter konvergovať rovnomerne k pôvodnej funkcii. Na druhú
stranu podľa \todo{} platí $S_n \imply f$ bodovo. V praktickom slova
zmysle nám to hovorí, že čiastočné súčty síce konvergujú k danej
funkcii, ale ich graf vykazuje isté nezrovnalosti. Jeho nepríjemnou
vlastnosťou je tendencia prestreliť hodnotu funkcie, a to
nezanedbateľnou hodnotou, ako sa môžeme presvedčiť na príklade 2.

\begin{priklad}
    Uvažujme nasledujúcu funkciu:
    \begin{equation}
        f(x) = \frac{1}{2} (\pi-x), \quad x\in(0,2\pi)
    \end{equation}
    Jej čiastočný Fourierov súčet je
    \begin{equation}
        S_n(x) = \sum_{k=1}^{n} \frac{\sin kx}{k}
    \end{equation}
    ktorý sa dá odvodiť podobne ako riešenie príkladu \todo{}.
    Sľúbený graf pre hodnoty $n=10,100,1000$
    \begin{figure}[htp]
        \centering
        \includegraphics{obrazky/transformacia/gibbs/gibbs_lin10}
        \includegraphics{obrazky/transformacia/gibbs/gibbs_lin100}
        \includegraphics{obrazky/transformacia/gibbs/gibbs_lin1000}
        \includegraphics{obrazky/transformacia/gibbs/gibbs_lin1000_2}
        \caption{}
        \label{fig:gibbs_lin}
    \end{figure}    

\end{priklad}

Zvyšok tejto kapitoly venujeme práve tejto funkcii a jej čiastočným
súčnom a odhadovaniu o koľko čiastočné súčty $S_n$ prestrelia hodnotu
$f(x)$. Príjemným výsledkom tohoto nášho snaženia bude záver pre
všetky funkcie $\in PCab$.

\begin{lema}
    Funkcia $S_n(x)$ má na intervale $(0,\pi)$ extremálne body
    $\frac{1}{n+1}\pi$ a
    $\frac{2k}{n}\pi, \frac{2k+1}{n+1}\pi$ pre
    $k \in \{0,1,2,\dots,\floor{\frac{1}{2}(n-1)} \}$
\end{lema}
\begin{dokaz}
    \begin{eqnarray}
        S_n'(x) =& \sum_{k=1}^n \cos kx = 
            \frac{1}{2\sin(\frac{1}{2}x)} \sum_{k=1}^n 2
            \sin(\frac{1}{2}x \cos(kx) =\\
            &=\frac{1}{2\sin(\frac{1}{2}x)} \sum_{k=1}^n
                \sin(\frac{1}{2}x + kx) +
                \sin(\frac{1}{2}x - kx) = \\
            &=\frac{1}{2\sin(\frac{1}{2}x)} \sum_{k=1}^n
                \sin(kx + \frac{1}{2}x ) -
                \sin(kx - \frac{1}{2}x ) = \\
            &=\frac{1}{2\sin(\frac{1}{2}x)} 
                \left(
                    \sin((n+\frac{1}{2}x) - \sin(\frac{1}{2}x
                \right) = \\
            &=\frac{ \sin(\frac{1}{2}nx) \cos(\frac{1}{2}(n+1)x)}
                {\sin(\frac{1}{2} x)}
    \end{eqnarray}
    Na danom intervale je $\sin(\frac{1}{2} x)$ kladné a preto
    extremálne body $S_n(x)$ sú nulové
    body $S_n'(x)$ a to sú nulové body funkcií
     $\sin(\frac{1}{2}nx)$ a $\cos(\frac{1}{2}(n+1)x)$ čiže
     $\frac{2 k \pi}{n},\quad k\in \{1,\dots,\floor{\frac{n-1}{2}}\}$ a
     $\frac{(2k+1)\pi}{n+1},\quad k\in
     \{0,q,\dots,..\floor{\frac{n-1}{2}}\}$. Zároveň však
     vieme, že pre body kde $\sin(\frac{1}{2} x)=0$ je
     $\cos(\frac{1}{2}(n+1)x)\not=0$ a naopak.
     Preto $S_n'(x)$ v týchto bodoch strieda znamienko a teda to nie
     sú inflexné body $S_n$.
\end{dokaz}

\begin{lema}
    Funkcia $S_n(x)$ má na intervale $(0,\pi)$ maximá
    $\frac{2k+1}{n+1}\pi k \in \{0,1,2,\dots,\floor{\frac{1}{2}(n-1)} \}$
    a minimá
    $\frac{2k}{n}\pi, $ pre
    $k \in \{1,2,\dots,\floor{\frac{1}{2}(n-1)} \}$
\end{lema}
\begin{dokaz}
    Platí
    \begin{equation}
        \frac{2k}{n}\pi < \frac{2k+1}{n+1}\pi < \frac{2(k+1)}{n} \pi,
        \quad k \in \{1,2,\dots,\floor{\frac{n-1}{2}} \}
    \end{equation}
    a tiež
    \begin{equation}
        \frac{2k+1}{n+1}\pi < \frac{2k+2}{n}\pi < \frac{2k+3}{n+1} \pi,
        \quad k \in\{1,2,\dots,\floor{\frac{n-1}{2}}-1 \}
    \end{equation}
    Teda, dané 2 postupnosti bodov sa striedajú. Využitím výsledku
    predchádzajúcej lemy a uvážením, že extremálne body sa striedajú
    máme výsledok na dosah. Stačí dokázať, že bod $\frac{1}{n+1}\pi$
    je maximom $S_n(x)$. To je ale zrejmé, lebo
     $S_n'(x)$ je kladná na intervale $(0, \frac{1}{n+1}\pi)$.
\end{dokaz}

\begin{veta}
Pre každé $s\in\N$, postupnosť
$\left\{S_n\left( \frac{2s-1}{n+1}\pi\right)\right\}_{n=1}^{\infty}$
má limitu $\int_0^{(2s-1)\pi} \sinc t \dd t$. Podobne, postupnosť
$\left\{S_n\left( \frac{2s}{n}\pi\right)\right\}_{n=1}^{\infty}$
má limitu $\int_0^{2s\pi} \sinc t \dd t$.
\end{veta}

\begin{dokaz}
    \begin{align}
        S_n\left(\frac{2s-1}{n+1}\pi\right) =
            \sum_{k=1}^n \frac{1}{k}
            \sin\left(\frac{k(2s-1)}{n+1}\pi\right) = \\
         \frac{2s-1}{n+1}\pi    
            \sum_{k=1}^n \frac{n+1}{k(2s-1)\pi}
            \sin\left(\frac{k(2s-1)}{n+1}\pi\right) = \\
         \frac{2s-1}{n+1}\pi    
            \sum_{k=1}^n \sinc\left( \frac{k(2s-1)}{n+1} \pi \right)            
        \label{eq:gibbs_convergence}
    \end{align}
    Na druhú stranu, zoberme si \todo{Riemannov} horný a dolný
    integrálny súčet funkcie $\sinc t$ na intervale $(0,L)$.
    Uvažujme rovnomerné rozdelenie intervalu na $n$ rovnakých častí.
    Potom $H_n \ge \frac{L}{n} \sum_{k=0}^{n-1} \sinc(\frac{k L}{n}) \ge D_n$
    ako sa môžeme ľahko presvedčiť.
    Porovnaním s \ref{eq:gibbs_convergence} a uvážením faktu
    $\sinc 0 =1$ dostávame
    \begin{equation}
        H_{n+1} \ge S_n\left(\frac{2s-1}{n+1}\pi\right) +
        \frac{2s-1}{n+1}\pi \ge D_{n+1}
    \end{equation}
    Následne
    \begin{equation}
       \int_0^(2s-1) \sinc t \dd t = 
       \limtoinf{n} H_{n+1} - \frac{2s-1}{n+1}\pi \ge 
       \limtoinf{n} S_n\left(\frac{2s-1}{n+1}\pi\right) \ge
       \limtoinf{n} D_{n+1} - \frac{2s-1}{n+1}\pi =
       \int_0^(2s-1) \sinc t \dd t                     
        \ge D_{n+1}
    \end{equation}
    Analogicky ukážeme aj druhú rovnosť    
\end{dokaz}

Označme $G(x) = \frac{2}{\pi} \int_0^{x \pi} \sinc t \dd t$ 
Keďže pre ľubovoľné $s\in N$, $\limtoinf{n} \frac{1}{2} (\pi -
\frac{2s-1}{n+1} )=\frac{\pi}{2}$, môžeme hovoriť, že
$n$-tý čiastočný súčet prestrelí hodnotu funkcie
v svojom $k$-tom maxime $G(2k-1)$ krát ak
$n$ pošleme do nekonečna.
Funkciu $G(k)$ môžete vidieť na obrázku \ref{fig:gibbs_sinint}.
\todo{figure}.


Tabelované hodnoty pre prvých niekoľko miním a maxím môžeme nájsť v
tabuľke \ref{tab:gibbs_table}. 

\begin{table}[htb]
    \centering
    \begin{tabular}{c|c|c|c|c|c|}
        k&1&2&3&4&5 \\ \hline
        Minimá &0.90282&0.94994&0.96641&0.97475&0.97978 \\
        Maximá &1.17898&1.06619&1.04021&1.02883&1.02246
    \end{tabular}
    \caption{Tabulované hodnoty miním a maxím}
    \label{tab:gibbs_table}
\end{table}

Z hodnôt je zrejmé, že čiastočné súčty prestrelia funkciu takmer o
18\%, čo je prekvapivo veľa. Ak by sme počítali túto hodnotu nie
vzhľadom na aktuálnu hodnotu $f(x)$ ale vzhľadom na veľkosť skoku,
táto hodnota bude polovičná. 
\todo{rovnomerna konvergencia na intervale, ak neobsahu bod
nespojitosti}

\begin{lema}
    Nech $g\in \PCab$. Potom v blízkosti každého bodu nespojitosti
    čiastočné súčty $S_n$ Fourierovho radu prestrelia funkčné hodnoty
    o skoro 9\%
\end{lema}
\begin{dokaz}
    Dôkaz bude neformálny. V prvom rade, podľa
    \todo{ref: shifting theorem} vieme napísať rad funkcie
    $f(x-t)$ kde $f$ sme uvažovali ako \todo{periodic extension}.
    Taktiež sa môžeme ľahko presvedčiť, že
    $\forall n: S_{f(x),n}(y) = S_{f(x-t),n}(y-t)$ a teda posunutím
    funkcie sa správanie radu okolo konkrétneho bodu nezmení.
    Zvyšok dôkazu založíme na tom, že každá funkcia $g\in\PCab$
    sa dá lineárne zložiť z konečného počtu posunutých funkcii $f(x-t_i)$
    a spojitej funkcie $h\in C$. Náhľad prečo to je tak - 
    Nech sú $t_i, i\in\{1,2,\dots,m\}$ body nespojitosti funkcie $g$
    a nech $u_i = \lim_{x\imply t_i^+} g(x) - \lim_{x\imply t_i^-} g(x)$.
    Potom $G(x) = \sum_i^m u_i f(x-t_i)$ je funkcia ktorá je spojitá
    na všetkých intervaloch kde je spojitá $g(x)$ a taktiež má skoky
    rovnakej veľkosti. Potom $G(x)-g(x)$ je spojitá na všetkých tých
    intervaloch kde je spojitá $g$ a taktiež v každom bode
    nespojitosti $t_i$ funkcie $g$ platí 
    $\lim_{x\imply t_i^+} (G(x)-g(x)) - \lim_{x\imply t_i^-}
    (G(x)-g(x)) = 0$. Potom je ale funkcia $H(x)=G(x)-g(x)$ spojitá.
    Máme teda $g=G-H$ kde $G$ má presne podmienky z lemy. Podľa
    \todo{uniform convergence} ale $H$ konverguje rovnomerne na celom
    intervale a teda nepokazí vlastnosť $G$.
\end{dokaz}

\todo{lit:hewitt}
    
