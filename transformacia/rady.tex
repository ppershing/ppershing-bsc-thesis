\section{Fourierove rady}

Teória okolo Fourierových radov sa zaoberá základnou otázkou "Ako
aproximovať funkciu pomocou základných periodických funkcií?".
Za základné funkcie si zvolíme $\sin(x), \cos(x)$.

\begin{definicia} Nech $f(x)$ je ľubovoľná funkcia definovaná na
intervale $(-L,L)$. Napíšme
    \begin{equation}
        f(x) \sim \frac{1}{2} a_0 + \sum_{n=0}^{\infty} a_n
        \cos\frac{n \pi x}{L} + b_n \sin\frac{n \pi x}{L}
    \end{equation}
a daný rad nazvime Fourierovým radom.
\end{definicia}
Rad v tejto podobe, kde konštanty $a_n, b_n$ ešte treba bližšie
špecifikovať sa nazýva trigonometrický nekonečný rad. Môže a nemusí
konvergovať, a pre tie hodnoty $x$ kde konverguje, môže a nemusí
konvergovať k $f(x)$. Celou úlohou tejto kapitoly bude špecifikovať
aké sú hodnoty $a_n, b_n$ a kedy rad konverguje.

Začneme základnými závislosťami. Integrovaním môžeme overiť, že pre
ľubovoľné $m,n \in \Z$ platí
\begin{eqnarray}
    &\displaystyle \int_{-L}^{L} \cos \frac{m \pi x}{L} 
     \cos\frac{n \pi x}{L}\, \dd x &= 
     \left\{
        \begin{array}{l l}
            0,& \quad m\not=n \\
            L,& \quad m=n>0 \\
            2L,& \quad m=n=0
        \end{array}
     \right. \nonumber\\
    &\displaystyle \int_{-L}^{L} \sin \frac{m \pi x}{L} 
     \sin\frac{n \pi x}{L}\, \dd x &= 
     \left\{
        \begin{array}{l l}
            0,& \quad m\not=n \\
            L,& \quad m=n > 0
        \end{array}    
     \right. \nonumber\\
    &\displaystyle \int_{-L}^{L} \sin \frac{m \pi x}{L}
     \sin \frac{n \pi x}{L}\, \dd x &= 0,
        \quad {\rm pre\, všetky\,} m,n
        \label{eq:basic_orthogonality}
\end{eqnarray}

---------------------------------------

\begin{definicia}

Množinu všetkých absolútne Lébesgueovsky integrovateľných funkcií z $R(a,b)$
resp. $C(a,b)$
pre ktoré platí
\begin{equation}
\int_a^b |f(x)|^2 dx < \infty
\end{equation}
budeme označovať $\LLab$.

\end{definicia}

\begin{lema}
$\LLab$ je vektorový priestor.
\end{lema}

\begin{dokaz}
\begin{itemize}
\item
Najdôležitejším bodom dôkazu je časť $f,g \in \LLab \imply (f+g)\in \LLab$.
Podľa AG nerovnosti platí 
\begin{equation*}
    xy \le \frac{1}{2}(x^2 + y^2).
\end{equation*}
Substitovaním $x=|f(x)|,y=|g(y)|$ dostávame
$|f(x)\overline{g(x)}| \le \frac{1}{2} 
( |f(x)|^2 + |g(x)|^2) $.
 Potom
 \begin{eqnarray*}
    & \int_a^b |f(x)+g(x)|^2 dx  \le
    \int_a^b (|f(x)|+|g(x)|)^2 dx \le \\    
    & \le  \int_a^b |f(x)|^2 + |g(x)|^2 + |f(x)||(g(x)| dx = \\
    & = \int_a^b |f(x)|^2 + |g(x)|^2 + |f(x) \overline{g(x)}| dx \le \\
    & \le \frac{3}{2}( \int_a^b |f(x)^2 dx + \int_a^b |g(x)|^2 dx)
    < \infty
 \end{eqnarray*} 
\item
$f \in LLab, c\in C \imply (c.f) \in \LLab$.
Dôkaz: $\int_a^b |cf(x)|^2 dx = \int_a^b |c|^2 |f(x)|^2 dx =
 |c|^2 \int_a^b |f(x)|^2 dx < \infty$.
\end{itemize}
\end{dokaz}

\begin{definicia}
 Skalárny súčin v $\LLab$ definujeme ako
  $(f,g) = \int_a^b f(x) g(x) dx$ resp. 
  $\ll f,g \gg = \int_a^b f(x) \overline{g(x)} dx$ v prípade
  komplexných čísel.
\end{definicia}

