\section{Fourierove rady}

%%% {{{ Fourierove rady, sin cos
Teória okolo Fourierových radov sa zaoberá základnou otázkou "Ako
aproximovať funkciu pomocou základných funkcií?".
Za základné funkcie si môžeme zvoliť ľubovoľnú sadu funkcii a potom
študovať či je daná aproximácia možná. Jedným zo zaujímavých
odrazových mostíkov sú harmonické funkcie ako $\sin(n x), \cos(nx), n
\in \N$.
Na obrázkoch \ref{fig:harmonic_illustration} je znázornený ich priebeh a skladanie.

\begin{figure}[htp]
    \centering
    \includegraphics{obrazky/phase_difference}
    \includegraphics{obrazky/frequency_difference}
    \includegraphics{obrazky/addition}
    \caption{Fázový posun, rôzne frekvencie a skladanie harmonických
    funkcií}\label{fig:harmonic_illustration}
\end{figure}

Ako sa presvedčíme v ďalšom texte, pomocou harmonických funkcií a ich
skladania budeme vedieť vyjadriť veľkú triedu iných funkcií a celá
táto práca sa venuje ich využitiu.

\begin{definicia} Nech $f(x)$ je ľubovoľná funkcia definovaná na
intervale $(-\pi,\pi)$. Napíšme
    \begin{equation}
        f(x) \sim \frac{1}{2} a_0 + \sum_{n=1}^{\infty} a_n
        \cos n x + b_n \sin n x
    \label{eq:trig_series}
    \end{equation}
a daný rad nazvime Fourierovým radom.
\end{definicia}

Rad v tejto podobe, kde konštanty $a_n, b_n$ ešte treba bližšie
špecifikovať sa nazýva trigonometrický nekonečný rad. Môže a nemusí
konvergovať, a pre tie hodnoty $x$ kde konverguje, môže a nemusí
konvergovať k $f(x)$. Celou úlohou tejto kapitoly bude špecifikovať
aké sú hodnoty $a_n, b_n$ a kedy rad konverguje.
%%% }}}

%%% {{{ Otrogonalita sin, cos a vzorec pre a_n,b_n
Začneme základnými závislosťami. Integrovaním môžeme overiť, že pre
ľubovoľné $m,n \in \Z$ platí
\begin{eqnarray}
    &\displaystyle \intLL \cos \frac{m \pi x}{L} 
     \cos\frac{n \pi x}{L}\, \dd x &= 
     \left\{
        \begin{array}{l l}
            0,& \quad m\not=n \\
            L,& \quad m=n>0 \\
            2L,& \quad m=n=0
        \end{array}
     \right. \nonumber\\
    &\displaystyle \intLL \sin \frac{m \pi x}{L} 
     \sin\frac{n \pi x}{L}\, \dd x &= 
     \left\{
        \begin{array}{l l}
            0,& \quad m\not=n \\
            L,& \quad m=n > 0
        \end{array}    
     \right. \nonumber\\
    &\displaystyle \intLL \sin \frac{m \pi x}{L}
     \sin \frac{n \pi x}{L}\, \dd x &= 0,
        \quad {\rm pre\, všetky\,} m,n
        \label{eq:basic_orthogonality}
\end{eqnarray}
Taktiež platí pre $n \in N, n>0$
\begin{align}
    & \intLL \cos \frac{n \pi x}{L} \dd x = {}& 0 \nonumber \\
    & \intLL \sin \frac{n \pi x}{L} \dd x = {}& 0 
    \label{eq:harmonic_integrals}
\end{align}


Pokračujme teda v našom probléme určiť koeficienty $a_n,b_n$ ak
poznáme funkciu $f(x)$.
Predpokladajme, že funkcia $f(x)$ sa dá zapísať podľa
\ref{eq:trig_series}. Navyše predpokladajme, že daný rad môže byť
intergovaťeľný \todo{term by term}, teda že integrál zo sumy je suma
integrálov (a teda tiež predpokladajme, že $f(x)$ je integrovateľná).
Potom integrovaním od $-\pi$ po $\pi$ dostaneme


\begin{equation}
    \intpipi f(x) \dd x = \frac{a_0}{2} \intpipi \dd x +
        \sum_{k=1}^{\infty} \left( 
            a_k \intpipi \cos kx  \dd x +
            b_k \intpipi \sin kx  \dd x
        \right)
\end{equation}
Využitím \ref{eq:harmonic_integrals} zmiznú všetky integrály v sume a
výsledok je
\begin{equation}
   a_0 = \frac{1}{\pi} \intpipi f(x) \dd x 
\end{equation}

Ako ďalší krok zoberme \ref{eq:trig_series}, vynásobme obe strany
$\cos(nx)$ a zintegrujme tak ako minule.

\begin{equation}
    \intpipi f(x) \cos nx  \dd x = \frac{a_0}{2} \intpipi \cos nx) \dd x +
        \sum_{k=1}^{\infty} \left(
            a_k \intpipi \cos kx \cos nx \dd x +
            b_k \intpipi \sin kx \cos nx \dd x
        \right)
\end{equation}

Aplikovaním \ref{eq:harmonic_integrals} na prvý integrál a 
\ref{eq:basic_orthogonality} na integrály v sume, všetky integrály
až na jeden budú nulové a ostane len integrál za koeficientom $a_n$.

\begin{equation}
    \intpipi f(x) \cos nx \dd x = a_n \intpipi \cos^2 nx \dd x = a_n \pi
\end{equation}

Podobne môžeme odvodiť

\begin{equation}
    \intpipi f(x) \sin nx \dd x = b_n \intpipi \sin^2 nx \dd x = b_n \pi
\end{equation}

Zhrnutím dosiahnutých výsledkov dokopy, môžeme tvrdiť

\begin{align}
    a_n &=  \frac{1}{\pi} \intpipi f(x) \cos n x \dd x \quad n\in \Nz
     \nonumber \\
    b_n &=  \frac{1}{\pi} \intpipi f(x) \sin n x \dd x \quad n\in \N
    \label{eq:furierove_koeficienty}
\end{align}
Čiže, konečne, ak $f(x)$ je intogravateľná a dá sa \todo{expand} na
trigometrický rad, a ak trigonometrický rad rad získaný násobením
$\cos nx$ a $\sin nx$ $n\in N$ sa dá intergovať \todo{term by term},
potom koeficienty $a_n, b_n$ sú dané podľa vzorca
\ref{eq:fourierove_koeficienty}.

\begin{definicia}
    Koeficienty vypočítané podľa vzorca
    \ref{eq:fourierove_koeficienty} nazveme Fourierove koeficienty
    funkcie $f(x)$.
\end{definicia}

%%% }}}

%%% {{{ Priklad: pulz
\begin{priklad}
Majme funkciu
    \begin{equation}
        f(x) = \left\{
            \begin{array}{l l}
                0 \quad x \in (-\pi,0) \\
                1 \quad x \in (0,\pi)
            \end{array}
        \right.
    \end{equation}
    
    \begin{poznamka}
        Všimnime si, že naša funkcia nie je definovaná v bode 0.
        Avšak pretože Fourierove koeficienty sú počítané integrovaním,
        zmena funkcie na konečnom počte bodov nemení hodnotu výsledku.
    \end{poznamka}

    Pre $n \in \Nz$
    \begin{equation}
        a_n = \frac{1}{\pi} \intpipi f(x) \cos nx \dd x =
        \frac{1}{\pi} \int_0^{\pi} \cos nx \dd x
    \end{equation}
    \begin{align}
        a_0 &= 1 \\
        a_n &= \frac{1}{\pi} \left[ \frac{1}{n} \sin nx
        \right]_0^{\pi} = 0 \quad n>0        
    \end{align}

    Podobne, pre $n\in \N$ máme
    \begin{align}
        b_n &= \frac{1}{\pi} \int_0^{\pi} \sin nx \dd x =
           \frac{1}{\pi} \left[ \frac{-1}{n} \cos nx \right]_0^{\pi} =
           \frac{1 - \cos n\pi}{n\pi} \\
        b_n &= \left\{
                \begin{array}{l l}
                    0 \quad & n {\rm\ je\ párne} \\
                    \frac{2}{n\pi}  \quad &n {\rm\ je\ nepárne}
                \end{array}
            \right.
    \end{align}

    Výsledný Fourierov rad pre $f(x)$ je 
    \begin{align}
        f(x) &\sim \frac{1}{2} + \frac{2}{\pi} \sin x + \frac{2}{3\pi}
            \sin 3x + \frac{2}{5\pi} \sin 5x + \cdots \\
            &\sim \frac{1}{2} + \frac{2}{\pi} \sum_{m=1}^{\infty}
                \frac{\sin\left[ (2m-1) x\right]}{2m-1}
    \end{align}

    Označme čiastočný súčet $S_n$ tohoto radu ako 
    \begin{align}
        S_0(x) = \frac{1}{2}
        S_n(x) = \frac{1}{2} + \frac{2}{\pi} \sum_{m=1}^{n}
                \frac{\sin\left[ (2m-1) x\right]}{2m-1} \quad n>0
    \end{align}
    
    Graf funkcie $f(x)$ a prvých pár čiastočných súčtov $S_n$ sa dá
    nájsť na obrázku \ref{fig:example_pulse}

    \begin{figure}[htp]
        \centering
        \includegraphics{obrazky/example_pulse}
        \caption{}
        \label{fig:example_pulse}
    \end{figure}    
\end{priklad}
%%% }}}

%%% {{{ priklad: x
\begin{priklad}
   Uvažujme funkciu 
   \begin{equation}
        f(x) = x \quad x \in(-\pi,\pi)
   \end{equation}

    Pretože funkcia $x \cos nx$ je nepárna, môžeme tvrdiť
   \begin{equation}
        a_n = \frac{1}{\pi} \intpipi x \cos nx \dd x =0
   \end{equation}

   \begin{align}
        b_n &= \frac{1}{\pi} \intpipi x \sin nx \dd x \\
        &= \frac{1}{\pi} \left[
            \frac{- x \cos nx}{n}
            \right]_{-\pi}^\pi -
            \frac{1}{\pi} \intpipi \frac{-1}{n} \cos nx \dd x  \\
        &=
        \frac{-2 \cos nx}{n} + 
            \left[
                \frac{\sin nx}{n^2}
            \right]_{-\pi}^{\pi} \\
        &= -\frac{2 \cos nx}{n}
   \end{align}
   A teda Fourierov rad je
   
   \begin{align}
   f(x) &\sim 2 \sin x - \sin 2x + \frac{2}{3} \sin 3x - \frac{2}{4}
    \sin 4x + \cdots \\
     &\sim 2\sum_{n=1}^{\infty} \frac{(-1)^{n+1}}{n} \sin nx
    \end{align}
    
    Graf funkcie $f(x)$ a prvých pár čiastočných súčtov $S_n$ sa dá
    nájsť na obrázku \ref{fig:example_lin}

    \begin{figure}[htp]
        \centering
        \includegraphics{obrazky/example_lin}
        \caption{}
        \label{fig:example_lin}
    \end{figure}    

\end{priklad}
%%% }}}

%%% {{{ Priklad: |x|
\begin{priklad}
    Ako tretí príklad si uvedieme funkciu
    \begin{equation}
        f(x) = |x| \quad x\in(-\pi,\pi)
    \end{equation}
    Vieme, že $|x| \sin nx$ je nepárna funkcia a preto
    \begin{equation}
        b_n = \frac{1}{\pi} \intpipi |x| \sin nx \dd x = 0
    \end{equation}
    Pre výpočet členov $a_n$ si rozdelíme integrál na 2 intervaly:
    \begin{align}
        a_0 &= \frac{1}{\pi} \intpipi |x| \dd x = \pi \\
        a_n &= \frac{1}{\pi} \intpipi |x| \cos nx \dd x \\
            &= \frac{1}{\pi} \int_0^{\pi} x \cos nx \dd x +
                \frac{1}{\pi} \int_{-\pi}^0 -x \cos nx \dd x \\
            &= \frac{2}{\pi} \int_0^{\pi} x \cos nx \dd x \\
            &= \frac{2}{\pi} \left[
                \frac{x \sin nx}{n}
                \right]_0^{\pi} -
                \int_0^{\pi} \frac{\sin nx}{n} \dd x \\
            &= \frac{2}{\pi} \left[
                    \frac{-\cos nx}{n^2} \right]_0^\pi \\
            &= \frac{2}{\pi} \frac{(1-\cos n\pi)}{n^2}
    \end{align}
    A výsledný Fourierov rad je
    \begin{align}
        f(x) &\sim \frac{\pi}{2} - \frac{4}{\pi} \cos x -
            \frac{4}{9\pi} \cos 3x - \cdots \\
            &\sim \frac{\pi}{2} - \frac{4}{\pi}
                \sum_{m=1}^{\infty} \frac{\cos(2m-1) x}{(2m-1)^2}            
    \end{align}
    
    Graf funkcie $f(x)$ a prvých pár čiastočných súčtov $S_n$ sa dá
    nájsť na obrázku \ref{fig:example_saw}

    \begin{figure}[htp]
        \centering
        \includegraphics{obrazky/example_saw}
        \caption{}
        \label{fig:example_saw}
    \end{figure}    
 V tomto príklade vidíme, že čiastočné súčty sa približujú k funkcii
 rýchlejšie ako v predchádzajúcich prípadoch a rovnomerne.
 Formálnejšie si tento jav popíšeme neskôr, keď budeme mať vybudovaný
 dostatočný aparát.
\end{priklad}
%%% }}}


---------------------------------------
%%% {{{ PC[a,b]
\begin{definicia}
    Triedu po častiach spojitých funkcií na intervale $[a,b]$ nazveme
    triedu funkcií $f(x)$ definovaných na intervale $[a,b]$ okrem
    konečného počtu bodov ak
    \begin{itemize}
        \item
            $\exists n$ a body $a=x_0<x_1<\cdots<x_n=b$ a funkcie
            $f_1, \cdots, f_n$
        \item
            $f_i$ je definovaná na $(x_{i-1},x_i)$
        \item
            pre $i \in 1,2,\dots,n$ je funkcia $f_i$ spojitá
        \item
            $\lim_{x \imply x_{i-1}^+} f_i$ a 
            $\lim_{x \imply x_i^-} f_i$ existujú a sú konečné
        \item
            $f(x) = f_i(x)$ pre $i: x_{i-1} < x < x_i$
    \end{itemize}
    Túto triedu budeme označovat $\PCab$
\end{definicia}

\begin{lema}
    Ak $f \in PCab$, potom $f$ je ohraničená na intervale $[a,b]$.
    \label{lema:ohranicenost_na_pcab}
\end{lema}

\begin{lema}
    $\PCab$ je vektorový priestor
\end{lema}
\begin{dokaz}
    \todo{neni zrejme dolezity, ale bolo by fajn zohnat zdroj}
\end{dokaz}
%%% }}}

%%% {{{ Dirichlet kernel
%%% Definicia dirichletovho kernelu
\begin{definicia}
Dirichletovým kernelom  rádu $n$ s periódou $2L$ nazvime periodickú funkciu
s periódou $2L$ definovanú na $(-L,L)$ vzorcom
\begin{equation}
    D_n(x) = \left\{
        \begin{array}{l l}
            \frac{\sin\left[(2n+1) \frac{\pi x}{2 L}\right]}{
            \sin\left(\frac{\pi x}{2 L}\right)} \quad& x\not=0 \\
            2n+1 \quad& x=0
        \end{array}
    \right.
\end{equation}
\end{definicia}

%%% Graf dirichletovho kernelu
Priebeh dirichletovho kernelu s periódou $2\pi$ je zobrazený na
obrázku
\ref{fig:dirichlet_kernel}

\begin{figure}[htp]
    \centering
    \includegraphics{obrazky/dirichlet_kernel}
    \caption{Dirichletov kernel}
    \label{fig:dirichlet_kernel}
\end{figure}

\begin{lema}
    Platí
    \begin{equation}
        D_n(x) = \sum_{m=-n}^{n} e^{2 \pi \imag m x/2L}
        \label{eq:dirichlet_kernel_alternative}
    \end{equation}
\end{lema}
\begin{dokaz}
    Ak $x=0$, dostávame
    \begin{equation}
        D_n(0) = \sum_{m=-n}^{n} e^0 = 2n+1
    \end{equation}
    Za predpokladu $x\not=0$ platí
    \begin{eqnarray}
        D_n(x) &=& \sum_{m=-n}^n e^{2 \pi \imag m x/2L} \\
               &=& e^{-2 \pi \imag n x/2L} \sum_{k=0}^{2n} 
                        e^{2 \pi \imag k x/2L} \\
               &=& e^{-2 \pi \imag n x/2L} 
                \frac{e^{2 \pi \imag (2n+1) x /2L}-1}{e^{2 \pi \imag
                x/2L}-1} \\
                &=& \frac{2 \imag e^{\pi\imag x/2L}}{
                          2 \imag e^{\pi\imag (2n+1) x/2L}} \cdot
                \frac{e^{2 \pi \imag (2n+1) x /2L}-1}{e^{2 \pi \imag
                x/T}-1} \\
                &=& \frac{2 \imag e^{\pi \imag x/2L}}{
                    e^{2 \pi \imag x/T}-1} \cdot
                    \frac{e^{2 \pi \imag (2n+1) x /2L}-1}{
                        2 \imag e^{\pi\imag (2n+1) x/2L}}\\
                &=& \frac{2 \imag}{
                    e^{\pi \imag x/2L}-e^{-\pi \imag x/2L}} \cdot
                    \frac{e^{\pi \imag (2n+1) x /2L}-
                        e^{-\pi \imag (2n+1) x /2L}
                    }{2 \imag}\\
                &=& \frac{\sin\left( (2n+1)\frac{\pi x}{2L} \right)}
                        {\sin \frac{\pi x}{2L}}
    \end{eqnarray}
\end{dokaz}

%%% Ciastocny sucet vyjadreny pomocou D_n
\begin{veta}
    N-tý čiastočný súčet Fourierovho radu funkcie $f \in \PCab$ je
    rovný
    \begin{equation}
        S_n(f,x) = \frac{1}{2L} \intLL f(y) D_n(x-y) \dd y
    \footnote{Môžeme si všimnúť, že danú integrál je konvolúcia.}
    \end{equation}
    \label{veta:dirichlet_expansion}
\end{veta}

\begin{dokaz}
    \begin{eqnarray}
        S_n(f,x) &=& \sum_{m=-n}^{n} \left(
            \frac{1}{2L} \intLL f(y) e^{-2\pi\imag n y/2L} \dd y
            \right) 
                e^{2\pi\imag n x/2L} \\
            &=& \frac{1}{2L} \intLL f(y) \left(
                    \sum_{m=-n}^{n} e^{2\pi\imag n (x-y)/T}
                \right) \dd y \\
    \end{eqnarray}
    Výsledok dostaneme aplikovaním
    \ref{eq:dirichlet_kernel_alternative} na danú sumu.
\end{dokaz}

Jednoduchým dôsledkom predchádzajúcej vety je
\begin{lema}
    \begin{equation}
        \intLL D_n(y) \dd y = 2L
    \end{equation}
\end{lema}
\begin{dokaz}
    Podľa predchádzajúcej vety platí 
    $S_n(f,x) = \frac{1}{2L} \intLL f(y) D_n(x-y) \dd y$.
    Špeciálne pre funkciu $f(x)=1$ platí $S_n(x)=1$ a teda
    \begin{equation}
        1=S_n(0) = \frac{1}{2L} \intLL 1 D_n(-y) \dd y =
            \frac{1}{2L} \intLL D_n(y) \dd y
    \end{equation}
    \label{lema:dirichlet_kernel_integration}
\end{dokaz}

Veta \ref{veta:dirichlet_expansion} nám hovorí, že n-tý čiastočný
súčet vieme vypočítať ako jednoduchý \todo{konvolučný} integrál.
Z obrázka \ref{fig:dirichlet_kernel} zase možno vidieť, že
Dirichletov kernel sa so zväčšujúcim $n$ koncentruje okolo stredu.
Naším ďalším cieľom bude ukázať, že ak $n\imply \infty$, tak
Dirichletov kernel "vybere" iba hodnotu $f(x)$. \todo{naozaj?}
Ako ďalej uvidíme, Dirichletov kernel je \todo{closely related to}
detla funkcii.
%%% }}}

%%% {{{ Riemann-Lebesgue lemma
\begin{lema}
    (Riemann-Lebesgueova lema):
    Nech $f: [a,b]\imply \R, f\in \PCab$. Potom
    \begin{equation}
        \lim_{\lambda\imply\infty} \int_a^b f(x) \sin \lambda x \dd x=
        \lim_{\lambda\imply\infty} \int_a^b f(x) \cos \lambda x \dd x=
        0
    \end{equation}
\end{lema}

\begin{dokaz}
    Lema vyzerá intuitívne, nakoľko ak zväčšujeme $\lambda$,
    perióda oscilácií sa zmenšuje a \todo{contribution}
    z kladných a záporných častí integrandu sa anulujú.

    \def\pil{\pi/\lambda}

    Nech $(c,d)$ je podinterval $[a,b]$ na ktorom je $f$ spojitá.
    Definujme 
    \begin{equation}
        I(\lambda) = \int_c^d f(x) \sin \lambda x \dd x
        \label{eq:riem_leb_I1}
    \end{equation}
    Substitúciou $x=y+\pil$ dostávame
    \begin{equation}
        I(\lambda) = \int_{c-\pil}^{d-\pil}
            f \left(y+\pil\right) \sin \lambda y \dd y
        \label{eq:riem_leb_I2}
    \end{equation}

    Sčítanie \ref{eq:riem_leb_I1} a \ref{eq:riem_leb_I2} dá
    \begin{eqnarray}
        2I(\lambda) =&-&\int_{c-\pil}^c f(x+\pil) \sin \lambda x \dd x 
                \\ &+&
                \int_{d-\pil}^d f(x) \sin \lambda x \dd x \\&+&
                \int_c^{d-\pil} \left( 
                    f(x) - f(x+\pil)
                    \right) \sin \lambda x \dd x
    \end{eqnarray}
    Označme maximum z $|f|$ na intervale $[c,d]$, ktoré vieme že
    existuje podľa lemy \ref{lema:ohranicenost_na_pcab}.
    Predpokladajme navyše že $\lambda$ je dostatočne veľká na to aby
    $\pil \le d-c$. Potom využijúc $|sin \lambda x|\le1$ máme
    \begin{equation}
        |I(\lambda)| \le K \pil + \frac{1}{2} \int_c^{d-\pil}
            \left|f(x) - f(x+\pil)\right|
        \label{eq:riem_leb_odhad}
    \end{equation}
    Pretože $f$ je spojitá na $(c,d)$ a oboje krajné limity sú konečné,
    je na $(c,d)$ aj rovnomerne spojitá. \todo{lema/odkaz}
    Potom $\forall \eps \exists \lambda_0: \forall \lambda>\lambda_0$
    platí
    \begin{equation}
        |f(x) - f(x+\pil)| < \frac{\eps}{d-c-\pil}
    \end{equation}
    No a pretože môžeme zvoliť $\lambda_0$ také, aby $K \pil < \eps/2
    \forall \lambda>\lambda_0$, podľa \ref{eq:riem_leb_odhad}
    $|I(\lambda)|<\eps$ a teda $I(\lambda)\imply0$ ak
    $\lambda\imply\infty$.
    
    Použitím rovnakého argumentu pre kosínus a aplikovaním výsledku 
    na všetky podintervaly $[a,b]$ na ktorých je $f$ spojitá môžeme
    zakončiť dôkaz.
\end{dokaz}
%%% }}}

%%% {{{ Fourierova veta
\begin{veta}
    \todo{(Fourier theorem)}:
    Ak $f, f' \in \PCab$ sú funkcie s periódou $2L$, potom 
    pravá strana \todo{} s $c_n$ podľa \todo{} komverguje
    bodovo k
    \begin{align}
        &\frac{1}{2}\left( \lim_{y\imply x^-} f(y) +
                \lim_{y\imply x^+} f(y)\right) \quad &\mbox{pre\ }
                x\in(-L,L) \\
        &\frac{1}{2}\left( \lim_{y\imply -L^+} f(y) +
                \lim_{y\imply L^-} f(y)\right) \quad &\mbox{pre\ }
                x=-L \mbox{\ alebo\ } L            
    \end{align}
    V prípade, že v bode $x$ je $f(x)$ spojitá, \todo{} konverguje
    bodovo k $f(x)$.
\end{veta}
\begin{dokaz}
    Nech $t \in (-L,L)$. Potom
    $\lim_{x\imply t^-} f(x) = f_-(t)$ a 
    $\lim_{x\imply t^+} f(x) = f_+(t)$.
    Podobne, pretože $f' \in \PCab$, platí
    \begin{equation}
        \lim_{h\imply 0} \frac{f_-(t) - f(t-h)}{h} = f'_-(t), \quad
        \lim_{h\imply 0} \frac{f(t+h)-f_+(t)}{h} = f'_+(t)
    \end{equation}
    Ak zvolíme $h$ dostatočne malé na to aby $f$ bola spojitá na
    $(t-h,t)$, podľa vety o strednej hodnote
    existuje $c \in (t-h,t)$ také že 
    $f_- - f(t-h) = f'(c) h$.
    Nakoľko podľa lemy \todo{najdi referenciu} je $f'$ ohraničená,
    existuje $M$ také že
    \begin{equation}
        |f_- - f(t-h)| \le \frac{1}{2} M h
    \end{equation}
    Použitím rovnakého argumentu pre $t+h$ môžeme dospieť k tvrdeniu
    \begin{equation}
        |f_- - f(t-h)| + |f(t+h) - f_+| \le Mh
        \label{eq:ft_ohranicenie}
    \end{equation}
    pre všetky $h>0$ také, že $f$ je spojitá na $(t-h,t)$ a $(t,t+h)$.

    Ak vo vete \ref{veta:dirichlet_expansion} zavedieme substitúciu
    $y=t+y'$ a využijeme fakt $D_n(t)=-D_n(t)$
    \begin{equation}
        S_n(f,t) = \frac{1}{2L} \intLL f(t+y') D_n(y') \dd y'
    \end{equation}
    Podobne dostaneme
    \begin{equation}
        S_n(f,t) = \frac{1}{2L} \intLL f(t-y') D_n(y') \dd y'
    \end{equation}
    A sčítaním máme
    \begin{equation}
        S_n(f,t) = \frac{1}{2L} \intLL \frac{1}{2} 
            \left( f(t+y') f(t-y') \right) D_n(y') \dd y'
    \end{equation}
    Čo môžeme pomocou lemy \ref{lema:dirichlet_kernel_integration}
    upraviť na tvar
    \begin{equation}
    S_n(f,t) - \frac{1}{2}(f_-+f_+)= \frac{1}{2L} \intLL 
        g(t,y') \sin\left( (2n+1)\frac{\pi y'}{2L} \right)\dd y'
        \label{eq:ft_sn}
    \end{equation}
    kde
    \begin{equation}
        g(t,y') = \frac{f(t+y')-f_+ + f(t-y')-f_-}{2 \sin(\pi y/ 2L)}
    \end{equation}
    Ak uvažujeme funkciu $g(t,y')$ ako funkciu premennej $y'$,
    $g(t,y')$ je po častiach spojitá a ohraničená, s možnou výnimkou
    v bode $y'=0$. Avšak, pre dostatočne malé $y'$ \ref{eq:ft_ohranicenie}
    ukazuje
    \begin{equation}
        |g(t,y')|\le \frac{M |y'|}{2|\sin(\pi y'/2L)|}
    \end{equation}
    Čo je ohraničené pre $y'\imply 0$.
    Funkcia $g$ preto splňuje podmienky Riemann-Lebesguovej vety
    a z \ref{eq:ft_sn} môžeme usudzovať
    $\lim_{n\imply\infty} S_n = \frac{1}{2}(f_- + f_+)$.
    Využitím periodickosti $f,f'$, ten istý dôkaz môžeme použiť pre
    $t=-L$ a $t=L$.
\end{dokaz}

%%% }}}

---------------------------------------
%%% {{{ L2
\begin{definicia}

Množinu všetkých absolútne Lébesgueovsky integrovateľných funkcií z $R(a,b)$
resp. $C(a,b)$
pre ktoré platí
\begin{equation}
\int_a^b |f(x)|^2 dx < \infty
\end{equation}
budeme označovať $\LLab$.

\end{definicia}

\begin{lema}
$\LLab$ je vektorový priestor.
\end{lema}

\begin{dokaz}
\begin{itemize}
\item
Najdôležitejším bodom dôkazu je časť $f,g \in \LLab \imply (f+g)\in \LLab$.
Podľa AG nerovnosti platí 
\begin{equation*}
    xy \le \frac{1}{2}(x^2 + y^2).
\end{equation*}
Substitovaním $x=|f(x)|,y=|g(y)|$ dostávame
$|f(x)\overline{g(x)}| \le \frac{1}{2} 
( |f(x)|^2 + |g(x)|^2) $.
 Potom
 \begin{eqnarray*}
    & \int_a^b |f(x)+g(x)|^2 dx  \le
    \int_a^b (|f(x)|+|g(x)|)^2 dx \le \\    
    & \le  \int_a^b |f(x)|^2 + |g(x)|^2 + |f(x)||(g(x)| dx = \\
    & = \int_a^b |f(x)|^2 + |g(x)|^2 + |f(x) \overline{g(x)}| dx \le \\
    & \le \frac{3}{2}( \int_a^b |f(x)^2 dx + \int_a^b |g(x)|^2 dx)
    < \infty
 \end{eqnarray*} 
\item
$f \in LLab, c\in C \imply (c.f) \in \LLab$.
Dôkaz: $\int_a^b |cf(x)|^2 dx = \int_a^b |c|^2 |f(x)|^2 dx =
 |c|^2 \int_a^b |f(x)|^2 dx < \infty$.
\end{itemize}
\end{dokaz}

\begin{definicia}
 Skalárny súčin v $\LLab$ definujeme ako
  $(f,g) = \int_a^b f(x) g(x) dx$ resp. 
  $\ll f,g \gg = \int_a^b f(x) \overline{g(x)} dx$ v prípade
  komplexných čísel.
\end{definicia}
%%% }}}
