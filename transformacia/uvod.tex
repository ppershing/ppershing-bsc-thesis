% vim:spell spelllang=sk
Cieľom nasledujúcej kapitoly bude oboznámiť čitateľa so základmi
Fourierovej analýzy. Na úvod odvodíme Fourierove rady, ich rôzne formy
zápisu. Neskôr sa budeme venovať matematickým podmienkam, za ktorých
sa dajú Fourierove rady použiť, ich konvergencii a~trochu sa
budeme venovať Gibbsovmu fenoménu vznikajúcemu pri transformácii
nespojitých funkcií. Ďalej vymenujeme ich vlastnosti s~dôkazmi.
Od Fourierových radov rýchlo prejdeme na samotnú transformáciu,
kde si len zhrnieme základné poznatky a~analógie s~radmi. Kapitolu
ukončíme strohým prehľadom transformácii súvisiacich s~FT a~porovnáme
ich vlastnosti.
