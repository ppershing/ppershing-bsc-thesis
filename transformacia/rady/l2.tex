
\subsection{Vektorový priestor $\LLab$}
%%% {{{ L2

%%% definicia L2
\begin{definicia}
    Množinu všetkých absolútne Lébesgueovsky integrovateľných 
    funkcií z $R(a,b)$ resp. $C(a,b)$ pre ktoré platí
    \begin{equation}
        \int_a^b |f(x)|^2 dx < \infty
    \end{equation}
    budeme označovať $\LLab$.
\end{definicia}

\begin{lema}
    $\LLab$ je vektorový priestor.
    \label{lema:LLab_je_vp}
\end{lema}

\begin{dokaz}
\begin{itemize}
\item
    Najdôležitejším bodom dôkazu je časť 
    $f,g \in \LLab \imply (f+g)\in \LLab$.
    Podľa AG nerovnosti platí 
    \begin{equation*}
        xy \le \frac{1}{2}(x^2 + y^2).
    \end{equation*}
    
    Substitovaním $x=|f(x)|,y=|g(x)|$ dostávame
    \begin{equation*}
        |f(x)\overline{g(x)}| \le \frac{1}{2} 
        ( |f(x)|^2 + |g(x)|^2)
    \end{equation*}
    
     Potom
    \begin{eqnarray*}
    & \int_a^b |f(x)+g(x)|^2 dx  \le
    \int_a^b (|f(x)|+|g(x)|)^2 dx \le \\    
    & \le  \int_a^b |f(x)|^2 + |g(x)|^2 + |f(x)||(g(x)| dx = \\
    & = \int_a^b |f(x)|^2 + |g(x)|^2 + |f(x) \overline{g(x)}| dx \le \\
    & \le \frac{3}{2}( \int_a^b |f(x)^2 dx + \int_a^b |g(x)|^2 dx)
    < \infty
    \end{eqnarray*}
\item
    $f \in \LLab, c\in C \imply (c.f) \in \LLab$.
    Dôkaz: $\int_a^b |cf(x)|^2 dx = \int_a^b |c|^2 |f(x)|^2 dx =
             |c|^2 \int_a^b |f(x)|^2 dx < \infty$.
\end{itemize}
\end{dokaz}

\begin{definicia}
    Skalárny súčin v $\LLab$ definujeme ako
    $(f,g) = \int_a^b f(x) g(x) \dd x$ resp. 
    $\innerc{f}{g} = \int_a^b f(x) \overline{g(x)} \dd x$ v prípade
    komplexných čísel.
    \label{def:llab_inner_product}
\end{definicia}
\begin{lema}
    Skalárny súčin ako je definovaný v definícii
    \ref{def:llab_inner_product}
    je naozaj skalárny súčin na $\LLab$
\end{lema}
\begin{dokaz}
Skalárny súčin musí spĺňať tieto podmienky:
\begin{itemize}
    \item $\inner{f+g}{h} = \inner{f}{h} + \inner{g}{h}$
    \item $\inner{cf}{g} = c \inner{f}{g}$
    \item $\inner{f}{g} = \inner{g}{f}$
    \item $\inner{f}{f} \ge 0$ a $\inner{f}{f}=0$ práve vtedy keď $f=0$.
\end{itemize}
Prvé tri podmienky vyplývajú zo základných vlastností Lébesguovho
integrálu. Zaujímavá je posledná podmienka.
$\int_a^b f(x) \overline{f(x)} \dd x = \int_a^b |f(x)|^2 \ge 0$.
Pretože pre komplexné čísla platí $|f(x)\overline{g(x)}|=|f(x)||g(x)|$.
Ostáva nám overiť, že $\int_a^b |f(x)|^2 = 0 \equiv f(x)=0$.
Žiaľ, toto nie je pravda, ak uvažujeme o funkciách v bodovom zmysle,
pretože zmena hodnoty na ľubovoľnej konečnej množine bodov nezmení
hodnotu integrálu. Namiesto toho budeme hovoriť, že
funkcia je identicky rovná 0 v $\LLab$, ak $S = \{x\in(a,b):
f(x)\not=0\}$ je množina miery 0.Potom daná rovnosť naozaj platí.
\end{dokaz}

\begin{definicia}
    Normou funkcie $f$ v $\LLab$ nazveme číslo
    $\norm{f} = \inner{f}{f}^{1/2}$.
    Všimnime si, že $\norm{f}\ge0$ pre všetky $f\in \LLab$.
\end{definicia}

\begin{definicia}
    Dve funkcie $f,g$ z $\LLab$ nazveme rovnaké, ak
    $\norm{f-g}=0$. Ako sme už videli, toto nutne neznamená, že sú
    identické bodovo. Funkcie v $\LLab$ si preto môžeme predstaviť ako
    triedy ekvivalencie bodových funkcii. Pri počítaní s funkciami v
    $\LLab$ budeme stále rátať s funkciami ako bodovými, avšak budeme
    rozumieť, že zmenením hodnôt funkcie na množine miery 0
    neovplyvníme výsledok.
\end{definicia}

\begin{poznamka}
    Neskôr uvidíme, že takáto "rovnosť" funkcií má za následok
    bijekciu medzi funkciou z $\LLab$ a jej Fourierovými
    koeficientami. Ako sme sa už predtým presvedčili, zmenenie hodnôt
    funkcie na konečnej množine nezmení Fourierove koeficienty.
    Z tohoto vyplýva, že korešpondencia medzi bodovými funkciami a ich
    fourierovými koeficientami je \todo{many to one}
\end{poznamka}

%%% }}}
