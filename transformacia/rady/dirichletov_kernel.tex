
\subsection{Dirichletov kernel}
%%% {{{ PC[a,b]
\begin{definicia}
    Triedu po častiach spojitých funkcií na intervale $[a,b]$ nazveme
    triedu funkcií $f(x)$ definovaných na intervale $[a,b]$ okrem
    konečného počtu bodov ak
    \begin{itemize}
        \item
            $\exists n$ a body $a=x_0<x_1<\cdots<x_n=b$ a funkcie
            $f_1, \cdots, f_n$
        \item
            $f_i$ je definovaná na $(x_{i-1},x_i)$
        \item
            pre $i \in 1,2,\dots,n$ je funkcia $f_i$ spojitá
        \item
            $\lim_{x \imply x_{i-1}^+} f_i$ a 
            $\lim_{x \imply x_i^-} f_i$ existujú a sú konečné
        \item
            $f(x) = f_i(x)$ pre $i: x_{i-1} < x < x_i$
    \end{itemize}
    Túto triedu budeme označovat $\PCab$
\end{definicia}

\begin{lema}
    Ak $f \in PCab$, potom $f$ je ohraničená na intervale $[a,b]$.
    \label{lema:ohranicenost_na_pcab}
\end{lema}

\begin{lema}
    $\PCab$ je vektorový priestor
\end{lema}
\begin{dokaz}
    \todo{neni zrejme dolezity, ale bolo by fajn zohnat zdroj}
\end{dokaz}
%%% }}}

%%% {{{ Dirichlet kernel
%%% Definicia dirichletovho kernelu
\begin{definicia}
Dirichletovým kernelom  rádu $n$ s periódou $2L$ nazvime periodickú funkciu
s periódou $2L$ definovanú na $(-L,L)$ vzorcom
\begin{equation}
    D_n(x) = \left\{
        \begin{array}{l l}
            \frac{\sin\left[(2n+1) \frac{\pi x}{2 L}\right]}{
            \sin\left(\frac{\pi x}{2 L}\right)} \quad& x\not=0 \\
            2n+1 \quad& x=0
        \end{array}
    \right.
\end{equation}
\end{definicia}

%%% Graf dirichletovho kernelu
Priebeh dirichletovho kernelu s periódou $2\pi$ je zobrazený na
obrázku
\ref{fig:dirichlet_kernel}

\begin{figure}[htp]
    \centering
    \includegraphics{obrazky/transformacia/rady/dirichlet_kernel}
    \caption{Dirichletov kernel}
    \label{fig:dirichlet_kernel}
\end{figure}

\begin{lema}
    Platí
    \begin{equation}
        D_n(x) = \sum_{m=-n}^{n} e^{2 \pi \imag m x/2L}
        \label{eq:dirichlet_kernel_alternative}
    \end{equation}
\end{lema}
\begin{dokaz}
    Ak $x=0$, dostávame
    \begin{equation}
        D_n(0) = \sum_{m=-n}^{n} e^0 = 2n+1
    \end{equation}
    Za predpokladu $x\not=0$ platí
    \begin{eqnarray}
        D_n(x) &=& \sum_{m=-n}^n e^{2 \pi \imag m x/2L} \\
               &=& e^{-2 \pi \imag n x/2L} \sum_{k=0}^{2n} 
                        e^{2 \pi \imag k x/2L} \\
               &=& e^{-2 \pi \imag n x/2L} 
                \frac{e^{2 \pi \imag (2n+1) x /2L}-1}{e^{2 \pi \imag
                x/2L}-1} \\
                &=& \frac{2 \imag e^{\pi\imag x/2L}}{
                          2 \imag e^{\pi\imag (2n+1) x/2L}} \cdot
                \frac{e^{2 \pi \imag (2n+1) x /2L}-1}{e^{2 \pi \imag
                x/T}-1} \\
                &=& \frac{2 \imag e^{\pi \imag x/2L}}{
                    e^{2 \pi \imag x/T}-1} \cdot
                    \frac{e^{2 \pi \imag (2n+1) x /2L}-1}{
                        2 \imag e^{\pi\imag (2n+1) x/2L}}\\
                &=& \frac{2 \imag}{
                    e^{\pi \imag x/2L}-e^{-\pi \imag x/2L}} \cdot
                    \frac{e^{\pi \imag (2n+1) x /2L}-
                        e^{-\pi \imag (2n+1) x /2L}
                    }{2 \imag}\\
                &=& \frac{\sin\left( (2n+1)\frac{\pi x}{2L} \right)}
                        {\sin \frac{\pi x}{2L}}
    \end{eqnarray}
\end{dokaz}

%%% Ciastocny sucet vyjadreny pomocou D_n
\begin{veta}
    N-tý čiastočný súčet Fourierovho radu funkcie $f \in \PCab$ je
    rovný
    \begin{equation}
        S_n(f,x) = \frac{1}{2L} \intLL f(y) D_n(x-y) \dd y
    \footnote{Môžeme si všimnúť, že danú integrál je konvolúcia.}
    \end{equation}
    \label{veta:dirichlet_expansion}
\end{veta}

\begin{dokaz}
    \begin{eqnarray}
        S_n(f,x) &=& \sum_{m=-n}^{n} \left(
            \frac{1}{2L} \intLL f(y) e^{-2\pi\imag n y/2L} \dd y
            \right) 
                e^{2\pi\imag n x/2L} \\
            &=& \frac{1}{2L} \intLL f(y) \left(
                    \sum_{m=-n}^{n} e^{2\pi\imag n (x-y)/T}
                \right) \dd y \\
    \end{eqnarray}
    Výsledok dostaneme aplikovaním
    \ref{eq:dirichlet_kernel_alternative} na danú sumu.
\end{dokaz}

Jednoduchým dôsledkom predchádzajúcej vety je
\begin{lema}
    \begin{equation}
        \intLL D_n(y) \dd y = 2L
    \end{equation}
\end{lema}
\begin{dokaz}
    Podľa predchádzajúcej vety platí 
    $S_n(f,x) = \frac{1}{2L} \intLL f(y) D_n(x-y) \dd y$.
    Špeciálne pre funkciu $f(x)=1$ platí $S_n(x)=1$ a teda
    \begin{equation}
        1=S_n(0) = \frac{1}{2L} \intLL 1 D_n(-y) \dd y =
            \frac{1}{2L} \intLL D_n(y) \dd y
    \end{equation}
    \label{lema:dirichlet_kernel_integration}
\end{dokaz}

Veta \ref{veta:dirichlet_expansion} nám hovorí, že n-tý čiastočný
súčet vieme vypočítať ako jednoduchý \todo{konvolučný} integrál.
Z obrázka \ref{fig:dirichlet_kernel} zase možno vidieť, že
Dirichletov kernel sa so zväčšujúcim $n$ koncentruje okolo stredu.
Naším ďalším cieľom bude ukázať, že ak $n\imply \infty$, tak
Dirichletov kernel "vybere" iba hodnotu $f(x)$. \todo{naozaj?}
Ako ďalej uvidíme, Dirichletov kernel je \todo{closely related to}
detla funkcii.
%%% }}}
