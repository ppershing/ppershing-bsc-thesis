
\subsection{Ortogonalita}
%%% {{{ Ortogonalita
\begin{definicia}
 Funkcie $f,g \in \LLab$ nazveme ortogonálne, ak
 $\inner{f}{g}=0$
\end{definicia}

%%% }}}

%%% {{{ Uzitocne lemy - pytagorova, cauchy-schwarz, triangle
\begin{lema}
    Pythagorova veta:
    Nech $f,g$ sú elementy vektorového priestoru a
    $f\perp g$.
    Potom $\norm{f}^2 + \norm{g}^2 = \norm{f+g}^2$.
\end{lema}
\begin{dokaz}
    \begin{align}
        \norm{f+g}^2 = \inner{f+g}{f+g} =  z linearity inner prod\\
        \inner{f,f} + \inner{f}{g} + \inner{g}{f} + \inner{g}{g} = \\
        \norm{f}^2 + \norm{g}^2 + \inner{f}{g} + \inner{g}{f} = 
        \norm{f}^2 + \norm{g}^2
    \end{align}
    Kde posledná rovnosť vyplýva z ortogonality $f,g$.
\end{dokaz}

%%% \cite{wiki:cauchy}
\begin{veta}
    Cauchy-Schwarzova nerovnosť:
    Nech $f,g$ sú prvky vektorového priestoru so skalárnym súčinom
    $\innerc{f}{g}$.
    Potom $|\innerc{f}{g}|^2 \le \innerc{f}{f} \innerc{g}{g}$.
\end{veta}
\begin{dokaz}
    Ak $g=0$, tak tvrdenie triviálne platí. Preto predpokladajme
    $g\not=0$. Nech $\alpha$ je komplexné číslo. Potom
    \begin{align}
        0\le \norm{f-\lambda g} = \innerc{f-\lambda g}{f-\lambda g} =
        \\
        \innerc{f}{f} - \overline{\lambda}\innerc{f}{g} -
        \lambda \innerc{g}{f} + |\lambda|^2 \innerc{g}{g}
    \end{align}
    Ak za $\lambda$ zvolíme $\innerc{f,g} \innerc{g,g}^-1$ (využijeme
    predpoklad $g\not=0$ a fakt $x\overline{x} = |x|^2$), dostaneme nerovnosť
    \begin{equation}
        0\le \innerc{f}{f} - |\innerc{f}{g}|^2 \innerc{g}{g}^-1
    \end{equation}
    ktorá platí vtedy a len vtedy ak
    \begin{equation}
        |\innerc{f,g}|^2 \le \innerc{f}{f} \innerc{g}{g}
    \end{equation}
\end{dokaz}

%%% \cite{wiki:triangle}
\begin{lema}
    Trojuholníková nerovnosť: Nech $f,g$ sú prvky vektorového
    priestoru. Potom $\norm{f}+\norm{g} \ge \norm{f+g}$.
\end{lema}
\begin{dokaz}
    Z dôkazu pytagorovej vety máme
    \begin{align}
        \norm{f+g}^2 = \inner{f}{f} + \inner{f}{g} + \inner{g}{f} +
        \inner{g}{g} =  \\
        \norm{f}^2 + 2Re\inner{f}{g} + \norm{g}^2 \le \\
        \norm{f}^2 + 2|\inner{f}{g}| + \norm{g}^2 \le 
        \text{podľa Cauchy-Schwarzovej nerovnosti}\\
        \norm{f}^2 + 2\norm{f}\norm{g} + \norm{g}^2 = \\
        (\norm{f} + \norm{g})^2
    \end{align}
    Dokazovanú nerovnosť dostaneme odmocnením 
\end{dokaz}

%%% \cite{bessel}
\begin{veta}
    \todo{Bessel's inequality}: Nech ${x_1,\dots,x_n}$ je ortonormálna
    množina a $x$ ľubovoľný prvok vektorového priestoru. Potom
    \begin{equation}
        \sum_{i=1}^n |\inner{x}{x_i}|^2 \le \norm{x}^2
    \end{equation}
\end{veta}
\begin{dokaz}
    Platí
    \begin{equation}
        x = \left( x- \sum_{i=1}^n \innerc{x}{x_i}x_i\right) +
            \left( \sum_{i=1}^n \innerc{x}{x_i}x_i \right)
    \end{equation}
    Podľa lemy \ref{lema:bessel_prereq} je pravá strana súčet
    dvoch kolmých vektorov. Aplikovaním trojuholníkovej vety
    na tieto 2 vektory dostávame
    \begin{equation}
        \norm{x}^2 =
         \norm{x- \sum_{i=1}^n \innerc{x}{x_i}x_i}^2 +
         \norm{\sum_{i=1}^n \innerc{x}{x_i}x_i}^2
    \end{equation}
    resp.
    \begin{equation}
        \norm{x}^2 \ge
          \innerc{\sum_{i=1}^n\innerc{x}{x_i}x_i}{
          \sum_{i=1}^n\innerc{x}{x_i}x_i}
    \end{equation}
    Na druhú stranu
    \begin{align}
          \innerc{\sum_{i=1}^n\innerc{x}{x_i}x_i}{
          \sum_{i=1}^n\innerc{x}{x_i}x_i} = \text{podobne ako
          predchvíľou} \\
          \sum_{i=1}^n \innerc{x}{x_i}\conjug{\innerc{x}{x_i}}
            \innerc{x_i}{x_i} = \\ 
          \sum_{i=1}^n \innerc{x}{x_i}\conjug{\innerc{x}{x_i}} =
          \sum_{i=1}^n |\innerc{x}{x_i}|^2
    \end{align}
\end{dokaz}

\begin{lema}
    \begin{equation}
        \left( x - \sum_{i=1}^n \innerc{x}{x_i}x_i\right)
        \perp \left( \sum_{i=1}^n \innerc{x}{x_i}x_i \right)
    \end{equation}
    \label{lema:bessel_prereq}
\end{lema}
\begin{dokaz}
    \begin{align}
        \innerc{ x - \sum_{i=1}^n \innerc{x}{x_i}x_i}{ 
                \sum_{i=1}^n \innerc{x}{x_i}x_i} = \\
        \innerc{x}{\sum_{i=1}^n \innerc{x}{x_i}x_i} -
        \innerc{\sum_{i=1}^n \innerc{x}{x_i}x_i}{ 
                \sum_{i=1}^n \innerc{x}{x_i}x_i} = \\
        \sum_{i=1}^n\innerc{x}{\innerc{x}{x_i}x_i} -
        \sum_{i=1}^n \sum_{j=1}^n
         \innerc{\innerc{x}{x_i}x_i}{\innerc{x}{x_j}x_j} = \\
         \sum_{i=1}^n\conjug{\innerc{x}{x_i}}\innerc{x}{x_i} -
        \sum_{i=1}^n \sum_{j=1}^n
         \innerc{x}{x_i}\conjug{\innerc{x}{x_j}}\innerc{x_i}{x_j} = \\         
        \sum_{i=1}^n\conjug{\innerc{x}{x_i}}\innerc{x}{x_i} -
        \sum_{i=1}^n
         \innerc{x}{x_i}\conjug{\innerc{x}{x_i}}\innerc{x_i}{x_i} = \\
        \sum_{i=1}^n\conjug{\innerc{x}{x_i}}\innerc{x}{x_i} -
        \sum_{i=1}^n
         \innerc{x}{x_i}\conjug{\innerc{x}{x_i}} = 0
    \end{align}
\end{dokaz}

%%% }}}
