% vim:spell spelllang=sk
\subsection{Exponenciálna forma Fourierovho radu}

Začneme základnou identitou, ktorá sa bude niesť celou sekciou a ktorá
nám umožní spraviť kompaktnejší zápis Fourierovho radu pomocou
komplexnej exponeciálnej funkcie.

\begin{veta}[Eulerova identita]
    \begin{equation}
        \cos \phi + \imag \sin \phi = e^{\imag \phi}
    \end{equation}
\end{veta}

Z danej vety sa dajú odvodiť nasledujúce zákonitosti
\begin{lema}
    \begin{align}
        \sin \phi &= \frac{e^{\imag \phi} - e^{-\imag \phi}}{2} \\
        \cos \phi &= \frac{e^{\imag \phi} + e^{-\imag \phi}}{2\imag} \\
    \end{align}
\end{lema}

Našou snahou bude ukázať, že Fourierov rad tak ako bol napísaný 
v definícii \ref{def:fourier_series} sa dá napísať aj v nasledujúcej
forme.

\begin{definicia}[Exponenciálna forma Fourierovho radu]
    Nech $f(x)$ je definovaná na $(-\pi,\pi)$. Potom rad
    \begin{equation}
        f(x) \sim \sum_{n=-\infty}^{\infty} c_n e^{\imag n x}
        \label{eq:exp_form}
    \end{equation}
    nazvime exponenciálnou formou Fourierovho radu
    \label{def:exp_form}
\end{definicia}
Všimnime si že
\begin{equation}
    c_n e^{\imag n x} + c_{-n} e^{\imag -n x} =
     (c_n +c_{-n}) \cos(nx) + (c_n - c_{-n}) \imag \sin(nx)
\end{equation}
Preto musí platiť
\begin{align}
    c_0 &= \frac{a_0}{2} \\
    c_n + c_{-n} &= a_n, \quad & n \in N \\
    c_n - c_{-n} &= \imag b_n, \quad & n \in N
    \label{eq:koeficienty_to_trie}
\end{align}
Dané rovnice možno ľahko prepísať do podoby
\begin{align}
    c_0 &= \frac{a_0}{2} \\
    c_n &= \frac{1}{2}(a_n + \imag b_n), \quad & n \in N \\
    c_{-n} &=\frac{1}{2}(a_n - \imag b_n), \quad & n \in N
    \label{eq:koeficienty_to_exp}
\end{align}
Posledným krokom k zavŕšeniu našeho poznania bude vyjadrenie $c_n$
pomocou vzorca \ref{eq:fourierove_koeficienty}.
\begin{align}
    c_0 &= \frac{a_0}{2} = \frac{1}{2\pi} \intpipi f(x) \dd x = 
            \frac{1}{2\pi} \intpipi f(x) e^{-0 \imag x} \dd x \\
    c_n &= \frac{1}{2}(a_n + \imag b_n) =
          \frac{1}{2\pi}\left(
            \intpipi f(x) \cos nx \dd x + 
            \imag \intpipi f(x) \sin nx \dd x  
            \right) = \\
         &=\frac{1}{2\pi} \intpipi f(x) (\cos nx + \imag \sin nx) \dd x =
          \frac{1}{2\pi} \intpipi f(x) e^{- \imag n x} \dd x \\
    c_{-n} &= \frac{1}{2}(a_n - \imag b_n) =
          \frac{1}{2\pi}\left(
            \intpipi f(x) \cos nx \dd x + 
            \imag \intpipi f(x) \sin nx \dd x  
            \right) = \\
         &=\frac{1}{2\pi} \intpipi f(x) (\cos -nx + \imag \sin -nx) \dd x =
          \frac{1}{2\pi} \intpipi f(x) e^{- \imag n x} \dd x        
\end{align}
Dospeli sme teda k finálnemu vzorcu
\begin{equation}
    c_n = \frac{1}{2\pi} \intpipi f(x) e^{-\imag n x} \dd x, 
        \quad n\in \Z
    \label{eq:fourierove_exp_koeficienty}
\end{equation}
Čo sa týka čiastočných súčtov, pre obe reprezentácie je $n$-tý
čiastočný súčet rovnaký (ak zadefinujeme $n$-tý čiastočný súčet
exponenciálnej formy ako $S_n(x) = \sum_{k=-n}^{n} c_k \exp{\imag k
x}$). Tento fakt je dôležitý, nakoľko nám umožňuje voľné zamieňanie
oboch reprezentácii, čo budeme v tejto publikácii aj následne robiť.

Ešte predtým ako ukončíme toto rozprávanie však treba poznamenať, že
táto cesta odvodenia nebola jediná možná a rovnako dobre sa dajú
odvodiť koeficienty podobnou cestou ako sme na začiatku odvodili
koeficienty $a_n,b_n$. Využitím nasledujúcej lemy a opakovaním postupu
dostaneme taktiež dostaneme koeficienty $c_n$.
\begin{lema}
    \begin{equation}
    \intpipi e^{i n x} \conjug{e^{i m x}} \dd x = 
    \intpipi e^{i n x} e^{-i m x} \dd x = 
    \left\{
        \begin{array}{l l}
            0,& \quad m\not=n \\
            2\pi,& \quad m=n            
        \end{array}
        \right.
    \end{equation}
\end{lema}
