% vim:spell spelllang=sk
\subsection{Periodické rozšírenia}

%%% {{{ definicia periody, suma per. funkcii
\begin{definicia}
  Funkciu $f(x)$ definovanú na $\R$ nazveme periodickou s periódou
  $P>0$
  ak
  \begin{equation}
    f(x) = f(x+P), \quad \forall x \in \R
  \end{equation}
\end{definicia}
\begin{poznamka}
    Ak je funkcia periodická s periódou $P$, potom je periodická
    aj s periódou $nP$ pre $\forall n \in N$. 
    Taktiež, každá konštantná funkcia je periodická s ľubovoľnou
    periódou $P>0$.
\end{poznamka}

\begin{lema}
    Nech $f_1(x), \dots, f_n(x)$ sú periodické funkcie so spoločnou
    periódou $P$. Potom ich suma
    \begin{equation}
        F(x) = \sum_{k=1}^n f_k(x)
    \end{equation}
    je taktiež periodická s periódou $P$.
    \label{lema:sum_of_periodic}
\end{lema}
Špeciálnym dôsledkom lemy \ref{lema:sum_of_periodic} je fakt, že
 $n$-tý čiastočný súčet Fourierovho radu ako sme ho definovali
 v definícii \ref{def:fourier_series} je periodický s periódou
 $2\pi$. Zjavne, táto perióda je definovaná prvým nekonštantným členom
 daného súčtu.
%%% }}} 

%%% {{{ periodicke rozsirenie, spojitost
\begin{definicia}[Periodické rozšírenie]
    Nech $f(x)$ je definovaná na intervale $[-L,L)$. Túto funkciu
    môžeme rozšíriť na celý interval nasledujúcim spôsobom:
    \begin{equation}
      \overline{f}(x) = f(x - 2L \floor{\frac{x-L}{2L}})
    \end{equation}
\end{definicia}

Nasledujúca lema nám vyjasní otázky okolo spojitosti takéhoto
rozšírenia.
\begin{lema}
    $\overline{f}(x)$ je spojitá na $(-\infty,\infty)$ vtedy a len
    vtedy ak
    \begin{itemize}
        \item   $f(x)$ je spojitá na $[-L,L)$
        \item   $\lim {x->L} f(x) = f(-L)$
    \end{itemize}
\end{lema}
%%% }}}

%%% {{{ parita funkcii
Nateraz sa vrátime k príkladom spomínaným v predchádzajúcej sekcii.
Všimnime si že \todo{ref pr 2} výsledný fourierov rad pozostáva iba zo
sínusov, zatiaľ čo \todo{ref pr 3} pozostáva iba z kosínusov. Toto
správanie je dôsledkom všeobecnejšieho tvrdenia, ktoré tu budeme
prezentovať.
\begin{definicia}[Párne a nepárne funkcie]
    Funkciu $f(x)$ nazveme párnu ak
    $\forall x: f(x) = f(-x)$. Podobne, funkciu nazveme nepárnu ak
    $\forall x: f(x) = - f(-x)$.
\end{definicia}
\begin{poznamka}
    Môžeme si všimnúť, že jedinou párnou a súčasne aj nepárnou
    funkciou je $f(x)=0$.
\end{poznamka}
Ľahko nahliadneme, že $\sin x$ je nepárna funkcia a $\cos x$ je párna
funkcia. Drtivá väčšina funkcií nie je ani párna ani nepárna, avšak
vieme ich zapísať ako súčet párnej a nepárnej funkcie ako vyplýva z
nasledujúcej vety
\begin{lema}
    Každú funkciu $f$ vieme zapísať ako súčet párnej funkcie $f_p$
    a nepárnej funkcie $f_n$ kde
    \begin{align}
        f_p(x) &= \frac{1}{2} (f(x) + f(-x)) \\
        f_n(x) &= \frac{1}{2} (f(x) - f(-x))
    \end{align}
\end{lema}
\begin{dokaz}
    Ľahko sa overí, že $f = f_p + f_n$. Taktiež, elementárnymi
    úpravami sa dá ukázať $f_p(x) = f_p(-x)$ a $f_n(x) = - f_n(-x)$
\end{dokaz}

\begin{veta}
    Nech $f$ je párna funkcia a $g$ je nepárna funkcia.
    \begin{equation}
        \int_{-L}^{L} f(x) g(x) \dd x = 0, \quad L \in \R_0^+ \union
        \infty
    \end{equation}
    \label{veta:int_parna_neparna}
\end{veta}
\begin{dokaz}
    \begin{equation}
    \begin{split}
      \int_{-L}^{L} f(x) g(x) \dd x &= 
                \int_{-L}^0 f(x) g(x) \dd x +
                \int_0^L f(x) g(x) \dd x \\
            &= \text{zavedením substitúcie $y=-x$ v prvom integráli}
            \\
            &= - \int_{0}^{L} f(-y) g(-y) \dd y + 
                \int_0^L f(x) g(x) \dd x \\
            &= \int_0^L f(x) g(x) - f(-x) g(-x) \dd x 
             = \int_0^L 0 \dd x= 0
    \end{split}
    \end{equation}
\end{dokaz}
%%% }}}

Špeciálne pre Fourierov rad máme
\begin{align}
    f_p(x) &= \frac{1}{2} a_0 + \sum_{n=1}^{\infty} a_n \cos n x \\
    f_n(x) &= \sum_{k=1}^{\infty} b_n \sin n x
\end{align}
Ľahko teda môžeme pozorovať, že pre párnu funkciu $f$ sú koeficienty
$b_n$ nulové a naopak, pre nepárnu funkciu sú koeficienty $a_n$
nulové. Môžeme však tvrdiť ešte viac
\begin{lema}
    Pre každú párnu funkciu $f$ platí
    \begin{align}
        a_n = 2 \int_0^{\pi} f(x) \cos n x \dd x, \quad & n \in \Nz
        b_n = 0, \quad& n\in N
    \end{align}
    Podobne pre každú nepárnu funkciu $f$ platí
    \begin{align}
        a_n = 0, \quad& n\in Nz
        b_n = 2 \int_0^{\pi} f(x) \sin n x \dd x, \quad & n \in \N
    \end{align}
    \label{lema:fourier_parity}
\end{lema}
\begin{dokaz}
    Tvrdenia jednoducho dostaneme využitím symetrie a rozdelením
    integrálu na 2 časti ako v dôkaze vety \ref{veta:int_parna_neparna}
\end{dokaz}
Predchádzajúca lema nám dáva nasledujúce pozorovanie: Predstavme si že
funkcia $f$ je definovaná len na polintervale $[0,\pi)$. Potom môžeme
položiť $b_n=0$ a dopočítať fourierove koeficienty $a_n$ podľa lemy
\ref{lema:fourier_parity}. Na druhú stranu, tiež môžeme položiť
$a_n=0$ a dopočítať $b_n$. Dané rozvoje nazvime \todo{half-range
Fourier sine/cosine series} a sú ekvivaletné párnemu/nepárnemu
rozšíreniu funkcie $f$. Vidíme teda, že osová a stredová symetria
spôsobujú absenciu niektorých členov Fourierovho rozvoja. Rôzne iné typy
symetrie určujú zaujímavé závislosti medzi členmi fourierovho rozvoja,
ale v tejto publikácii ich necháme nepreskúmané.

