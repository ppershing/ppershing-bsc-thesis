
\subsection{Bodová konvergencia}
%%% {{{ Riemann-Lebesgue lemma
\begin{lema}
    (Riemann-Lebesgueova lema):
    Nech $f: [a,b]\imply \R, f\in \PCab$. Potom
    \begin{equation}
        \lim_{\lambda\imply\infty} \int_a^b f(x) \sin \lambda x \dd x=
        \lim_{\lambda\imply\infty} \int_a^b f(x) \cos \lambda x \dd x=
        0
    \end{equation}
\end{lema}

\begin{dokaz}
    Lema vyzerá intuitívne, nakoľko ak zväčšujeme $\lambda$,
    perióda oscilácií sa zmenšuje a \todo{contribution}
    z kladných a záporných častí integrandu sa anulujú.

    \def\pil{\pi/\lambda}

    Nech $(c,d)$ je podinterval $[a,b]$ na ktorom je $f$ spojitá.
    Definujme 
    \begin{equation}
        I(\lambda) = \int_c^d f(x) \sin \lambda x \dd x
        \label{eq:riem_leb_I1}
    \end{equation}
    Substitúciou $x=y+\pil$ dostávame
    \begin{equation}
        I(\lambda) = \int_{c-\pil}^{d-\pil}
            f \left(y+\pil\right) \sin \lambda y \dd y
        \label{eq:riem_leb_I2}
    \end{equation}

    Sčítanie \ref{eq:riem_leb_I1} a \ref{eq:riem_leb_I2} dá
    \begin{eqnarray}
        2I(\lambda) =&-&\int_{c-\pil}^c f(x+\pil) \sin \lambda x \dd x 
                \\ &+&
                \int_{d-\pil}^d f(x) \sin \lambda x \dd x \\&+&
                \int_c^{d-\pil} \left( 
                    f(x) - f(x+\pil)
                    \right) \sin \lambda x \dd x
    \end{eqnarray}
    Označme maximum z $|f|$ na intervale $[c,d]$, ktoré vieme že
    existuje podľa lemy \ref{lema:ohranicenost_na_pcab}.
    Predpokladajme navyše že $\lambda$ je dostatočne veľká na to aby
    $\pil \le d-c$. Potom využijúc $|sin \lambda x|\le1$ máme
    \begin{equation}
        |I(\lambda)| \le K \pil + \frac{1}{2} \int_c^{d-\pil}
            \left|f(x) - f(x+\pil)\right|
        \label{eq:riem_leb_odhad}
    \end{equation}
    Pretože $f$ je spojitá na $(c,d)$ a oboje krajné limity sú konečné,
    je na $(c,d)$ aj rovnomerne spojitá. \todo{lema/odkaz}
    Potom $\forall \eps \exists \lambda_0: \forall \lambda>\lambda_0$
    platí
    \begin{equation}
        |f(x) - f(x+\pil)| < \frac{\eps}{d-c-\pil}
    \end{equation}
    No a pretože môžeme zvoliť $\lambda_0$ také, aby $K \pil < \eps/2
    \forall \lambda>\lambda_0$, podľa \ref{eq:riem_leb_odhad}
    $|I(\lambda)|<\eps$ a teda $I(\lambda)\imply0$ ak
    $\lambda\imply\infty$.
    
    Použitím rovnakého argumentu pre kosínus a aplikovaním výsledku 
    na všetky podintervaly $[a,b]$ na ktorých je $f$ spojitá môžeme
    zakončiť dôkaz.
\end{dokaz}
%%% }}}

%%% {{{ Fourierova veta
\begin{veta}
    \todo{(Fourier theorem)}:
    Ak $f, f' \in \PCab$ sú funkcie s periódou $2L$, potom 
    pravá strana \todo{} s $c_n$ podľa \todo{} komverguje
    bodovo k
    \begin{align}
        &\frac{1}{2}\left( \lim_{y\imply x^-} f(y) +
                \lim_{y\imply x^+} f(y)\right) \quad &\mbox{pre\ }
                x\in(-L,L) \\
        &\frac{1}{2}\left( \lim_{y\imply -L^+} f(y) +
                \lim_{y\imply L^-} f(y)\right) \quad &\mbox{pre\ }
                x=-L \mbox{\ alebo\ } L            
    \end{align}
    V prípade, že v bode $x$ je $f(x)$ spojitá, \todo{} konverguje
    bodovo k $f(x)$.
\end{veta}
\begin{dokaz}
    Nech $t \in (-L,L)$. Potom
    $\lim_{x\imply t^-} f(x) = f_-(t)$ a 
    $\lim_{x\imply t^+} f(x) = f_+(t)$.
    Podobne, pretože $f' \in \PCab$, platí
    \begin{equation}
        \lim_{h\imply 0} \frac{f_-(t) - f(t-h)}{h} = f'_-(t), \quad
        \lim_{h\imply 0} \frac{f(t+h)-f_+(t)}{h} = f'_+(t)
    \end{equation}
    Ak zvolíme $h$ dostatočne malé na to aby $f$ bola spojitá na
    $(t-h,t)$, podľa vety o strednej hodnote
    existuje $c \in (t-h,t)$ také že 
    $f_- - f(t-h) = f'(c) h$.
    Nakoľko podľa lemy \todo{najdi referenciu} je $f'$ ohraničená,
    existuje $M$ také že
    \begin{equation}
        |f_- - f(t-h)| \le \frac{1}{2} M h
    \end{equation}
    Použitím rovnakého argumentu pre $t+h$ môžeme dospieť k tvrdeniu
    \begin{equation}
        |f_- - f(t-h)| + |f(t+h) - f_+| \le Mh
        \label{eq:ft_ohranicenie}
    \end{equation}
    pre všetky $h>0$ také, že $f$ je spojitá na $(t-h,t)$ a $(t,t+h)$.

    Ak vo vete \ref{veta:dirichlet_expansion} zavedieme substitúciu
    $y=t+y'$ a využijeme fakt $D_n(t)=-D_n(t)$
    \begin{equation}
        S_n(f,t) = \frac{1}{2L} \intLL f(t+y') D_n(y') \dd y'
    \end{equation}
    Podobne dostaneme
    \begin{equation}
        S_n(f,t) = \frac{1}{2L} \intLL f(t-y') D_n(y') \dd y'
    \end{equation}
    A sčítaním máme
    \begin{equation}
        S_n(f,t) = \frac{1}{2L} \intLL \frac{1}{2} 
            \left( f(t+y') f(t-y') \right) D_n(y') \dd y'
    \end{equation}
    Čo môžeme pomocou lemy \ref{lema:dirichlet_kernel_integration}
    upraviť na tvar
    \begin{equation}
    S_n(f,t) - \frac{1}{2}(f_-+f_+)= \frac{1}{2L} \intLL 
        g(t,y') \sin\left( (2n+1)\frac{\pi y'}{2L} \right)\dd y'
        \label{eq:ft_sn}
    \end{equation}
    kde
    \begin{equation}
        g(t,y') = \frac{f(t+y')-f_+ + f(t-y')-f_-}{2 \sin(\pi y/ 2L)}
    \end{equation}
    Ak uvažujeme funkciu $g(t,y')$ ako funkciu premennej $y'$,
    $g(t,y')$ je po častiach spojitá a ohraničená, s možnou výnimkou
    v bode $y'=0$. Avšak, pre dostatočne malé $y'$ \ref{eq:ft_ohranicenie}
    ukazuje
    \begin{equation}
        |g(t,y')|\le \frac{M |y'|}{2|\sin(\pi y'/2L)|}
    \end{equation}
    Čo je ohraničené pre $y'\imply 0$.
    Funkcia $g$ preto splňuje podmienky Riemann-Lebesguovej vety
    a z \ref{eq:ft_sn} môžeme usudzovať
    $\lim_{n\imply\infty} S_n = \frac{1}{2}(f_- + f_+)$.
    Využitím periodickosti $f,f'$, ten istý dôkaz môžeme použiť pre
    $t=-L$ a $t=L$.
\end{dokaz}

%%% }}}
