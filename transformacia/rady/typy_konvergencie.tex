\subsection{Typy konvergencie}

Predtým než začneme vyšetrovať konvergenciu Fourierových radov, je
vhodné si zopakovať základné poznatky o konvergencii radov všeobecne
a tak si pripraviť pôdu na ďalšie rozpravy.

Majme formálny rad funkcií definovaných na intervale $I$
\begin{equation}
    \sum_{k=1}^{\infty} u_k(x)
    \label{eq:konv_rad}
\end{equation}
a označme $n$-tý čiastočný súčet ako
\begin{equation}
    S_n(x) = \sum_{k=1}^{n} u_k(x), \quad x\in I
    \label{eq:konv_sucet}
\end{equation}
Potom postupnosť $S_1,S_2, \dots$ tvorí postupnosť čiastočných súčtov
nekonečného radu \ref{eq:konv_rad}. Konvergencia nekonečného radu tak
môže byť popísaná pomocou konvergencie postupnosti čiastočných súčtov.
V skutočnosti, existuje viacero rôznych možností, čo si predstavujeme
pod konvergenciou ako vidno v nasledujúcej definícii:

\begin{definicia}
    Nekonečný rad \ref{eq:konv_rad} konverguje k sume $S(x)$ na
    intervale $I$
    \begin{itemize}
        \item {\emph\todo{mean square}} ak 
            \begin{equation}
                \limtoinf{n} \int_I |S_n(x) - S(x)|^2 \dd x = 0
            \end{equation}
        \item {\emph bodovo} ak 
            \begin{equation}
                \forall x\in I: \limtoinf{n} |S_n(x) - S(x)| = 0
            \end{equation}
        \item {\emph rovnomerne na intervale $I$} ak 
            \begin{equation}
                \limtoinf{n} \max_I |S_n(x) - S(x)| = 0
            \end{equation}        
    \end{itemize}
\end{definicia}
Každý z týchto typov konvergencie popisuje spôsob ako sa graf $S_n$
blíži ku grafu $S$.
Bodová konvergencia znamená, že každý bod $S_n$ sa blíži k $S$ ak
$n\imply\infty$, avšak nehovorí nič o relatívnom približovaní rôznych
bodov. Naopak, rovnomenrná konvergencia zaručuje, že graf $S_n$ sa
vtesná do tesného pásu šítky $\eps$ pre $\forall \eps>0$.
A konečne \todo{mean square} konvergencia hovorí, že plocha medzi
krivkami sa blíži k nule. Špeciálne, táto konvergencia nezaručuje
bodovú konvergenciu, pretože zmenou konečného počtu bodov nezmeníme
hodnotu integrálu.

\begin{lema}
    Každá postupnosť, ktorá je rovnomerne konvergentná je aj bodovo a
    \todo{mean-square} konvergentná
\end{lema}
\begin{dokaz}
    Dôkaz je triviálny a prenecháme ho čitaťeľovi.
\end{dokaz}

V skutočnosti, rovnomerná konvergencia je ultimátna zbraň a je veľmi
silná, avšak je ťažké ju dosiahnuť a väčšinou to vyžaduje špeciálne
predpoklady. Vo väčšine reálnych problémov nám bude postčovať
\todo{mean-square} konvergencia. To nám ale nezabráni zhrnúť si
niektoré poznatky o rovnomernej konvergencii.

\begin{lema}
    Nech rad \ref{eq:konv_rad} konverguje rovnomerne na intervale $I$
    k sume $S(x)$. Pokiaľ každá z funkcií $u_k$ je spojitá na $I$,
    potom $S$ je nutne spojitá na $I$.
\end{lema}
\todo{chcem dokaz?}

Dôsledok tejto lemy je zjavný - Fourierov rad nemôže konvergovať
rovnomerne k funkcii $f$ ak $f$ je nespojitá.

\begin{lema}[Weierstrass M-test]
  Postačujúca podmienka na konvergenciu radu \ref{eq:konv_rad}
  je existencia postupnosti $\left\{M_i\right\}_{i=1}^\infty$
  nezáporných konštánt $M_i>0$ takých, že
  $\sum M_i$ konverguje a $\forall i: \forall x\in I: |u_i(x)|\le M_i$
\end{lema}
\todo{dokaz - nejaka referencia}
