% vim:spell spelllang=sk
\section{Vlastnosti Fourierových radov}
Cieľom tejto sekcie je v stručnosti pozbierať a dokázať
najdôležitejšie vlastnosti Fourierových radov. V ďalšom texte budeme
uvažovať exponenciálnu reprezentáciu z
\eqref{eq:fourierove_exp_koeficienty} a budeme ju označovať ako
$\fourier f(x) = \{c_n\}_{n=-\infty}^{\infty}$.

\begin{veta}[Linearita]
Nech $f(x), g(x) \in \LLab$ a $\fourier{f(x)} = \{c_n\}$,
$\fourier{g(x)} = \{d_n\}$.
Potom $\forall \alpha,\beta \in \C$ platí
\begin{itemize}
    \item $\fourier{\alpha f(x)} = \{\alpha c_n\}$
    \item $\fourier{f(x) + g(x)} = \{c_n + d_n\}$
    \item $\fourier{\alpha f(x) + \beta g(x)} = \{\alpha c_n + \beta
    d_n\}$
\end{itemize}
\end{veta}
\begin{dokaz}
 Tvrdenie jednoducho vyplýva z linearity skalárneho súčinu
 (integrálu) na $\LLab$.
\end{dokaz}

\begin{veta}[Posúvanie v čase]
Nech $f(x) \in \LLab$, $\fourier{f(x)} = \{c_n\}$.
Nech $f'(x)$ je $2\pi$-periodické rozšírenie $f$. Potom
$\fourier{f'(x-x_0)} = \{ e^{-\imag n x_0} c_n \}$.
\label{veta:time_shift}
\end{veta}
\begin{dokaz}
    \begin{equation*}
    \begin{split}
      \{\fourier f'(x-x_0)\}_n &= 
        \frac{1}{2\pi} \intpipi f'(x-x_0) e^{-\imag x n} \dd x 
        = \frac{1}{2\pi} \int_{-\pi - x_0}^{\pi - x_0} f'(x) e^{-\imag x
            n} e^{-\imag x_0 n} \dd x \\
        &= \frac{1}{2\pi} e^{-\imag x_0 n} \intpipi f(x) e^{-\imag x n} \dd x
         = e^{-\imag x_0 n} c_n
    \end{split}
    \end{equation*}
\end{dokaz}

\begin{veta}[Časový reverz]
Nech $f(x) \in \LLab$, $\fourier{f(x)} = \{c_n\}$. Potom
$\fourier{f(-x)} = \{c_{-n} \}$.
\label{veta:time_reverse}
\end{veta}
\begin{dokaz}
    \begin{equation*}
      \{\fourier f(-x)\}_n
        = \frac{1}{2\pi} \intpipi f(-x) e^{-\imag x n} \dd x 
        = \frac{1}{2\pi} \intpipi f(y) e^{\imag y n} \dd y
        = c_{-n}
    \end{equation*}
\end{dokaz}

\begin{veta}[Škálovanie času]
Nech $f(x) \in \LLab$, $\fourier{f(x)} = \{c_n\}$. Potom
$\displaystyle f(a x) = \suminf{n} c_n e^{\imag a x n}$.
\end{veta}
\begin{dokaz}
Priama substitúcia do vzorca \eqref{eq:exp_form}
\end{dokaz}


\begin{veta}[Konjugácia]
Nech $f(x) \in \LLab$, $\fourier{f(x)} = \{c_n\}$. Potom
$\fourier f(x) = \{ \conjug{c}_{-n} \}$.
\end{veta}
\begin{dokaz}
    \begin{equation*}
    \begin{split}
      \{\fourier \conjug{f}(x)\}_n &= 
        \frac{1}{2\pi} \intpipi \conjug{f}(x) e^{-\imag x n} \dd x 
        = \conjug{  \frac{1}{2\pi} \intpipi f(x) \conjug{e^{-\imag x n}}
            \dd x} \\
        &= \conjug{  \frac{1}{2\pi} \intpipi f(x) e^{\imag x n}
            \dd x} 
        = \conjug{c_{-n}}
    \end{split}
    \end{equation*}
\end{dokaz}

Integráciu a derivovanie sme už rozoberali, preto iba zhrnieme
dosiahnuté výsledky.
\begin{veta}[O derivovaní]
Nech $f(x)$ je po častiach hladká funkcia, $\fourier{f(x)} =
\{c_n\}$. Potom $\fourier{f'(x)} = \{ \imag n c_n \}$.
\end{veta}
\begin{dokaz}
    Pozri vetu \ref{veta:fourier_derivacia}
\end{dokaz}

\begin{veta}[O integrovaní]
Nech $f(x)$ je po častiach spojitá funkcia, $\fourier{f(x)} =
\{c_n\}$. Potom $\fourier{F(x)-c_0 x} = \{ \frac{c_n}{\imag} : n\not=0\}$.
Špeciálne, $C_0 = \frac{1}{2} \intpipi F(x) \dd x$.
\end{veta}
\begin{dokaz}
    Pozri vetu \ref{veta:fourier_integrovanie}
\end{dokaz}

\begin{veta}[O modulácii]
Nech $f(x),g(x) \in \LLab$, $\fourier{f(x)} = \{c_n\}$,
$\fourier{g(x)} = \{d_n\}$. Potom
$\fourier f(x)g(x) = \{ \suminf{i} c_i d_{n-i} \}$
\end{veta}
\begin{dokaz}
    \begin{equation*}
    \begin{split}
      \{\fourier f(x)g(x)\}_n &= 
        \frac{1}{2\pi} \intpipi f(x)g(x) e^{-\imag x n} \dd x 
        = \frac{1}{2\pi} \intpipi g(x) e^{-\imag x n} \suminf{k} c_k
        e^{\imag k x} \dd x \\
        &= \suminf{k} c_k \frac{1}{2\pi} \intpipi f(x) e^{-\imag (n-k)
        x}
        = \suminf{k} c_k d_{n-k}
    \end{split}
    \end{equation*}
\end{dokaz}

\begin{veta}[O konvolúcii]
Nech $f(x),g(x) \in \LLab$, $\fourier{f(x)} = \{c_n\}$,
$\fourier{g(x)} = \{d_n\}$. Nech $f', g'$ sú $2\pi$-periodické
rozšírenia $f,g$. Potom $\fourier f(y)*g(y) = 
    \fourier (\intpipi f'(y)g'(x-y) \dd y) = \{ 2\pi c_n d_n \}$

\end{veta}
\begin{dokaz}
    \begin{equation*}
    \begin{split}
        \{\fourier \intpipi f'(y)g'(y-x) \dd y\}_n 
        &= \frac{1}{2\pi} \intpipi e^{-\imag n x}
            \intpipi f'(y)g'(x-y) \dd y \dd x \\
         &= \intpipi f'(y) \frac{1}{2\pi} \intpipi  g'(x-y) e^{-\imag x n}
         \dd x \dd y \\
        &=_\text{(podľa vety \ref{veta:time_shift})}
          \intpipi f'(y) d_{n} e^{-\imag n y} \dd y \\
        &= 2\pi c_n d_n
    \end{split}
    \end{equation*}
\end{dokaz}

\begin{veta}[Persevalova veta]
 Nech $f(x) \in \LLab$, $\fourier f(x) = \{ c_n\}$. Potom
 \begin{equation*}
   \frac{1}{2\pi} \intpipi |f(x)|^2 \dd x = \suminf{n} |c_n|^2
 \end{equation*}
 Persevalova veta je teda akýsi zákon zachovania - celková hodnota
 energie v priestorovej doméne je rovnaká ako celková hodnota energie
 v časovej doméne.
\end{veta}
\begin{dokaz}
    \begin{equation*}
    \begin{split}
        \frac{1}{2\pi} \intpipi |f(x)|^2 \dd x 
          &= \frac{1}{2\pi} \intpipi f(x) \conjug{f}(x) \dd x 
           = \frac{1}{2\pi} \intpipi f(x) 
            \conjug{ \suminf{n} c_n e^{\imag n x} } \dd x \\
          &= \suminf{n} \conjug{c_n} \frac{1}{2\pi}
                \intpipi f(x) e^{-\imag n x} \dd x
            = \suminf{n} \conjug{c_n} c_n \\
          &= \suminf{n} |c_n|^2
    \end{split}
    \end{equation*}
\end{dokaz}

\nocite{properties_series}
\nocite{bracewell}
