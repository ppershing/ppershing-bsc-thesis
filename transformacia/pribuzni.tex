% vim:spell spelllang=sk
\section{Príbuzné transformácie}

Fourierova transformácia súvisí s veľa rôznymi transformáciami,
postupne si ich uvedieme a porovnáme.

\subsection{Transformácie príbuzné spojitej Fourierovej transformácii}
%%% {{{ Taylorove rady
\subsubsection{Taylorove rady}
Začiatok nášho porovnávania týchto dvoch transformácii venujeme
súvislosti medzi Fourierovými a Taylorovými radmi. Prvým faktom je, že
obe sú nekonečné rady. Nie veľmi zaujímavý fakt. Hoci tieto 2
transformácie nemajú veľa spoločného, existujú isté spoločné črty.
Uvažujme Taylorov rad funkcie $f(x)$ okolo bodu 0 s polomerom konvergencie $r$
\begin{equation}
    f(x) = \sum_{n=0}^{\infty} a_n x^n, \quad -r < x < r
\end{equation}
Prirodzene, Taylorov rad sa dá napísať aj pre komplexné funkcie, v tom
prípade
\begin{equation}
    f(s e^{\imag \phi}) = \sum_{n=0}^{\infty} a_n s^n e^{\imag \phi n},
        \quad  0 \le s <r
\end{equation}
Daná rovnica je veľmi podobná rovnici \todo{} pre exponenciálnu formu
Fourierovho radu v premennej $\phi$.
Samozrejme, toto nie je je jediná spoločná črta. V \cite{taylor} sa
čitateľ môže dočítať o ich súvislostiach pri ultrasférických
polynómoch.

%%% }}}

%%% {{{ Laplace
\subsubsection{Laplaceova transformácia}
\begin{definicia}
(Obojstrannou) Laplaceovou transformáciou funkcie $f(x)$ nazveme
\begin{equation}
F(y) = \mathcal{L} f(x) = \int_{-\infty}^{\infty} e^{-xy} f(x) \dd x
\end{equation}
\end{definicia}
\begin{poznamka}
    Jednostranná Laplaceova transformácia využíva integrovanie na
    intervale $[0,\infty)$ a je v bežnej praxi používaná viacej. Na
    porovnávanie sa nám však viacej hodí uvedený variant.
\end{poznamka}
Vidíme, že Laplaceova transformácia je brat Fourierovej transformácie,
jediným rozdielom medzi nimi je absencia imaginárnej jednotky a
konštanta $2\pi$, ktorá je určená variantom Fourierovej transformácie
použitej v tejto publikácii.
Inverzná Laplaceova transformácia je podľa \cite{lpt} 
definovaná ako komplexný integrál
\begin{equation}
    f(x) = \mathcal{L}^{-1} F(y) = \frac{1}{2 \pi \imag}
        \int_{\gamma - \imag \infty}^{\gamma + \imag \infty}
        e^{x y} F(y) \dd y
\end{equation}
kde $\gamma$ je reálne číslo také, aby kontúra  integrácie bola v
regióne konvergencie $F(y)$.

Dôsledkom náramnej podobnosti oboch transformácii je veľký počet
spoločných vlastností. Obe transformácie sú lineárne. Obe sa správajú
veľmi podobne pri integrácii, derivácii, posúvaní, škálovaní a
konvolúcii.

Tabuľka \ref{tab:laplace_vs_fourier} porovnáva základné vlastnosti
Laplaceovej transformácie a Fourierovej transformácie vedľa seba. Dá
sa tak ľahko presvedčiť o ich veľkej podobnosti.

\begin{sidewaystable}[htp]
%   \centering 
    \begin{tabular}{l l l l l}
    & \multicolumn{2}{c}{Laplace doména}
    & \multicolumn{2}{c}{Fourier doména} \\
    Operácia & priestorová& frekvenčná&
        priestorová & frekvenčná \\
    Transformácia &
        $l(t)$ & $L(\omega)$ &
        $f(t)$ & $F(\omega)$ \\
    Linearita &
        $a l(t) + b g(t)$ & $a L(\omega) + b G(\omega)$ &
        $a f(t) + b g(t)$ & $a F(\omega) + b G(\omega)$ \\
    Derivácia frekvencie &
        $-t l(t)$ & $L'(\omega)$ &
        $-2\pi\imag t f(t) $ & $F'(\omega)$\\
    Derivácia v čase &
        $l'(t)$ & $\omega L(\omega) - l(0^{-})$ &
        $f'(t)$ & $2\pi\imag\omega F(\omega)$\\
    Škálovanie &
        $l(at)$ & $\frac{1}{|a|} L(\frac{\omega}{a})$ &
        $f(at)$ & $\frac{1}{|a|} F(\frac{\omega}{a})$\\
    Posúvanie &
        $e^{a t} l(t)$ & $L(\omega -a)$ &
        $e^{2\pi\imag at} f(t)$ & $L(\omega - a)$\\
    Konvolúcia &
        $l(t)*g(t)$ & $L(\omega)G(\omega)$ &
        $f(t)*g(t)$ & $F(\omega)G(\omega)$
    \end{tabular}
    \label{tab:laplace_vs_fourier}
    \caption{Porovnanie Laplacovej a Fourierovej transformácie}
\end{sidewaystable}
%%% }}}

%%% {{{ Hartley
\subsubsection{Hartleyova transformácia}
Podobne ako Fourierova transformácia používa kernel zložený zo sínusov
a kosínusov, konkrétne exponenciálny kernel $e^{-2\pi \imag\omega t}=
\cos(2\pi \omega t) -\imag \sin(2 \pi \omega t)$, Hartleyouva
transformácia používa takzvaný kosínovo-sínusový kernel
\newcommand{\cas}{{\rm cas}}
$\cas(2 \pi \omega t)=\cos(2 \pi \omega t) + \sin(2 \pi \omega t)$.
Transformácia je preto narozdiel od FT reálna.
\begin{equation}
 H(\omega) = \{\mathcal{H} f\}(\omega) =
    \intinfinf f(t) \cas(2 \pi \omega t) \dd t
\end{equation}

Hartleyova transformácia je inverzom samej seba, t.j.
\begin{equation}
    f = \{\mathcal{H}\{\mathcal{H} f\}\}
\end{equation}

Jej hlboký súvis s Fourierovou transformáciou sa dá popísať
nasledujúcimi prevodnými rovnicami medzi oboma transformáciami

\begin{equation}
    F(\omega) = \frac{H(\omega)+H(-\omega)}{2} 
                -\imag \frac{H(\omega) - H(-\omega)}{2}
\end{equation}

\begin{equation}
 H(\omega) = \Re(F(\omega)) - \Im(F(\omega))
\end{equation}
%%% }}}

%%% {{{ Fractional fourier
\subsubsection{Čiastková Fourierova transformácia}
Predstavme si nachvíľu Fourierovu transformáciu ako operátor
v rovine reprezentovanej priestorovou/časovou doménou ako osou $x$ a
frekvenčnou doménou ako osou $y$.
Jednu aplikáciu $\fourier$ môžeme chápať ako otočenie o 90 stupňov
proti smeru hodinových ručičiek.
Dve postupné aplikácie $\fourier\fourier f(x) = f(-x)$, čo nie je nič
iné ako otočenie o 180 stupňov v našej rovine. Napokon,
$\fourier\fourier\fourier\fourier f(x) = f(x)$ zodpovedajúc otočeniu o
plnú kružnicu, teda identitu.
Fourierova transformácia rádu $a$ kde $a$ je celé číslo spĺňa
nasledujúce podmienky \\
{\centering
\begin{tabular}{l l}
    konzistencia& $\fourier^a = \fourier$ pre $a=1$ \\
    aditivita& $\fourier^a \fourier^b = \fourier^{a+b}$ \\
    comutativita & $\fourier^a \fourier^b = \fourier^b \fourier^a$ \\
    linearita& $\fourier^a(f+g) = \fourier^a f + \fourier^a g$
\end{tabular}
}
\\
Preto ak by existovalo algebraické rozšírenie definície $\fourier^a$
pre $a\in \R$ spĺňajúce tieto
podmienky, mohli by sme ho považovať za čiastkovú Fourierovu
transformáciu. Takéto rozšírenie naozaj existuje.
Čiastková Fourierova transformácia je definovaná pre angulárne
frekvencie napríklad v \cite{saxsin}.
Pre $a\in[0,1]$ 
\begin{equation}
    \fourier^a f(t) = F_a(\omega) =
        \frac{e^{\imag( 1/4 \pi - 1/2 \phi)}}{\sqrt{2\pi \sin \phi}}
        \intinfinf f(t) exp\left[ 
            -\imag \frac{\omega t - \frac{1}{2}( t^2 + \omega^2)
            \cos\phi }{\sin \phi}
        \right] \dd t
    \label{eq:fractional_fourier}
\end{equation}
kde $\phi = \frac{1}{2} \pi a$.
Dosadením za $a=1$ môžeme nahliadnuť, že rovnica
\ref{eq:fractional_fourier} prejde na štandardnú Fourierovu
transformáciu v angulárnej frekvencii, čiže
\begin{equation}
    \begin{split}
    \fourier^1 f(t) &=
        \frac{e^{\imag( 1/4 \pi - 1/4 \pi)}}{\sqrt{2 \pi \sin \frac{\pi}{2}} }
        \intinfinf f(t) exp\left[ 
            -\imag \frac{\omega t - \frac{1}{2}( t^2 + \omega^2)
            \cos\frac{\pi}{2} }{\sin \frac{\pi}{2}}
        \right] \dd t \\
        &= \frac{1}{\sqrt{2 \pi}} 
            \intinfinf f(t) exp\left[ -\imag \omega t\right]
        \end{split}
\end{equation}
. Proces pre $a=0$ je bolestivý, avšak dá sa ukázať, že
$\lim_{a\imply0} \fourier^a f(t) = f(t)$.

Na ilustráciu ako sa postupne prelieva energia z priestorovej do
časovej domény sme spravili čiastkovú Fourierovu transformáciu
obdĺžnikového signálu postupne pre rôzne koeficienty $a$. Ilustrácia
je na obrázku \ref{fig:fractional_fourier_transform}.

%%% {{{ fig:fractional_fourier_transform
\begin{figure}[htp]
    \def\imagepath{obrazky/transformacia/fractional_fourier_transform}
    \centering
    \subfigure[$a=0.00$]{
        \includegraphics{\imagepath/rect_000}
    }
    \subfigure[$a=0.01$]{
        \includegraphics{\imagepath/rect_001}
    }
    \subfigure[$a=0.05$]{
        \includegraphics{\imagepath/rect_005}
    }
    \subfigure[$a=0.20$]{
        \includegraphics{\imagepath/rect_020}
    }
    \subfigure[$a=0.35$]{
        \includegraphics{\imagepath/rect_035}
    }
    \subfigure[$a=0.50$]{
        \includegraphics{\imagepath/rect_050}
    }
    \subfigure[$a=0.65$]{
        \includegraphics{\imagepath/rect_065}
    }
    \subfigure[$a=0.80$]{
        \includegraphics{\imagepath/rect_080}
    }
    \subfigure[$a=0.95$]{
        \includegraphics{\imagepath/rect_095}
    }
    \subfigure[$a=0.99$]{
        \includegraphics{\imagepath/rect_099}
    }
    \subfigure[$a=1.00$]{
        \includegraphics{\imagepath/rect_100}
    }
    \label{fig:fractional_foruier_transform}
    \caption{Ilustrácia čiastkovej FT pre rôzne koeficienty $a$}
\end{figure}
%%% }}}

Čiastková Fourierova transformácia je teda generalizáciou normálnej
transformácie. Má aj veľa vlastností, ktoré sme už poznali z klasickej
FT. To čo je nové, a pritom veľmi zaujímavé je plynulý prechod medzi
priestorovou a frekvenčnou doménou. Čiastková FT našla preto svoje
miesto v optike a pri riešení diferenciálnych
rovníc vo vlnovej fyzike.
%%% }}}

\subsection{Transformácie príbuzné diskrétnej Fourierovej
transformácii}
Tak ako spojitá FT má svojich príbuzných, nie je tomu inak ani pre jej
diskrétnu verziu. Samozrejme, poniektoré z predtým uvedených
transformácii majú aj svoje diskrétne verzie, my sa však budeme viacej
venovať dvom špeciálnym transformáciam.

\subsubsection{Z-transformácia ako rozšírenie DFT}
Zaujímavým prirodzeným rozšírením DTFT (Diskrétne časovej Fourierovej
transformácie) a DFT je z-transformácia. Obojstranná verzia je
napísaná v rovnici \ref{eq:z_transform_bilateral}, jednostranná verzia
v rovnici \ref{eq:z_transform_unilateral}.
\begin{eqnarray}
    \label{eq:z_transform_bilateral}
    X(z) &= \mathcal{Z}\{x\} &= \sum_{n=-\infty}^{\infty} x_n z^{-n} \\
    \label{eq:z_transform_unilateral}
    X(z) &= \mathcal{Z}\{x\} &= \sum_{n=0}^{\infty} x_n z^{-n}
\end{eqnarray}
Môžeme si všimnúť, že ak $|z|=1$, potom z-transformácia prechádza do
Fourierovej transformácie.
Špeciálne, pre nás je veľmi zaujímavá je konečná verzia
\begin{equation}
 X(z) = \sum_{n=0}^{N-1} x_n z^{-n}
\end{equation}
pretože je použitá práve v Bluesteinovom algoritme na výpočet FFT.
Bluesteinov chirp z-algoritmus je celý navrhnutý na počítanie
ľubovoľnej z-transformácie. Je preto ešte užitočnejší ako FFT. Rôzne
aplikácie z-transformácie zahrňujú analýzu signálov s lineárnou zmenou
frekvencie, ako sa dá nájsť napríklad v \cite{nasa}.

\subsubsection{Diskrétna kosínová transformácia}
Na záver kapitoly sme si nechali asi najznámejšieho príbuzného
diskrétnej Fourierovej transformácie. Diskrétna kosínová transformácia
(ďalej len DCT) je v súčasnej informatike používaná snáď ešte
výraznejšie ako DFT. Je to spôsobené jej niektorými príjemnými
vlastnosťami. Asi najvýznamnejšou je že je to čisto reálna
transformácia a nie je nutné preto počítať s komplexnými číslami.
Druhý dôvod je koncentrácia energie okolo menších frekvencií, ako si
ukážeme pri kompresii obrazu a zvuku.
Narozdiel od DFT, DCT používa za svoje základné funkcie iba (posunuté) kosínusy.
Pretože je to používaná transformácia, má minimálne 4 veľmi známe
varianty. My si uvedieme pravdepodobne tú najpoužívanejšiu, jej vzorec
je \ref{eq:dct_transform}

\begin{equation}
    X_k = \sum_{n=0}^{N-1} x_n \cos (\frac{\pi}{N} (n + \frac{1}{2}) k),
    \quad k=0,1,\dots,N-1
    \label{eq:dct_transform}
\end{equation}
Zaujímavou vlastnosťou je, že DCT je až na faktor 2 presne rovná DFT z
$4N$ reálnych čísel s párnou symetriou, kde vstupy na párnych indexoch
sú nulové. Na výpočet DCT sa teda teoreticky dá priamo aplikovať DFT
algoritmus, stačí mu predpracovať vstup na symetrický iba s nepárnymi
indexami. Táto spojitosť však zachádza ďalej - algoritmy použité na
výpočet DFT idú spravidla jednoducho prepísať a zjednodušiť priamo na
DCT, čím sa transformácia ukazuje v lukratívnom svetle. Napokon - jej
využitia budú spomínané v nasledujúcej kapitole.

\todo{citacie}
\nakedcite{wiki:laplace_transform}
\nakedcite{wiki:hartley_transform}
\nakedcite{wiki:z_transform}
\nakedcite{wiki:dct}
\nakedcite{bracewell}
\nakedcite{wolfram:z)transform}
