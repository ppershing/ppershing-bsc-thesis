% vim:spell spelllang=sk
\section{Príbuzné transformácie}

Fourierova transformácia súvisí s veľa rôznymi transformáciami,
postupne si ich uvedieme a porovnáme.

%%% {{{ Taylorove rady
\subsubsection{Taylorove rady}
Začiatok nášho porovnávania týchto dvoch transformácii venujeme
súvislosti medzi Fourierovými a Taylorovými radmi. Prvým faktom je, že
obe sú nekonečné rady. Nie veľmi zaujímavý fakt. Hoci tieto 2
transformácie nemajú veľa spoločného, existujú isté spoločné črty.
Uvažujme Taylorov rad funkcie $f(x)$ okolo bodu 0 s polomerom konvergencie $r$
\begin{equation}
    f(x) = \sum_{n=0}^{\infty} a_n x^n, \quad -r < x < r
\end{equation}
Prirodzene, Taylorov rad sa dá napísať aj pre komplexné funkcie, v tom
prípade
\begin{equation}
    f(s e^{\imag \phi}) = \sum_{n=0}^{\infty} a_n s^n e^{\imag \phi n},
        \quad  0 \le s <r
\end{equation}
Daná rovnica je veľmi podobná rovnici \todo{} pre exponenciálnu formu
Fourierovho radu v premennej $\phi$.
Samozrejme, toto nie je je jediná spoločná črta. V \cite{taylor} sa
čitateľ môže dočítať o ich súvislostiach pri ultrasférických
polynómoch.

%%% }}}

\subsubsection{Laplacova transformácia}
\begin{definicia}
(Obojstrannou) Laplacovou transformáciou funkcie $f(x)$ nazveme
\begin{equation}
F(y) = \mathcal{L} f(x) = \int_{-\infty}^{\infty} e^{-xy} f(x) \dd x
\end{equation}
\end{definicia}
\begin{poznamka}
    Jednostranná Laplacova transformácia využíva integrovanie na
    intervale $[0,\infty)$ a je v bežnej praxi používaná viacej.
\end{poznamka}
Vidíme, že Laplacova transformácia je brat Fourierovej transformácie,
jediným rozdielom medzi nimi je absencia imaginárnej jednotky.
Inverzná Laplacova transformácia je podľa \cite{lpt} 
definovaná ako komplexný integrál
\begin{equation}
    f(x) = \mathcal{L}^{-1} F(y) = \frac{1}{2 \pi \imag}
        \int_{\gamma - \imag \infty}^{\gamma + \imag \infty}
        e^{x y} F(y) \dd y
\end{equation}
kde $\gamma$ je reálne číslo také, aky kontúra  integrácie bola v
regióne konvergencie $F(y)$.

Dôsledkom náramná podobnosť oboch transformácii je veľký počet
spoločných vlastností. Obe transformácie sú lineárne. Obe sa správajú
veľmi podobne pri integrácii, derivácii, posúvaní, škálovaní a

konvolúcii.
Tabuľka \ref{tab:laplace_vs_fourier} porovnáva základné vlastnosti Laplacovej
transformácie a Fourierovej transformácie vedľa seba.

\todo{Tabulku doplnit ...}

\begin{table}[htp]
   \centering 
    \begin{tabular}{l l l l l}
    Operácia & Time domain Laplace & Frequency domain Laplace&
        Time domain Fourier & Frequency domain Fourier \\
        Linearita & $a f(t) + b g(t)$ & $a F(\omega) + b G(\omega)$ &
                    $a f(t) + b g(t)$ & $a F(\omega) + b G(\omega)$ \\
        Derivácia frekvencie & $t f(t)$ & $- F'(\omega)$ \\
        Derivácia v čase & $f'(t)$ & $\omega F(\omega) - f(0^{-})$ \\
        Škálovanie & $f(at)$ & $\frac{1}{|a|} F(\frac{\omega}{a})$ \\
        Posúvanie & $e^{a t}$ & $F(\omega -a)$ \\
        Konvolúcia & $f(t)*g(t)$ & $F(\omega)G(\omega)$ &
                     $f(t)*g(t)$ & $F(\omega)G(\omega)$
    \end{tabular}
    \label{tab:laplace_vs_fourier}
    \caption{Porovnanie Laplacovej a Fourierovej transformácie}
\end{table}
