% vim:spell spelllang=sk
\section{Fourierova transformácia}
Po pomerne zdĺhavej rozprave o Fourierových radoch teraz konečne
prejdeme na hlavnú tému tejto práce a to Fourierovu transformáciu.
Nebudeme do detailov rozoberať jej vlastnosti a robiť podrobné dôkazy,
pretože väčšina vecí sa dá odvodiť analogicky ako sme ich odvodzovali
u radov, avšak treba využívať matematický aparát s väčšou
obozretnosťou. Najskôr si popíšeme spojitú Fourierovu transformáciu a
v druhej časti prejdeme na diskrétnu verziu, ktorá bude hlavným
predmetom skúmania v kapitole o aplikáciach.

\subsection{Spojitá Fourierova transformácia}
Prvou prekážkou, ktorá bude pôsobiť pri generalizovaní Fourierových
radov na interval $(-\infty,\infty)$ je absencia ortogonálnej sady
vektorov. Totižto, pre každý periodickú funkciu $f(x)\not=0$ na
množine nenulovej miery platí $\intinfinf f(x)^2 \dd x = \infty$.
Preto nie je možné preniesť dôkazy konvergencie z vektorového
priestoru $\LLab$. Napriek tomu ale vieme Fourierove rady zovšeobecniť
na celý ínterval $(-\infty,\infty)$.
\begin{definicia}
    Fourierovou transformáciou intergovaťeľnej funkcie $f(x)$ nazveme
    funkciu $\dual{f}$ definovanú ako
    \begin{equation}
        \dual{f}(\omega) = \intinfinf f(x) e^{-\imag \omega x} \dd x
    \end{equation}
    \label{def:fourier_transform}
\end{definicia}
Ak $f$ je integrovateľná (t.j $\intinfinf |f(x)| < \infty$), potom jej
Fourierova transformácia $\dual{f}(\omega)$ je ohraničená, ale nie nutne
integrovateľná. Ako príklad môže slúžiť obdĺžniková funkcia 
\begin{equation}
    f(x) = \left\{  \begin{array}{l l}
                        0&x <-\pi, x>\pi \\
                        \frac{1}{2\pi}& -\pi<x<\pi
                    \end{array}
            \right.
\end{equation}
, ktorej Fourierova transformácia je $\sinc{\pi \omega}$. Táto funkcia
nie je integrovateľná hoci jej integrál $\intinfinf \sinc(x) \dd x = 1$.
Je to analógia striedavého harmonického radu, ktorý konverguje, ale
rad jeho absolútnych hodnôt diverguje.
Preto nie je možné vo všeobecnosti napísať inverznú transformáciu.
Avšak, ak $f(x)$ aj $\dual{f}(\omega)$ sú obe integrovateľné, potom platí
\begin{equation}
    f(x) = \frac{1}{2\pi} \intinfinf 
        \dual{f}(\omega) e^{\imag \omega x} \dd \omega
    \label{eq:inverse_fourier}
\end{equation}
skoro všade. Ak $f(x)$ je spojitá, potom rovnica
\ref{eq:inverse_fourier} platí na celom $\R$.

Okrem definície \ref{def:fourier_transform} sa v praxi používajú aj
minimálne 2 ďalšie varianty. Uvedieme tu variantu v štandardnej
frekvencii (Definícia \ref{def:fourier_transform} je v takzvanej
angulárnej frekvencii).
\begin{eqnarray}
    \dual{f}(\omega) &= \intinfinf f(x) e^{-2\pi\imag x \omega} \dd x\\
    f(x) &= \intinfinf \dual{f}(\omega) e^{2\pi\imag x \omega} \dd
    \omega
\end{eqnarray}

Podobne ako pri radoch, aj pri Fourierovej transformácii máme na výber
z veľkého počtu vlastností. Ich zhrnutie sa dá nájst v tabuľke
\ref{tab:fourier_transform_vlastnosti}. Zhrnutie základných
transformačných párov je uvedené v tabuľke
\ref{tab:fourier_transform_pairs}. Obe tabuľky sú prevzaté z
\cite{wiki:fourier_transform}

\begin{table}[htp]
    \centering
    \begin{tabular}{llll}
    Vlastnosť&Funkcia&FT, obyčajná frekvencia&FT, angulárna
    frekvencia\\
    Transformácia&$f(x)$&
        $\dual{f}(\omega) = \intinfinf f(x) e^{-2\pi\imag x\omega} \dd x$&
        $\dual{f}(\omega) = \intinfinf f(x) e^{-\imag x \omega} \dd x$\\
    Linearita&$a f(x) + b g(x)$&
        $a \dual{f}(\omega) + b \dual{g}(\omega)$&
        $a \dual{f}(\omega) + b \dual{g}(\omega)$\\
    Posuv v čase&$f(x-a)$&
        $e^{-2\pi\imag a \omega} \dual{f}(\omega)$&
        $e^{-\imag a \omega} \dual{f}(\omega)$\\
    Posuv vo frekvencii&$ e^{2\pi\imag a x} f(x)$&
        $\dual{f}(\omega-a)$&
        $\dual{f}(\omega-2\pi a)$\\
    Škálovanie v čase&$ f(a x)$&
        $\frac{1}{|a|}\dual{f}(\frac{\omega}{a})$&
        $\frac{1}{|a|}\dual{f}(\frac{\omega}{a})$\\
    Dvojitá transformácia&$ \dual{f}(x)$&
        $f(-\omega)$&
        $2\pi f(-\omega)$\\
    Diferencovanie v čase&$ f'(x)$&
        $2\pi\imag\omega \dual{f}(\omega)$&
        $\imag\omega \dual{f}(\omega)$\\
    Diferencovanie vo frekvencii&$x f(x)$&
        $\frac{\imag}{2\pi} \dual{f}'(\omega)$&
        $\imag \dual{f}'(\omega)$\\
    Konvolúcia&$ (f*g)(x)$&
        $\dual{f}(\omega) \dual{g}(\omega)$&
        $\dual{f}(\omega) \dual{g}(\omega)$\\
    Modulácia&$ f(x)g(x)$&
        $(\dual{f}*\dual{g})(\omega)$&
        $\frac{1}{2\pi}(\dual{f}*\dual{g})(\omega)$\\
    Parsevalova veta&
      $\intpipi f(x)\conjug{g(x)} \dd x=$&
      $=\intpipi \dual{f}(\omega) \conjug{\dual{g}(\omega)} \dd\omega$&
      $=\frac{1}{4\pi^2}\intpipi \dual{f}(\omega) \conjug{\dual{g}(\omega)}
      \dd\omega$\\
    \end{tabular}
    \caption{Vlastnosti Fourierovej transformácie}
    \label{tab:fourier_transform_vlastnosti}
\end{table}

\begin{table}[htp]
    \centering
    \begin{tabular}{lll}
    Funkcia&FT štandardná frekvencia&FT angulárna frekvencia\\
    $1$&$\delta(\omega)$&$2\pi\delta(\omega)$\\
    $\delta(\omega)$&$1$&$1$\\
    $\text{rect}(ax)$&$\frac{1}{|a|}\sinc(\frac{\omega}{a})$&
               $\frac{1}{|a|}\sinc(\frac{\omega}{2\pi a})$\\
    $\sinc(ax)$&$\frac{1}{|a|}\text{rect}(\frac{\omega}{a})$&
                $\frac{1}{|a|}\text{rect}(\frac{\omega}{2\pi a})$\\
    $\text{tri}(ax)$&$\frac{1}{|a|}\sinc^2(\frac{\omega}{a})$&
               $\frac{1}{|a|}\sinc^2(\frac{\omega}{2\pi a})$\\
    $\sinc^2(ax)$&$\frac{1}{|a|}\text{tri}(\frac{\omega}{a})$&
                $\frac{1}{|a|}\text{tri}(\frac{\omega}{2\pi a})$\\
    $e^{-ax^2}$&$\sqrt{\frac{\pi}{a}} e^{-\frac{(\pi\omega)^2}{a}}$&
                $\sqrt{\frac{\pi}{a}} e^{-\frac{\omega^2}{4a}}$\\
    $e^{-a|x|}$& $\frac{2a}{a^2 + 4\pi\omega^2}$&
                 $\frac{2a}{a^2 + \omega^2}$\\
    $e^{\imag a x}$&$\delta{\omega-\frac{a}{2\pi}}$&
                 $2\pi\delta{\omega-a}$\\
    $\cos(ax)$&$\frac{\delta(\omega-\frac{a}{2\pi})+
                      \delta(\omega+\frac{a}{2\pi})}{2}$&
                $\pi (\delta(\omega-a)+\delta(\omega+a))$\\
    $\sin(ax)$&$\imag\frac{\delta(\omega+\frac{a}{2\pi})-
                      \delta(\omega-\frac{a}{2\pi})}{2}$&
                $\imag\pi (\delta(\omega+a)-\delta(\omega-a))$\\
    $\frac{1}{x}$&$-\imag\pi\sgn(\omega)$&
                  $-\imag\pi\sgn(\omega)$
    \end{tabular}
    \caption{Vybraté funkcie a ich Fourierova transformácia}
    \label{tab:fourier_transform_pairs}
\end{table}
\todo{sinc je sin(pix)/pix}


\subsection{Diskrétna Fourierova transformácia}

Diskrétna Fourierova transformácia narozdiel od spojitej verzie operuje nad
periodickým signálom, ktorý je zadaný v diskrétnych bodoch.

\begin{poznamka}
    Existuje takzvaná "Diskrétna časová Fourierova transformácia",
    ktorá pracuje s neperiodickým signálom zadanom v diskrétnych
    bodoch, tejto sa avšak nebudeme venovať.
\end{poznamka}

Jej základom je ortogonalita postupností $\{\sin 2\pi n
k\}_{n=0}^{N-1}$ a $\{\cos 2\pi n k\}_{n=0}^{N-1}$. Nakoľko spomínané
postupnosti vieme vyjadriť ako vektory vo vektorovom priestore
$\C^N$, budú všetky vlastnosti odvodené pre Fourierove rady (okrem
vlastností týkajúcich sa kalkulu) analogicky prenesiteľné na diskrétnu
FT. Aby sme nenapínali, v nasledujúcej definícii si transformáciu aj
zadefinujeme.

\begin{definicia}[Diskrétna Fourierova transformácia]
        Diskrétnou Fourierovou transformáciou veľkosti $N$ nazveme 
        transformáciu $\C^N\imply\C^N$ pre ktorú platí
        \begin{equation}
            X_k = \sum_{n=0}^{N-1} x_n e^{\frac{-2\pi\imag}{N} k n}
        \end{equation}
\end{definicia}

Inverzná transformácia je definovaná ako
\begin{equation}
    x_k = \frac{1}{N} \sum_{n=0}^{N-1} X_n e^{\frac{2\pi\imag}{N} kn}
\end{equation}

Narozdiel od Fourierovej transformácie a Fourierových radov, diskrétna
verzia je vždy invertovateľná - nekladieme žiadne špeciálne nároky na
vstupné údaje.
V tabuľke \ref{tab:dft_properties} prevzanej z \cite{dft_properties}
sú zhrnuté najdôležitejšie vlastnosti diskrétnej Foruierovej
transformácie.
\begin{table}
    \centering
    \begin{tabular}{lll}
    Operácia&Priestorová/časová doména&Frekvenčná doména\\
    transformácia&$x_n$&$X_k$\\
    Rotácia&$x_{n-m \mod N}$&$X_k e^{-\frac{2 \pi\imag}{N} k m}$\\
    Časový reverz&$x_{-n \mod N}$&
        $X_{(-k) \mod N}$\\
    Konjugácia&$\conjug{x_n}$&
        $\conjug{X_{(-k) \mod N}}$\\
    Cyklická konvolúcia&
        $(x*y)_n= \sum_{m=0}^{N-1} x_m y_{(n-m)\mod N}$&
        $X_k Y_k$\\
    Modulácia&
        $x_n y_n$& $ (X*Y)_k$\\
    Symetria&
        $\forall n:x_n\in R$&
        $X_k = \conjug{X_{(-k)\mod N}}$\\
    Parsevalova veta&
        $\sum_{n=0}^{N-1}( |x_n|^2)=$&
        $=\frac{1}{N} \sum_{k=0}^{N-1}(|X_k|^2)$
    \end{tabular}

    \caption{Vlastnosti diskrétnej Fourierovej transformácie}
    \label{tab:dft_properties}

\end{table}
